\vspace{-0.2em}
\section{The \tuner tuner}
\label{sec:sync_tuner}
\vspace{-0.2em}
%In this section we describe \tuner, our tuner for momentum SGD.
%We introduce a noisy quadratic model and work on a local quadratic approximation of $f(x)$ to apply the tuning rule of \eqref{eqn:noiseless_tuning_rule} to SGD on an arbitrary objective.
%\tuner is our implementation of that rule.

Here we describe our tuner for momentum SGD that uses the same learning rate for all variables.
We first introduce a noisy quadratic model $f(x)$ as the local approximation of an arbitrary one-dimensional objective. On this approximation, we extend the tuning rule of \eqref{eqn:noiseless_tuning_rule} to SGD. In section~\ref{sec:tuner}, \emph{we generalize the discussion to multidimensional objectives; it yields the \tuner tuning rule}.


\paragraph{Noisy quadratic model}
\label{sec:noisy_quadratics}

\newcommand{\oac}{origin-adjusted curvature }
\newcommand{\bx}{\bar{x}}


We consider a scalar quadratic 
\begin{equation}
	f(x) = \frac{h}{2} x^2 + C 
	= \sum_i \frac{h}{2n}(x-c_i)^2
	\triangleq \frac{1}{n} \sum_i f_i(x)
%	\quad \sum_i c_i = 0.
	\label{eqn:noise_quad_1d}
\end{equation}
with $\sum_i c_i = 0$. $f(x)$ is a quadratic approximation of the original objectives with $h$ and $C$ derived from measurement on the original objective. The function $f(x)$ is defined as the average of $n$ {\em component functions}, $f_i$.
This is a common model for SGD, where we use only a single data point (or a mini-batch) drawn uniformly at random, $S_t \sim \mathrm{Uni}([n])$ to compute a noisy gradient, $\nabla f_{S_t}(x)$, for step $t$.
Here, $C=\frac{1}{2n}\sum_i h c_i^2$ denotes the {\em gradient variance}.
As optimization on quadratics decomposes into scalar problems along the principal eigenvectors of the Hessian, the scalar model in~\eqref{eqn:noise_quad_1d} is sufficient to study local quadratic approximations of multidimensional objectives.
Next we get an {\em exact} expression for the mean square error after running momentum SGD on the scalar quadratic in~\eqref{eqn:noise_quad_1d} for $t$ steps.



\begin{lemma}
\label{lem:main_lemma}
Let $f(x)$ be defined as in \eqref{eqn:noise_quad_1d},
$x_1=x_0$ and $x_t$ follow the momentum update \eqref{eqn:momentum_gd} with stochastic gradients $\nabla f_{S_t}(x_{t-1})$ for $t \geq 2$.
Let $\mat{e}_1=[1, 0]^T$, the expectation of squared distance to the optimum $x^*$ is
	\begin{equation}
	\begin{aligned}
		\E (x_{t+1} - x^{*})^2 & = (\mat{e}^{\top}_1 \mat{A}^t [x_1 - x^{*}, x_0-x^{*}]^{\top})^2 \\
		& + \alpha^2 C \mat{e}^{\top}_1 (\mat{I} - \mat{B}^t)(\mat{I} - \mat{B})^{-1}\mat{e}_1	,
		\label{equ:squared_dist_exact}	
	\end{aligned}
	\end{equation}
where the first and second term correspond to squared bias 
and variance, and their corresponding momentum dynamics are captured by operators
	\begin{equation}
	\begin{gathered}
		\mat{A} = \begin{bmatrix}
		1-\alpha h + \mu & - \mu\\
		1 & 0 \\
		\end{bmatrix}, \\
%		\quad
		\mat{B} = 
		\begin{bmatrix}
		(1-\alpha h + \mu)^2 &  \mu^2 & -2\mu(1-\alpha h + \mu)\\
		1 & 0 & 0 \\
		1-\alpha h + \mu & 0 & - \mu
		\end{bmatrix}.
		\label{equ:mat_def}
	\end{gathered}
	\end{equation}
\end{lemma}

\yell{
Even though it is possible to numerically work on~\eqref{equ:squared_dist_exact} directly,
we use a scalar, asymptotic surrogate in~\eqref{eqn:asymptotic_surrogate} based on the spectral radii of operators to simplify analysis and expose insights.
This decision is supported by our findings in Section~\ref{sec:momentum_operator}: the spectral radii can capture empirical convergence rate.}
\begin{equation}
\begin{aligned}
%	\E ( x_{t+1} - x^{*} )^2
%	\approx  \rho(\mat{A})^{2t} ( x_0 - x_{*} )^2 
%		+ (1-\rho(\mat{B})^{t}) \frac{\alpha^2 C}{1-\rho(\mat{B})}
	&\E ( x_{t+1} - x^{*} )^2 \\
	\approx & \rho(\mat{A})^{2t} ( x_0 - x_{*} )^2 
		+ (1-\rho(\mat{B})^{t}) \frac{\alpha^2 C}{1-\rho(\mat{B})}
	\label{eqn:asymptotic_surrogate}
\end{aligned}
\end{equation}

One of our design decisions for \tuner 
is to always work in the robust region of Lemma~\ref{lem:robustness}.
We know that this implies a spectral radius $\sqrt{\mu}$ of the momentum operator, $\mat{A}$, for the bias. 
Lemma~\ref{lem:spectral_var_control} shows that under the exact same condition, the variance operator $\mat{B}$ has spectral radius $\mu$.
 

\begin{lemma}
\label{lem:spectral_var_control}
The spectral radius of the variance operator, $\mat{B}$ is $\mu$, if ${(1-\sqrt{\mu})^2} \leq  \alpha h \leq {(1+\sqrt{\mu})^2}$.
\end{lemma}

As a result, the surrogate objective of \eqref{eqn:asymptotic_surrogate}, takes the following form in the robust region.  
\begin{equation}
	\E ( x_{t+1} - x^{*} )^2 
	\approx \mu^t ( x_0 - x^{*} )^2
		+ (1-\mu^t) \frac{\alpha^2 C}{1-\mu}
	\label{eqn:noisy_square_dist}
\end{equation}
We extend this surrogate to multidimensional cases to extract a noisy tuning rule for \tuner.

%\vspace{-0.25em}
\subsection{Tuning rule}
\label{sec:tuner}
\vspace{-0.25em}

In this section, we present \textsc{SingleStep}, the tuning rule of YellowFin (Algorithm~\ref{alg:basic-algo}). Based on the surrogate in~\eqref{eqn:noisy_square_dist}, \textsc{SingleStep} is a multidimensional SGD version of the noiseless tuning rule in~\eqref{eqn:noiseless_tuning_rule}. We first generalize~\eqref{eqn:noiseless_tuning_rule} and~\eqref{eqn:noisy_square_dist} to multidimensional cases, and then discuss \textsc{SingleStep}.% in details.

As discussed in Section~\ref{sec:robust_properties}, GCN $\nu$ captures the dynamic range of generalized curvatures in a one-dimensional objective with varying curvature. The consequent robust region described by~\eqref{eqn:noiseless_tuning_rule} implies homogeneous spectral radii. 
%\emph{In multidimensional cases, we use a single learning rate and momentum for the entire model.} 
On a multidimensional non-convex objective, each one-dimensional slice passing a minimum $x^*$ can have \emph{varying curvature}. As we use \emph{a single $\mu$ and $\alpha$ for the entire model}, if $\nu$ simultaneously captures the dynamic range of generalized curvature over all these slices, $\mu$ and $\alpha$ in~\eqref{eqn:noiseless_tuning_rule} are in the robust region for all these slices. This implies homogeneous spectral radii $\sqrt{\mu}$ according to Lemma~\ref{lem:robustness}, empirically facilitating convergence at a common rate along all the directions. 

Given homogeneous spectral radii $\sqrt{\mu}$ along all directions, the surrogate in~\eqref{eqn:noisy_square_dist} generalizes on the local quadratic approximation of multiple dimensional objectives. On this approximation with minimum $x^*$, the expectation of squared distance to $x^*$, $\E \| x_0 - x^*\|^2$, decomposes into independent scalar components along the eigenvectors of the Hessian. We define gradient variance $C$ as the sum of gradient variance along these eigenvectors. The one-dimensional surrogates in~\eqref{eqn:noisy_square_dist} for the independent components sum to $\mu^t\| x_0 - x^* \|^2 + (1-\mu^t)\alpha^2 C / (1 - \mu)$, the \emph{multidimensional surrogate} corresponding to the one in~\eqref{eqn:noisy_square_dist}. 
%Given homogeneous spectral radii $\sqrt{\mu}$ along all directions, the surrogate in~\eqref{eqn:noisy_square_dist} generalizes via multidimensional local quadratic approximations. Assuming the quadratic approximation aligns with standard axes, the expectation of squared distance to the optimum of the approximation, $\E\|x_t - x^*\|^2$, decomposes along the standard axes. As the initial squared distance $\| x_0 - x^*\|^2$ and gradient variance $C$ also decompose along the axes, the one dimensional surrogates along axes sums to $\mu^t\| x_0 - x^* \|^2 + (1-\mu^t)\alpha^2 C / (1 - \mu)$, the surrogates corresponding to the one in~\eqref{eqn:noisy_square_dist}. 
%	Note for quadratic approximation not aligned with the axes, this \emph{multidimensional surrogates} is attained by decomposing along eigenvectors of the quadratic approximation's Hessian, instead of standard axes. 

%%\begin{minipage}{0.5\linewidth}
%%\vspace{-0.25em}
%\begin{equation}
%\begin{aligned}
%	\textsc{(SingleStep)} \notag \\
%	 \mu_t, \alpha_t & = && \arg \min_{\mu} \mu D^2
%		+ \alpha^2 C \\
%	s.t. & &&\mu \geq \left(\frac{\sqrt{h_{\max}/h_{\min} }-1}{\sqrt{h_{\max}/h_{\min}}+1}\right)^2 \\
%	& &&\alpha = \frac{(1-\sqrt{\mu})^2}{h_{\min}}
%\end{aligned}
%\label{equ:noisy_min}
%\end{equation}
%%\end{minipage}
%\begin{minipage}{0.025\linewidth}
%\ 
%\end{minipage}
%\begin{minipage}{0.45\linewidth}
%\vspace{-1.5em}
%\begin{algorithm}[H]
%	\footnotesize
%	\caption{\jianedits{\tuner}}
%	\begin{algorithmic}
%	\State \textbf{state: } $\alpha \gets 1.0$, $\mu \gets 0.0$%, $w\gets20$
%	\Function{\tuner}{$\text{gradient } g_t$, $\beta$}
%	\State $h_{\max}, h_{\min} \gets \Call{CurvatureRange}{g_t, \beta}$
%	\State $C \gets \Call{Variance}{g_t, \beta}$ 
%	\State $D \gets \Call{Distance}{g_t, \beta}$ 
%
%	\State $\mu_t, \alpha_t \gets \Call{SingleStep}{C, D, h_{\max}, h_{\min}}$
%	\State $\mu \gets \beta \cdot \mu + (1 - \beta) \cdot \mu_t$
%	\State $\alpha \gets \beta \cdot \alpha + (1 - \beta) \cdot \alpha_t$ %\Comment{Smoothing learning rate and momentum for stable control}
%	\Return $\mu, \alpha$
%	\EndFunction
%	\end{algorithmic}
%	\label{alg:basic-algo}
%\end{algorithm}
%\end{minipage}
%%%%%%%%%%%%%%%%%%%%%%%%% new algorithm %%%%%%%%%%%%%%%%%%%%%%%%
%%\begin{minipage}{0.475\linewidth}
%%\vspace{-0.25em}
%\begin{algorithm}[H]
%	\footnotesize
%	\caption{\jianedits{\tuner}}
%	\begin{algorithmic}
%%	\State \textbf{state: } $\alpha \gets 1.0$, $\mu \gets 0.0$%, $w\gets20$
%	\Function{\tuner}{$\text{gradient } g_t$, $\beta$}
%	\State $h_{\max}, h_{\min} \gets \Call{CurvatureRange}{g_t, \beta}$
%	\State $C \gets \Call{Variance}{g_t, \beta}$ 
%	\State $D \gets \Call{Distance}{g_t, \beta}$ 
%
%	\State $\mu_t, \alpha_t \gets \Call{SingleStep}{C, D, h_{\max}, h_{\min}}$
%%	\State $\mu_t, \alpha_t \gets \Call{SingleStep}{C, D, h_{\max}, h_{\min}}$
%%	\State $\mu \gets \beta \cdot \mu + (1 - \beta) \cdot \mu_t$
%%	\State $\alpha \gets \beta \cdot \alpha + (1 - \beta) \cdot \alpha_t$ %\Comment{Smoothing learning rate and momentum for stable control}
%	\Return $\mu_t, \alpha_t$
%	\EndFunction
%	\end{algorithmic}
%	\label{alg:basic-algo}
%\end{algorithm}
%%\end{minipage}
%\begin{minipage}{0.475\linewidth}
%\vspace{-0.25em}
\vspace{-0.25em}
\begin{algorithm}[h]
%	\footnotesize
	\caption{\jianedits{\tuner}}
	\begin{algorithmic}
%	\State \textbf{state: } $\alpha \gets 1.0$, $\mu \gets 0.0$%, $w\gets20$
	\Function{\tuner}{$\text{gradient } g_t$, $\beta$}
	\State $h_{\max}, h_{\min} \gets \Call{CurvatureRange}{g_t, \beta}$
	\State $C \gets \Call{Variance}{g_t, \beta}$ 
	\State $D \gets \Call{Distance}{g_t, \beta}$ 

	\State $\mu_t, \alpha_t \gets \Call{SingleStep}{C, D, h_{\max}, h_{\min}}$
%	\State $\mu_t, \alpha_t \gets \Call{SingleStep}{C, D, h_{\max}, h_{\min}}$
%	\State $\mu \gets \beta \cdot \mu + (1 - \beta) \cdot \mu_t$
%	\State $\alpha \gets \beta \cdot \alpha + (1 - \beta) \cdot \alpha_t$ %\Comment{Smoothing learning rate and momentum for stable control}
	\Return $\mu_t, \alpha_t$
	\EndFunction
	\end{algorithmic}
	\label{alg:basic-algo}
\end{algorithm}
\vspace{-0.25em}
%\end{minipage}
%%%%%%%%%%%%%%%%%%%%%%%%%%%%%%%%%%%%%%%%%%%%%%%%%%%%%%%%%%%%%%%%%%%%%%%
%%%%%%%%%%%%%%%%%%%%%%%%%%% new algorithm %%%%%%%%%%%%%%%%%%%%%%%%%%%%%%%%%%%%%%%%
\begin{table*}[t]
\begin{minipage}{0.37\textwidth}
\vspace{-1em}
\algrenewcommand\alglinenumber[1]{\scriptsize #1:}
	\begin{algorithm}[H]
	\small
	\setstretch{1.01}
	\caption{\small Curvature range}
	\begin{algorithmic}
		\State \textbf{state: } $h_{\max}$, $h_{\min}$, $h_i, \forall i \in\{1,2,3,...\}$
		\Function{CurvatureRange}{gradient $g_t$, $\beta$}
			\State $h_t \gets \| g_t \|^2$
			\State $h_{\max,t}\gets\!\!\!\max\limits_{t - w \leq i \leq t}\!h_i$, $h_{\min,t}\gets\!\!\!\min\limits_{t - w \leq i \leq t}\!h_i$
			\State $h_{\max} \gets \beta \cdot h_{\max} + (1 - \beta) \cdot h_{\max,t}$ %\hfill Smoothed largest curvature.
			\State $h_{\min} \gets \beta \cdot h_{\min} + (1 - \beta) \cdot h_{\min,t}$ %\hfill Smoothed smallest curvature.
%			\State $h_{\max} \gets \beta \  h_{\max} + (1 - \beta) \  h_{\max,t}$ %\hfill Smoothed largest curvature.
%			\State $h_{\min} \gets \beta \ h_{\min} + (1 - \beta) \ h_{\min,t}$ %\hfill Smoothed smallest curvature.
			\Return $h_{\max}$, $h_{\min}$
		\EndFunction
	\end{algorithmic}
	\label{alg:curv_func}
	\end{algorithm}
\end{minipage}
\begin{minipage}{0.315\textwidth}
\vspace{-1em}
\algrenewcommand\alglinenumber[1]{\scriptsize #1:}
	\begin{algorithm}[H]
	\small
	\setstretch{1.5}
	\caption{\small Gradient variance}
	\begin{algorithmic}
	\State \textbf{state: } $\overline{g^2}\gets0$, $\overline{g}\gets0$
	\Function{Variance}{gradient $g_t$, $\beta$}
		\State $\overline{g^2}\gets\beta \cdot \overline{g^2} + (1 - \beta) \cdot g_t \odot g_t$
		\State $\overline{g}\gets\beta \cdot \overline{g} + (1 - \beta) \cdot g_t$
%		\Return $\| \overline{g^2} - \overline{g}^2 \|_1$ %\hfill Sum of elements in the vect
		\Return $\bm{1}^T\!\!\cdot\left(\overline{g^2} - \overline{g}^2\right)$ %\hfill Sum of elements in the vect
	\EndFunction
	\end{algorithmic}
	\label{alg:var_func}
	\end{algorithm}
\end{minipage}
\begin{minipage}{0.3\textwidth}
\vspace{-1em}
\algrenewcommand\alglinenumber[1]{\scriptsize #1:}
	\begin{algorithm}[H]
	\small
	\setstretch{1.25}
	\caption{\small Distance to opt.}
	\begin{algorithmic}
	\State \textbf{state: } $\overline{\|g\|}\gets0$, $\overline{h}\gets0$
		\Function{Distance}{gradient $g_t$, $\beta$}
		\State $\overline{\|g\|}\gets \beta \cdot \overline{\|g\|} + (1 - \beta) \cdot \|g_t\|$
		\State $\overline{h} \gets \beta \cdot \overline{h} + (1 - \beta) \cdot \| g_t \|^2$
		\State $D \gets \beta \cdot D + (1 - \beta) \cdot \overline{\|g\|} /\overline{h}$
		\Return $D$
	\EndFunction
	\end{algorithmic}
	\label{alg:dist_func}
	\end{algorithm}
\end{minipage}
\end{table*}
%%%%%%%%%%%%%%%%%%%%%%%%%%%% end of new algorithm

\begin{wrapfigure}[9]{R}{0.55\linewidth}
\vspace{-2.5em}
\hspace{-1em}
\begin{minipage}{\linewidth}
	\begin{equation}
	\begin{aligned}
	&\textsc{(SingleStep)} \\
	 \mu_t, \alpha_t = & \arg \min_{\mu} \mu D^2
		+ \alpha^2 C \\
	s.t.\  \mu \geq & \left(\frac{\sqrt{h_{\max}/h_{\min} }-1}{\sqrt{h_{\max}/h_{\min}}+1}\right)^2 \\
	\alpha =& \frac{(1-\sqrt{\mu})^2}{h_{\min}}
	\end{aligned}
	\label{equ:noisy_min}
	\end{equation}
\end{minipage}
\end{wrapfigure}
Let $D$ be an estimate of the current model's distance to a local quadratic approximation's minimum, and $C$ denote an estimate for gradient variance.
\textsc{SingleStep} minimizes the \emph{multidimensional surrogate} after a single step (i.e. $t=1$) while ensuring $\mu$ and $\alpha$ in the robust region for all directions. \emph{A single instance of \textsc{SingleStep} solves a single momentum and learning rate for the entire model at each iteration.}
Specifically, the extremal curvatures $h_{min}$ and $h_{max}$ denote estimates for the largest and smallest generalized curvature respectively. They are meant to capture both generalized curvature variation along all different directions (like the classic condition number)
and also variation that occurs as the {\em landscape evolves}. The constraints keep the global learning rate and momentum in the robust region (defined in Lemma~\ref{lem:robustness}) 
for slices along all directions.
%along eigenvectors of the quadratic approximation's Hessian. 
%\textsc{SingleStep} can be solved in closed form; we refer to Appendix~\ref{sec:opt} for relevant details on the closed form solution. 

The problem in~\eqref{equ:noisy_min} does not need iterative solver but has an analytical solution. Substituting only the second constraint, the objective becomes $p(x)=x^2D^2 + (1-x)^4/h_{\min}^2C$ with $x=\sqrt{\mu} \in [0, 1)$. By setting the gradient of $p(x)$ to 0, we can get a cubic equation whose root $x=\sqrt{\mu_p}$ can be computed in closed form using Vieta's substitution. As $p(x)$ is uni-modal in $[0, 1)$, the optimizer for \eqref{equ:noisy_min} is exactly the maximum of $\mu_p$ and $(\sqrt{h_{\max}/h_{\min} }-1 )^2 / (\sqrt{h_{\max}/h_{\min}}+1)^2$, the right hand-side of the first constraint in~\eqref{equ:noisy_min}.

\tuner uses functions \textproc{CurvatureRange}, \textproc{Variance} and \textproc{Distance} to measure quantities $h_{\max}$, $h_{\min}$, $C$ and $D$ respectively. These measurement functions can be designed in different ways.
We present the implementations we used for our experiments,
based completely on gradients,  in Section~\ref{sec:oracles}.



%Let $D$ denote an estimate of the current model's distance to a local quadratic approximation's minimum, and $C$ denote an estimate for gradient variance. The extremal curvatures $h_{min}$ and $h_{max}$ denote estimates for the largest and smallest generalized curvature respectively. They are meant to capture both local curvature variation along all different directions (like the classic condition number)
%and also variation that occurs as the {\em landscape evolves}. Thus the constraints keep the global learning rate and momentum in the robust region (defined in Lemma~\ref{lem:robustness}) along all eigendirections of the quadratic approximation. According to Lemma~\ref{lem:robustness} and~\ref{lem:spectral_var_control}, these constraints support the decomposition of \textsc{SingleStep} objective, as the sum of ~\eqref{eqn:noisy_square_dist} (with $t=1$) along the eigendirections of the quadratic approximation.
%\textsc{SingleStep} can be solved in closed form; we refer to Appendix~\ref{sec:opt} for discussion on the closed form solution. 
%\tuner uses functions \textproc{CurvatureRange}, \textproc{Variance} and \textproc{Distance} to measure quantities $h_{\max}$, $h_{\min}$, $C$ and $D$ respectively. These measurement functions can be designed in different ways.
%We present the implementations we used for our experiments,
%based completely on gradients,  in Section~\ref{sec:oracles}.

%\textsc{SingleStep} minimizes the surrogate for the expected squared distance from the optimum of a local quadratic approximation  \eqref{eqn:noisy_square_dist} after a single step ($t=1$),
%while keeping all directions in the robust region \eqref{eqn:robust_region}.
%This is the SGD version of the noiseless tuning rule in \eqref{eqn:noiseless_tuning_rule}.
%It can be solved in closed form; we refer to Appendix~\ref{sec:opt} for discussion on the closed form solution. 
%\tuner uses functions \textproc{CurvatureRange}, \textproc{Variance} and \textproc{Distance} to measure quantities $h_{\max}$, $h_{\min}$, $C$ and $D$ respectively. These measurement functions can be designed in different ways.
%We present the implementations we used for our experiments,
%based completely on gradients,  in Section~\ref{sec:oracles}.




\subsection{Measurement functions in \tuner}
\label{sec:oracles}
This section describes our implementation of the measurement oracles used by \tuner: \textproc{CurvatureRange}, \textproc{Variance}, and \textproc{Distance}.
We design the measurement functions with the assumption of a negative log-probability objective; this is in line with typical losses in machine learning, e.g. cross-entropy for neural nets and maximum likelihood estimation in general.
Under this assumption, the Fisher information matrix---i.e.\ the expected outer product of noisy gradients---approximates the Hessian of the objective~\citep{johnfisherinfo2016,pascanu2013revisiting}. This allows for measurements purely being approximated from minibatch gradients with overhead linear to model dimensionality.
These implementations are not guaranteed to give accurate measurements.
Nonetheless, their use in our experiments in Section~\ref{sec:experiments} shows that they are sufficient for \tuner to outperform the state of the art on a variety of objectives. We also refer to Appendix~\ref{sec:practical_impl} for details on zero-debias~\citep{kingma2014adam}, slow start~\citep{schaul2013no} and smoothing for curvature range estimation.

%\begin{minipage}{0.37\textwidth}
%\algrenewcommand\alglinenumber[1]{\scriptsize #1:}
%	\begin{algorithm}[H]
%	\scriptsize
%	\caption{\small Curvature range}
%	\begin{algorithmic}
%		\State \textbf{state: } $h_{\max}$, $h_{\min}$, $h_i, \forall i \in\{1,2,3,...\}$
%		\Function{CurvatureRange}{gradient $g_t$, $\beta$}
%			\State $h_t \gets \| g_t \|^2$
%			\State $h_{\max,t}\!\!\gets\!\!\!\!\!\!\!\max\limits_{t - w \leq i \leq t}\!h_i$, $h_{\min,t}\!\!\gets\!\!\!\!\!\!\!\min\limits_{t - w \leq i \leq t}\!h_i$
%			\State $h_{\max} \gets \beta \cdot h_{\max} + (1 - \beta) \cdot h_{\max,t}$ %\hfill Smoothed largest curvature.
%			\State $h_{\min} \gets \beta \cdot h_{\min} + (1 - \beta) \cdot h_{\min,t}$ %\hfill Smoothed smallest curvature.
%			\Return $h_{\max}$, $h_{\min}$
%		\EndFunction
%	\end{algorithmic}
%	\label{alg:curv_func}
%	\end{algorithm}
%\end{minipage}
%\begin{minipage}{0.315\textwidth}
%\algrenewcommand\alglinenumber[1]{\scriptsize #1:}
%	\begin{algorithm}[H]
%	\scriptsize
%	\setstretch{1.5}
%	\caption{\small Gradient variance}
%	\begin{algorithmic}
%	\State \textbf{state: } $\overline{g^2}\gets0$, $\overline{g}\gets0$
%	\Function{Variance}{gradient $g_t$, $\beta$}
%		\State $\overline{g^2}\gets\beta \cdot \overline{g^2} + (1 - \beta) \cdot g_t \odot g_t$
%		\State $\overline{g}\gets\beta \cdot \overline{g} + (1 - \beta) \cdot g_t$
%		\Return $\| \overline{g^2} - \overline{g}^2 \|_1$ %\hfill Sum of elements in the vect
%	\EndFunction
%	\end{algorithmic}
%	\label{alg:var_func}
%	\end{algorithm}
%\end{minipage}
%\begin{minipage}{0.3\textwidth}
%\algrenewcommand\alglinenumber[1]{\scriptsize #1:}
%	\begin{algorithm}[H]
%	\scriptsize
%	\setstretch{1.25}
%	\caption{\small Distance to opt.}
%	\begin{algorithmic}
%	\State \textbf{state: } $\overline{\|g\|}\gets0$, $\overline{h}\gets0$
%		\Function{Distance}{gradient $g_t$, $\beta$}
%		\State $\overline{\|g\|}\gets \beta \cdot \overline{\|g\|} + (1 - \beta) \cdot \|g_t\|$
%		\State $\overline{h} \gets \beta \cdot \overline{h} + (1 - \beta) \cdot \| g_t \|^2$
%		\State $D \gets \beta \cdot D + (1 - \beta) \cdot \overline{\|g\|} /\overline{h}$
%		\Return $D$
%	\EndFunction
%	\end{algorithmic}
%	\label{alg:dist_func}
%	\end{algorithm}
%\end{minipage}

\paragraph{Curvature range}
Let $g_t$ be a noisy gradient, we estimate the curvatures range in Algorithm~\ref{alg:curv_func}. We notice that the outer product $g_tg_t^T$ has an eigenvalue $h_t=\| g_t \|^2$ with eigenvector $g_t$. Thus under our negative log-likelihood assumption, we use $h_t$ to approximate the curvature of Hessian along gradient direction $g_t$. Note here we use empirical Fisher $g_tg_t^T$ instead of Fisher information matrix. Empirical Fisher is typically used in practical natural gradient methods~\citep{martens2014new, roux2008topmoumoute, duchi2011adaptive}. For practically efficient measurement, we use the empirical Fisher as a coarse proxy of Fisher information matrix which approximates the Hessian of the objective. 
Specifically in Algorithm~\ref{alg:curv_func}, we maintain $h_{\min}$ and $h_{\max}$ as running averages of extreme curvature $h_{\min, t}$ and $h_{\max, t}$, from a sliding window of width 20\footnote{We use window width 20 across all the models and experiments in our paper. We refer to Section~\ref{sec:experiments} for details on selecting the window width}.
As gradient directions evolve, we estimate curvatures along different directions. Thus $h_{\min}$ and $h_{\max}$ capture the curvature variations.

%\vspace{-0.5em}
\paragraph{Gradient variance}
To estimate the gradient variance in Algorithm~\ref{alg:var_func}, 
we use running averages $\overline{g}$ and $\overline{g^2}$ to keep track of $g_t$ and $g_t \odot g_t$, the first and second order moment of the gradient. 
As $\Var(g_t) = \E{g_t^2} - \E{g_t} \odot \E{g_t}$, we estimate the gradient variance $C$ in \eqref{equ:noisy_min} using $C=\bm{1}^T\!\!\cdot(\overline{g^2} - \overline{g}^2)$. %To get stable estimates, we use $C$, the running average of $C_t$ as the quantity representing gradient variance.

%\vspace{-0.25em}
\paragraph{Distance to optimum}
In Algorithm~\ref{alg:dist_func}, we estimate the distance to the optimum of the local quadratic approximation.
Inspired by the fact that $\| \nabla f(\mat{x}) \| \leq \| \mat{H} \| \| \mat{x} - \mat{x}^{\star}\|$ for a quadratic $f(x)$ with Hessian $\mat{H}$ and minimizer $\mat{x}^{*}$,  
we first maintain $\overline{h}$ and $\overline{\|g\|}$ as running averages of curvature $h_t$ and gradient norm $\| g_t \|$. Then the distance is approximated using $\overline{\|g\|} / \overline{h}$. %according to inequality $\| \nabla f(\mat{x}) \| \leq \| \mat{H} \| \| \mat{x} - \mat{x}^{\star}\|$.

%\begin{minipage}{0.37\textwidth}
%\algrenewcommand\alglinenumber[1]{\scriptsize #1:}
%	\begin{algorithm}[H]
%	\scriptsize
%	\caption{\small Curvature range}
%	\begin{algorithmic}
%		\State \textbf{state: } $h_{\max}$, $h_{\min}$, $h_i, \forall i \in\{1,2,3,...\}$
%		\Function{CurvatureRange}{gradient $g_t$, $\beta$}
%			\State $h_t \gets \| g_t \|^2$
%			\State $h_{\max,t}\!\!\gets\!\!\!\!\!\!\!\max\limits_{t - w \leq i \leq t}\!h_i$, $h_{\min,t}\!\!\gets\!\!\!\!\!\!\!\min\limits_{t - w \leq i \leq t}\!h_i$
%			\State $h_{\max} \gets \beta \cdot h_{\max} + (1 - \beta) \cdot h_{\max,t}$ %\hfill Smoothed largest curvature.
%			\State $h_{\min} \gets \beta \cdot h_{\min} + (1 - \beta) \cdot h_{\min,t}$ %\hfill Smoothed smallest curvature.
%			\Return $h_{\max}$, $h_{\min}$
%		\EndFunction
%	\end{algorithmic}
%	\label{alg:curv_func}
%	\end{algorithm}
%\end{minipage}
%\begin{minipage}{0.315\textwidth}
%\algrenewcommand\alglinenumber[1]{\scriptsize #1:}
%	\begin{algorithm}[H]
%	\scriptsize
%	\setstretch{1.5}
%	\caption{\small Gradient variance}
%	\begin{algorithmic}
%	\State \textbf{state: } $\overline{g^2}\gets0$, $\overline{g}\gets0$
%	\Function{Variance}{gradient $g_t$, $\beta$}
%		\State $\overline{g^2}\gets\beta \cdot \overline{g^2} + (1 - \beta) \cdot g_t \odot g_t$
%		\State $\overline{g}\gets\beta \cdot \overline{g} + (1 - \beta) \cdot g_t$
%		\Return $\| \overline{g^2} - \overline{g}^2 \|_1$ %\hfill Sum of elements in the vect
%	\EndFunction
%	\end{algorithmic}
%	\label{alg:var_func}
%	\end{algorithm}
%\end{minipage}
%\begin{minipage}{0.3\textwidth}
%\algrenewcommand\alglinenumber[1]{\scriptsize #1:}
%	\begin{algorithm}[H]
%	\scriptsize
%	\setstretch{1.25}
%	\caption{\small Distance to opt.}
%	\begin{algorithmic}
%	\State \textbf{state: } $\overline{\|g\|}\gets0$, $\overline{h}\gets0$
%		\Function{Distance}{gradient $g_t$, $\beta$}
%		\State $\overline{\|g\|}\gets \beta \cdot \overline{\|g\|} + (1 - \beta) \cdot \|g_t\|$
%		\State $\overline{h} \gets \beta \cdot \overline{h} + (1 - \beta) \cdot \| g_t \|^2$
%		\State $D \gets \beta \cdot D + (1 - \beta) \cdot \overline{\|g\|} /\overline{h}$
%		\Return $D$
%	\EndFunction
%	\end{algorithmic}
%	\label{alg:dist_func}
%	\end{algorithm}
%\end{minipage}
























