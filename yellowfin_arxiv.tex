%%%%%%%% ICML 2018 EXAMPLE LATEX SUBMISSION FILE %%%%%%%%%%%%%%%%%

\documentclass{article}

% Recommended, but optional, packages for figures and better typesetting:
\usepackage[verbose=true,letterpaper]{geometry}
\usepackage{microtype}
\usepackage{graphicx}
\usepackage{subfigure}
\usepackage{booktabs} % for professional tables

% hyperref makes hyperlinks in the resulting PDF.
% If your build breaks (sometimes temporarily if a hyperlink spans a page)
% please comment out the following usepackage line and replace
% \usepackage{icml2018} with \usepackage[nohyperref]{icml2018} above.
%\usepackage{hyperref}
\usepackage[draft]{hyperref}
\usepackage[numbers]{natbib}

% Attempt to make hyperref and algorithmic work together better:
\newcommand{\theHalgorithm}{\arabic{algorithm}}

% Use the following line for the initial blind version submitted for review:
%\usepackage{icml2018}


\usepackage{amsmath,amsbsy,amssymb,amsfonts,amsthm}
\usepackage{bm}
\usepackage{graphicx}
\usepackage{xspace}
\usepackage{color}
%\usepackage[algcompatible]{algpseudocode}
\usepackage{algorithm, algpseudocode}
%\usepackage{algorithmic}}
%\usepackage{algpseudocode}
\usepackage{wrapfig}
\def\compactify{\itemsep=0pt \topsep=0pt \partopsep=0pt \parsep=0pt}

\usepackage{setspace}
\usepackage{enumitem}
\usepackage{capt-of}


\newtheorem{theorem}{Theorem}
\newtheorem{definition}[theorem]{Definition}
\newtheorem{assumption}[theorem]{Assumption}
\newtheorem{remark}[theorem]{Remark}
\newtheorem{claim}[theorem]{Claim}
\newtheorem{proposition}[theorem]{Proposition}
\newtheorem{lemma}[theorem]{Lemma}
\newtheorem{corollary}[theorem]{Corollary}

\newcommand{\E}{\mathbb{E}}
\newcommand{\Var}{\mathrm{Var}}
\newcommand{\mat}[1]{\bm{\mathit{#1}}}
\algdef{SE}[VARIABLES]{States}{EndStates}
   {\algorithmicvariables}
   {\algorithmicend\ \algorithmicvariables}
\algnewcommand{\algorithmicvariables}{\textbf{States}}
\algrenewcommand\Return{\State \algorithmicreturn{} }
\newcommand*{\AddNote}[4]{
    \begin{tikzpicture}[overlay, remember picture]
        \draw [decoration={brace,amplitude=0.5em},decorate,ultra thick,red]
            ($(#3)!(#1.north)!($(#3)-(0,1)$)$) --  
            ($(#3)!(#2.south)!($(#3)-(0,1)$)$)
                node [align=center, text width=2.5cm, pos=0.5, anchor=west] {#4};
    \end{tikzpicture}
}


\title{\tuner and the Art of Momentum Tuning}


\author{
  Jian Zhang, Ioannis Mitliagkas, Christopher R\'e \\
  Department of Computer Science\\
  Stanford University\\
  \texttt{\{zjian,imit,chrismre\}@cs.stanford.edu} \\
  %% examples of more authors
  %% \And
  %% Coauthor \\
  %% Affiliation \\
  %% Address \\
  %% \texttt{email} \\
  %% \AND
  %% Coauthor \\
  %% Affiliation \\
  %% Address \\
  %% \texttt{email} \\
  %% \And
  %% Coauthor \\
  %% Affiliation \\
  %% Address \\
  %% \texttt{email} \\
  %% \And
  %% Coauthor \\
  %% Affiliation \\
  %% Address \\
  %% \texttt{email} \\
}


\newcommand{\fix}{\marginpar{FIX}}
\newcommand{\new}{\marginpar{NEW}}
\let\textproc\undefined
\newcommand\textproc{\textsc}
\newcommand{\tuner}{\textsc{YellowFin}\xspace}
\newcommand{\asynctuner}{closed-loop \textsc{YellowFin}\xspace}
\newcommand{\Asynctuner}{Closed-loop \textsc{YellowFin}\xspace}

\newcommand{\yell}[1]{#1}
\newcommand{\outline}[1]{}
\newcommand{\forjian}[1]{{\color{magenta}FOR JIAN: #1}}
\newcommand{\notes}[1]{{\color{green}NOTES: #1}}
\newcommand{\jianedits}[1]{#1}

\setlength{\parskip}{1.2ex}
\setlength{\parindent}{0pt}


% If accepted, instead use the following line for the camera-ready submission:
%\usepackage[accepted]{icml2018}

% The \icmltitle you define below is probably too long as a header.
% Therefore, a short form for the running title is supplied here:
%\icmltitlerunning{\tuner and the Art of Momentum Tuning}

\begin{document}
\maketitle
%\twocolumn[
%\icmltitle{\tuner and the Art of Momentum Tuning}

% It is OKAY to include author information, even for blind
% submissions: the style file will automatically remove it for you
% unless you've provided the [accepted] option to the icml2018
% package.

% List of affiliations: The first argument should be a (short)
% identifier you will use later to specify author affiliations
% Academic affiliations should list Department, University, City, Region, Country
% Industry affiliations should list Company, City, Region, Country

% You can specify symbols, otherwise they are numbered in order.
% Ideally, you should not use this facility. Affiliations will be numbered
% in order of appearance and this is the preferred way.
%\icmlsetsymbol{equal}{*}
%
%\begin{icmlauthorlist}
%\icmlauthor{Aeiau Zzzz}{equal,to}
%\icmlauthor{Bauiu C.~Yyyy}{equal,to,goo}
%\icmlauthor{Cieua Vvvvv}{goo}
%\icmlauthor{Iaesut Saoeu}{ed}
%\icmlauthor{Fiuea Rrrr}{to}
%\icmlauthor{Tateu H.~Yasehe}{ed,to,goo}
%\icmlauthor{Aaoeu Iasoh}{goo}
%\icmlauthor{Buiui Eueu}{ed}
%\icmlauthor{Aeuia Zzzz}{ed}
%\icmlauthor{Bieea C.~Yyyy}{to,goo}
%\icmlauthor{Teoau Xxxx}{ed}
%\icmlauthor{Eee Pppp}{ed}
%\end{icmlauthorlist}
%
%\icmlaffiliation{to}{Department of Computation, University of Torontoland, Torontoland, Canada}
%\icmlaffiliation{goo}{Googol ShallowMind, New London, Michigan, USA}
%\icmlaffiliation{ed}{School of Computation, University of Edenborrow, Edenborrow, United Kingdom}
%
%\icmlcorrespondingauthor{Cieua Vvvvv}{c.vvvvv@googol.com}
%\icmlcorrespondingauthor{Eee Pppp}{ep@eden.co.uk}
%
%% You may provide any keywords that you
%% find helpful for describing your paper; these are used to populate
%% the "keywords" metadata in the PDF but will not be shown in the document
%\icmlkeywords{Machine Learning, ICML}
%
%\vskip 0.3in
%]

% this must go after the closing bracket ] following \twocolumn[ ...

% This command actually creates the footnote in the first column
% listing the affiliations and the copyright notice.
% The command takes one argument, which is text to display at the start of the footnote.
% The \icmlEqualContribution command is standard text for equal contribution.
% Remove it (just {}) if you do not need this facility.

%\printAffiliationsAndNotice{}  % leave blank if no need to mention equal contribution
%\printAffiliationsAndNotice{\icmlEqualContribution} % otherwise use the standard text.

\begin{abstract}
\noindent Hyperparameter tuning is one of the most time-consuming workloads in deep learning. 
State-of-the-art optimizers, such as AdaGrad, RMSProp and  Adam,
reduce this labor by adaptively tuning an individual learning rate for each variable.
Recently researchers have shown renewed interest in simpler methods like momentum SGD as they may yield better test metrics.
Motivated by this trend, we ask: can simple adaptive methods based on SGD perform as well or better? We revisit the momentum SGD algorithm and show that hand-tuning a single learning rate and momentum makes it competitive with Adam.
We then analyze its robustness to learning rate misspecification and objective curvature variation.
Based on these insights, we design \tuner, an automatic tuner for momentum and learning rate in SGD.
%In asynchronous-parallel training, \tuner optionally uses a momentum-sensing component with a negative-feedback loop to compensate for the dynamics of asynchrony on the fly.
\tuner optionally uses a negative-feedback loop to compensate for the momentum dynamics in asynchronous settings on the fly.
We empirically show that \tuner can converge in fewer iterations than Adam on ResNets and LSTMs for image recognition, language modeling and constituency parsing,
%	with a speedup of $0.8$x to $6.83$x in synchronous and up to $5.09$x in asynchronous settings.% on ResNet and LSTM models.
with a speedup of up to $3.28$x in synchronous and up to $2.69$x in asynchronous settings.% on ResNet and LSTM models.
\end{abstract}

\section{Introduction}
Accelerated forms of stochastic gradient descent (SGD), pioneered by
\citet{polyak1964some} and \citet{nesterov1983method}, are the de-facto
training algorithms for deep learning.
Their use requires a sane choice for their {\em hyperparameters}: 
typically a {\em learning rate} and {\em momentum parameter} \citep{sutskever2013importance}.
However, tuning hyperparameters is arguably the most time-consuming part of deep learning, with many papers outlining best tuning practices written
\citep{bengio2012practical,orr2003neural,bengio2012deep,bottou2012stochastic}.
Deep learning researchers have proposed a number of methods to deal with hyperparameter optimization, ranging from grid-search and 
smart black-box methods \citep{bergstra2012random,snoek2012practical}
to adaptive optimizers.
Adaptive optimizers aim to eliminate hyperparameter search by tuning on the fly for a single training run:
algorithms like AdaGrad \citep{duchi2011adaptive}, RMSProp \citep{tieleman2012lecture} and Adam \citep{kingma2014adam} use the magnitude of gradient elements to tune learning rates {\em individually for each variable} and  have been largely successful in relieving practitioners of tuning the learning rate. 
%\yell{
%This increased flexibility sounds great,
%however our experiments and recent analysis in literature \citep{wilson2017marginal} suggest that methods that adapt multiple learning rates, yield marginal benefits compared to momentum SGD.
%\citet{wilson2017marginal} argue that those methods also have worse generalization.
%We make another argument: adaptive methods also suffer from not tuning their momentum parameter.
%}

\begin{wrapfigure}[12]{R}{0.55\textwidth}
\vspace{-2.25em}
\begin{minipage}{1.0\linewidth}
\begin{figure}[H]
%	\includegraphics[width=1.\linewidth]{experiment_results/spotlight_default_adam.pdf}
	\includegraphics[width=0.99\linewidth]{experiment_results/spotlight.pdf}
%	\vspace{-1em}
	\caption{\tuner in comparison to Adam on a ResNet (CIFAR100, cf.\ Section~\ref{sec:experiments}) in synchronous and asynchronous settings.}
	\label{fig:spotlight}
%	\vspace{-0.5em}
\end{figure}
\end{minipage}
\end{wrapfigure}
Recently some researchers
 have started favoring simple momentum SGD over the previously mentioned adaptive methods~\citep{chen2016thorough,gehring2017convolutional}, often reporting better test scores \citep{wilson2017marginal}.
Motivated by this trend, we ask the question:
\emph{can simpler adaptive methods based on momentum SGD perform as well or better?}
%We revisit SGD with Polyak's momentum and a single learning rate for all variables.
We empirically show, with a hand-tuned learning rate, Polyak's momentum SGD achieves faster convergence than Adam for a large class of models.
We then formulate the optimization update as a dynamical system and study certain robustness properties of the momentum operator.
Inspired by our analysis, we design \tuner, an automatic hyperparameter tuner for momentum SGD.
\tuner simultaneously tunes the learning rate and momentum on the fly, and can handle the complex dynamics of asynchronous execution.
Our contribution and outline are as follows:
\begin{itemize}[leftmargin=2em]
\setlength\itemsep{0.2em}
\item
In Section~\ref{sec:momentum_operator}, we demonstrate examples where momentum offers convergence robust to learning rate misspecification and curvature variation in a class of non-convex objectives.
This robustness is desirable for deep learning.
It stems from a known but obscure fact:
the momentum operator's spectral radius is constant in a large subset of the hyperparameter space.
%\vspace{-1em}
\item
In Section~\ref{sec:sync_tuner}, we use these robustness insights and a simple quadratic model analysis to motivate the design of \tuner, an automatic tuner for momentum SGD.
\tuner uses on-the-fly measurements from the gradients to tune both a single learning rate and a single momentum.
%\setlength\itemsep{0.2em}
\item In Section~\ref{sec:stability}, we discuss common stability concerns related to the phenomenon of exploding gradients \citep{pascanu2013difficulty}.
We present a natural extension to our basic tuner, using adaptive gradient clipping, to stabilize training for objectives with exploding gradients.
%We present a natural extension to our basic tuner that stabilizes training for those objectives without sacrificing performance via adaptive gradient clipping.
\item In Section~\ref{sec:async_tuner} we present \asynctuner, suited for asynchronous training.
It uses a novel component for  measuring the total momentum in a running system, including any asynchrony-induced momentum, a phenomenon described in \cite{mitliagkas2016asynchrony}.
This measurement is used in a negative feedback loop to control the value of algorithmic momentum.% on the fly.

\end{itemize}


We provide a thorough empirical evaluation of the performance and stability of our tuner.
In Section~\ref{sec:experiments}, we demonstrate empirically that \yell{on ResNets and LSTMs}
\tuner can converge in fewer iterations compared to:
(i) hand-tuned momentum SGD (up to $1.75$x speedup);
and (ii) hand-tuned Adam ($0.77$x to $3.28$x speedup).
%and (ii) default Adam (0.8x to 6.83x speedup).
Under asynchrony, the closed-loop control architecture speeds up \tuner, 
making it up to $2.69$x faster than Adam. 
%making it up to $5.08$x faster than Adam. 
Our experiments include runs on $7$ different models, randomized over at least $3$ different random seeds. 
\tuner is stable and achieves consistent performance: the normalized sample standard deviation of test metrics varies from $0.05\%$ to $0.6\%$.
We released PyTorch and TensorFlow implementations
\footnote{TensorFlow: goo.gl/zC2rjG.  PyTorch: goo.gl/N4sFfs}%\footnote{PyTorch: https://github.com/AnonRepository/YellowFin\_Pytorch} 
that can be used as drop-in replacements for any optimizer.
%\footnote{ TensorFlow implementation: \url{https://github.com/JianGoForIt/YellowFin}}\footnote{PyTorch implementation: \url{https://github.com/JianGoForIt/YellowFin_Pytorch}.}.
%\tuner has also been implemented by various members of the ML community in Caffe2, Tensor2Tensor. 
\tuner has also been implemented in various other packages.
Its large-scale deployment in industry has taught us important lessons about stability; we discuss those challenges and our solution in Section~\ref{sec:stability}.
We conclude with related work and discussion in Section~\ref{sec:related} and~\ref{sec:discussion}.


\vspace{-0.1em}
\section{The momentum operator}
\label{sec:momentum_operator}

\newcommand{\gc}{generalized curvature\xspace}
\newcommand{\Gc}{Generalized curvature\xspace}
\vspace{-0.15em}
In this section, we identify the main technical insight behind the design of \tuner:
%After preliminaries on momentum gradient descent, 
 gradient descent with momentum can exhibit linear convergence robust to learning rate misspecification and to curvature variation.
The robustness to learning rate misspecification means tolerance to a less-carefully-tuned learning rate.
On the other hand, the robustness to curvature variation means empirical linear convergence on a class of non-convex objectives with varying curvatures.
After preliminary on momentum, 
we discuss these two properties desirable for deep learning objectives.
%In this section, we analyze the main insights for designing \tuner, showing that momentum is robust to learning rate misspecification and curvature variation for a class of non-convex objectives.
%%momentum is robust to learning rate misspecification and curvature variation for a class of non-convex objectives.

%In this section, we analyze two properties of momentum: it is robust to learning rate misspecification and to curvature variation for a class of non-convex objectives, both desirable for deep learning.


%\vspace{-0.5em}
\subsection{Preliminaries}
\label{sec:robust_preliminaries}
We aim to minimize some objective $f(x)$.
In machine learning, $x$ is referred to as {\em the model} and the objective is some {\em loss function}.
A low loss implies a well-fit model.
Gradient descent-based procedures use the gradient of the objective function, $\nabla f(x)$, to update the model iteratively.
These procedures can be characterized by the convergence rate with respect to the distance to a minimum.
\begin{definition} [Convergence rate]
	Let $x^*$ be a local minimum of $f(x)$ and $x_t$ denote the model after $t$ steps of an iterative procedure. %$f(x):\mathbb{R}^d \mapsto \mathbb{R}$ 
%	and $x_t$ be the model after $t$ steps in an iterative minimization procedure. 
The iterates converge to $x^*$ with linear rate $\beta$,
	if \[ \| x_{t} - x^* \| = O(\beta^t \| x_0 - x^* \|).\]
\end{definition}
Polyak's momentum gradient descent \citep{polyak1964some} is one of these iterative procedures, given by
%Polyak's momentum gradient descent update \citep{polyak1964some} is given by
\begin{align}
	x_{t+1}  &= x_t - \alpha \nabla f(x_t) + \mu (x_t - x_{t-1}),
	\label{eqn:momentum_gd}
\end{align} 
where $\alpha$ denotes a single learning rate and $\mu$ a single momentum for all model variables.   
Momentum's main appeal is its established ability to {accelerate convergence} \citep{polyak1964some}. 
On a $\gamma$-strongly convex $\delta$-smooth function with condition number $\kappa=\delta/\gamma$, the optimal convergence rate of gradient descent without momentum
is $O(\frac{\kappa-1}{\kappa+1})$~\citep{nesterov2013introductory}.
On the other hand, for certain classes of strongly convex and smooth functions, like quadratics,
 the optimal momentum value,
\vspace{-0.5em}
\begin{equation}
	\mu^* = \left(\frac{\sqrt{\kappa}-1}{\sqrt{\kappa}+1}\right)^2,
	\label{eqn:optimal_momentum}
\end{equation}
yields the optimal accelerated linear convergence rate $O(\frac{\sqrt{\kappa}-1}{\sqrt{\kappa}+1})$.
{\em This guarantee does not generalize to arbitrary strongly convex smooth functions} \citep{lessard2016analysis}.
Nonetheless, this linear rate can often be observed in practice even on non-quadratics (cf. Section~\ref{sec:robust_properties}).

{\bf Key insight:}
%{\bf The main insight we expose has been hidden in proofs of the optimal convergence rate:}
%{\bf Main insights:}
%On quadratics with different curvature along different axes, when using \emph{gradient descent without momentum}, 
%we get different convergence rate to the minimum along the axes; the achieved overall rate, $O(\frac{\kappa-1}{\kappa+1})$, is just the slowest rate over all axes. 
%In sharp contrast, Polyak's momentum in \eqref{eqn:momentum_gd} converges to a minimum with the accelerated linear rate $\sqrt{\mu^*}$ from {\em all directions},
%when we use the optimal momentum value, $\mu^*$.
%This can be seen by analyzing a quadratic, though it might not generalize to all strongly convex, smooth functions.
%Concretely, consider a strongly convex quadratic, and let $x_{i, t}$ and $x_i^*$ be the i-th coordinates of $x_t$ and $x^*$.
%%%%%%%%%%%%%%%%%%%%%%%%%%%%%%% backup discussion %%%%%%%%%%%%%%%%%%%%%%%%%%%%
%yields the optimal accelerated linear convergence rate $O(\frac{\sqrt{\kappa}-1}{\sqrt{\kappa}+1})$ 
%%(i.e. $O(\sqrt{\mu^*})$)
%%}.
%%\footnote{
%%{
%%\yell{
%This guarantee does not generalize to arbitrary non-quadratics \citep{lessard2016analysis}.
%Nonetheless, the linear convergence can be observed in practice (cf. Section~\ref{sec:robust_properties}).
%%}
%%}
%
%{\bf A fact is often hidden away in proofs of the guarantee:} 
%{\bf For any momentum $\mu > \mu^*$, there exists a range of different learning rate $\alpha$
%\emph{for any momentum $\mu \geq \mu^*$, 
%using a range of different learning rate,
Consider a quadratic objective with condition number $\kappa > 1$.
Even though its curvature is different along the different directions,
% and different curvatures along different axes.
%On this quadratic, 
%when using \emph{gradient descent without momentum}, 
%we get different convergence rate along the axes; the achieved overall rate, $O(\frac{\kappa-1}{\kappa+1})$, is just the slowest rate over all axes.
Polyak's momentum gradient descent, with $\mu \geq \mu^*$, achieves \emph{the same linear convergence rate $\sqrt{\mu}$ along all directions}. Specifically, let $x_{i, t}$ and $x_i^*$ be the i-th coordinates of $x_t$ and $x^*\!$.
%%%%%%%%%%%%%%%%%%%%%%%%%%%%%%%%%%%%%%%%%%%%%%%%%%%%%%%%%%%%%%%%%%%%%%%%%%%%%%
%the optimum and the model after $t$ iterations respectively, 
For any $\mu \geq \mu^*$ with an appropriate learning rate, the update in~\eqref{eqn:momentum_gd} can achieve $| x_{i, t} - x_i^* | \leq \sqrt{\mu}^t | x_{i,0} - x_i^* |$ simultaneously along all axes $i$.
This insight has been hidden away in proofs.
%The same constant convergence rate holds along any arbitrary projection.

%%This insight
%This hidden fact 
%reveals a common linear convergence rate on quadratics along all axes.
In this quadratic case, curvature is different across different axes, but remains {constant on any one-dimensional slice}. 
In the next section (Section~\ref{sec:robust_properties}), we extend this insight to non-quadratic one-dimensional functions.
We then present the \emph{main technical insight behind the design of \tuner: 
%\emph{
similar linear convergence rate $\sqrt{\mu}$ can be achieved in a class of one-dimensional non-convex objectives where curvature varies}; this linear convergence behavior is robust to learning rate misspecification and to the varying curvature. These \emph{robustness properties} are behind a tuning rule for learning rate and momentum in Section~\ref{sec:robust_properties}. We extend this rule to handle SGD noise and generalize it to multidimensional objectives in Section~\ref{sec:sync_tuner}.

%In the next subsection, we unpack, extend and explains this insight as two properties on the convergence behavior of Polyak's momentum gradient descent:
%(i) robustness to learning rate misspecification: 
%%this is the dual view to the curvature robustness described above and means 
%convergence can be tolerant to less-carefully-tuned learning rate;
%(ii) robustness to the varying curvature {\em along a one-dimensional objective (or a scalar-slice of multidimensional objective)}:
%we already discussed curvature variation along different directions on quadratics, as well as the role of the condition number;
%next we will see how curvature and condition number admit a meaningful generalization that allows us to characterize convergence on one-dimensional objectives with varying curvatures.
%%We will see that we can observe constant linear rates and acceleration even on one-dimensional objectives by using our rule to tune a positive value for the momentum parameter.
%%Then we extend this tuning rule to multi-dimensional objectives.

%(e.g.\ the `$\mu=0.0$` curve in Figure~\ref{fig:lr_robustness})
%on quadratics,
%$\mu^*$ is the smallest value for which the update in~\eqref{eqn:momentum_gd} can achieve \emph{the same linear convergence rate 1) along all axes with varying curvatures, 2) over a range of different learning rate}.  For quadratics, curvature varies across axes but remains constant on the slice along each axis. In Section~\ref{sec:robust_properties}, we show with examples that similar robust linear convergence can be achieved in a class of one dimensional non-convex objectives where curvature varies.
%%%%%%%%%%%%%%%%%%%%%%%%%%% backup discussion %%%%%%%%%%%%%%%%%%%%%%%%%%%%%%%%
%Then {for any $\mu\!\geq\!\mu^*\!$}, there exists {a range of $\alpha$}, such that $| x_{i, t} - x_i^* | \leq\sqrt{\mu}^t | x_{i,0} - x_i^* |$ for all axis $i$. For quadratics, curvature varies across axes but remains constant on a slice along each axis. In Section~\ref{sec:robust_properties}, we show with examples that \emph{similar robust linear convergence can be achieved in a class of one dimensional non-convex objectives where curvature varies}.
%%%%%%%%%%%%%%%%%%%%%%%%%%%%%%%%%%%%%%%%%%%%%%%%%%%%%%%%%%%%%%%%%%%%%%%%%%%%%%%
%{\bf This fact is often hidden away in proofs}. 
%We shed light on some of its previously unknown implications in Section~\ref{sec:robust_properties}.

% {\bf This property is often hidden away in proofs}. We extend the intuition of this property from quadratics to one dimensional non-quadratic in Section~\ref{sec:robust_properties}. 

%We shed light on some of its previously unknown implications in Section~\ref{sec:robust_properties}. % and use them in our tuner in Section~\ref{sec:sync_tuner}.

\subsection{Robustness properties of the momentum operator}
%\vspace{-0.2em}
\label{sec:robust_properties}
In this section, we analyze the dynamics of momentum on a class of one-dimensional, non-convex objectives.
We first introduce the notion of {\em generalized curvature} and use it to describe the momentum operator.
Then we discuss the robustness properties of the momentum operator.
%, and extract a tuning rule for learning rate $\alpha$ and momentum $\mu$. 
%\forjian{Should we explicitly mention what is the robustness?}
%In the last section, we discuss the same linear convergence rate of Polyak's momentum gradient descent on different axes with different curvatures for quadratics. 
%For quadratics, the curvature along each axes is constant. 
%In this section, we extend the discussion to a simple class of one dimensional non-convex functions with curvature variations.
%%In this section we analyze the dynamics of momentum on a simple class of one dimensional, non-convex objectives.
%We first introduce the notion of {\em generalized curvature} and use it to describe the momentum operator.
%Then we discuss the robustness properties of the momentum operator: achieving linear convergence rate 

Curvature along different directions is encoded in the different eigenvalues of the Hessian. 
%The classic condition number captures the local variation of curvature, i.e. the eigenvalues of local Hessian.
It is the only feature of a quadratic needed to characterize the convergence of gradient descent. Specifically, gradient descent achieves a linear convergence rate $|1 - \alpha h_c|$ on one-dimensional quadratics with constant curvature $h_c$. 
On one-dimensional \emph{non-quadratic objectives with varying curvature}, this neat characterization is lost.
We can recover it by defining a new kind of ``curvature'' with respect to a specific minimum.

\begin{definition}[\Gc]
\label{def:generalized_curvature}
Let $x^*$ be a local minimum of $f(x):\mathbb{R}\rightarrow\mathbb{R}$.
Generalized curvature with respect to $x^*$, denoted by $h(x)$, satisfies the following.
\begin{equation}
	 f'(x) = h(x) (x - x^*). 
	\label{eqn:generalized_curvature}
\end{equation}
%We call $h(x)$ the {\em generalized curvature}.
\end{definition}
\Gc describes, in some sense, \emph{non-local curvature} with respect to minimum $x^*$.
It coincides with curvature on quadratics.
On non-quadratic objectives, it characterizes the convergence behavior of gradient descent-based algorithms.
Specifically, we recover the fact that starting at point $x_t$, distance from minimum $x^*$ is reduced by $|1-\alpha h(x_t)|$ in one step of gradient descent.
%It can captures the longer-range, non-local variations of curvature.
%For quadratic objectives, it coincides with the standard definition of curvature, and is the sole quantity related to the objective that influences the dynamics of gradient descent.
%For example, the contraction of a gradient descent step is $1-\alpha h(x_t)$.
%Using a state-space augmentation, we can rewrite the momentum update of~\eqref{eqn:momentum_gd}, and define the {\em momentum operator} $\mat{A}_t$ at time $t$ as
%\begin{equation}
%\begin{aligned}
%{\begin{pmatrix}
%x_{t+1} - x^*\\
%x_t - x^* \\
%\end{pmatrix}}
%&=
%{\begin{bmatrix}
%1-\alpha h(x_t) + \mu & - \mu \\
%1 & 0 \\
%\end{bmatrix}}
%{\begin{pmatrix}
%x_t - x^* \\
%x_{t-1} - x^*\\
%\end{pmatrix}} \\
%&\triangleq
%\mat{A}_t
%{\begin{pmatrix}
%x_t - x^* \\
%x_{t-1} - x^*\\
%\end{pmatrix}}.
%\label{equ:one_dim_22_rec}
%\end{aligned}
%\end{equation}
Using a state-space augmentation, we can rewrite the momentum update of~\eqref{eqn:momentum_gd} as
\begin{equation}
\begin{aligned}
{\begin{pmatrix}
x_{t+1} - x^*\\
x_t - x^* \\
\end{pmatrix}}
&= \mat{A}_t
{\begin{pmatrix}
x_t - x^* \\
x_{t-1} - x^*\\
\end{pmatrix}} %\\
%&\triangleq
%\mat{A}_t
%{\begin{pmatrix}
%x_t - x^* \\
%x_{t-1} - x^*\\
%\end{pmatrix}}.
\label{equ:one_dim_22_rec}
\end{aligned}
\end{equation}
where the {\em momentum operator} $\mat{A}_t$ at time $t$ is defined as
\begin{equation}
	\mat{A}_t \triangleq {\begin{bmatrix}
	1-\alpha h(x_t) + \mu & - \mu \\
	1 & 0 \\
	\end{bmatrix}}
\end{equation}



\begin{lemma}[Robustness of the momentum operator]
\label{lem:robustness}
Assume that generalized curvature $h$ and hyperparameters $\alpha,\mu$ satisfy
\begin{align}
{(1-\sqrt{\mu})^2} &\leq \alpha h(x_t) \leq {(1+\sqrt{\mu})^2}.
\label{eqn:robust_region}
\end{align}
%then the spectral radius of the momentum operator, which describes convergence behavior, only depends on  momentum: $	\rho(\mat{A}_t) = \sqrt{\mu}
Then as proven in Appendix~\ref{sec:proof_robustness}, the spectral radius of the momentum operator at step $t$ depends solely on the  momentum parameter: $	\rho(\mat{A}_t) = \sqrt{\mu}$, for all $t$. 
The inequalities in \eqref{eqn:robust_region} define the {\bf robust region}, the set of learning rate $\alpha$ and momentum $\mu$ achieving this $\sqrt{\mu}$ spectral radius.
\end{lemma}
%The spectral radius of an operator, $A_t$ gives the convergence rate when the same operator is applied multiple times, i.e.\ $\mat{A}_t\cdots\mat{A}_t$.
%In this case, different operators are applied, $\mat{A}_t\cdots\mat{A}_1$,
%all of which have the same spectral radius. 
%This means that, as already discussed, {\em we do not give convergence rate guarantees}.
%However, we do show examples where the time-homogeneous spectral radius guaranteed by Lemma~\ref{lem:robustness} translates to constant linear rates on non-convex objectives and predicts the convergence rate of most model variables in deep learning objectives (Figure~\ref{fig:curvature_robustness}).
%Next we discuss two useful implications of this result.
%%Thus we explain Lemma~\ref{lem:robustness} as the {\bf robustness properties} of momentum operator: \emph{time-homogeneous spectral radii $\sqrt{\mu}$ implies asymptotic linear convergence robust with respect to learning rate and to curvature variations}.
%%Note the spectral radius of the composition of operators $\mat{A}_t\cdots\mat{A}_1$, all with spectral radius $\sqrt{\mu}$, does not always follow the asymptotics of $\sqrt{\mu}^t$.
%%In other words, {\em we do not provide a convergence rate guarantee}. Instead, we show the robustness properties with examples.

%%%%%%%%%%% backup discussion %%%%%%%%%%%%%%%%%
%The proof is given in Appendix~\ref{sec:proof_robustness}.
We know that the spectral radius of an operator, $\mat{A}$, describes its asymptotic behavior when applied multiple times: $\| A^t x \| \approx O(\rho(\mat{A})^t)$.\footnote{
For any $\epsilon > 0$, there exists a matrix norm $\|\cdot\|$ such that $\|\mat{A}\| \leq \rho(A) + \epsilon$~\citep{simon2012spectralradius}.
}
%\footnote{For any $\epsilon > 0$, there exists a matrix norm $\|\cdot\|$ such that $\|\mat{A}_t\cdots\mat{A}_1\| \leq \rho(\mat{A}_t\cdots\mat{A}_1) + \epsilon$~\citep{simon2012spectralradius}.}.
Unfortunately, the same does not always hold for the composition of {\em different } operators, even if they have the same spectral radius, $\rho(\mat{A}_t)=\sqrt{\mu}$.
It is not always true that $\| \mat{A}_t\cdots\mat{A}_1 x\| = O(\sqrt{\mu}^t)$.
%According to~\eqref{equ:one_dim_22_rec}, the operator composition $\mat{A}_t\cdots\mat{A}_1$ describes the evolution of distance $|x_t -x^*|$.
%Consequently, the spectral radius of this composition $\mat{A}_t\cdots\mat{A}_1$ is closely related to the convergence behavior, namely the asymptotic relation between $|x_t -x^*|$ and the initial distance $|x_0 - x^*|$. 
%%The spectral radius of $\mat{A}_t$ can describe the evolution of distance $|x_t -x^*|$ in~\eqref{equ:one_dim_22_rec}. Consequently, the spectral radius of the composition $\mat{A}_t\cdots\mat{A}_1$ is closely related\footnote{For any $\epsilon > 0$, there exists a matrix norm $\|\cdot\|$ such that $\|\mat{A}_t\cdots\mat{A}_1\| \leq \rho(\mat{A}_t\cdots\mat{A}_1) + \epsilon$~\citep{simon2012spectralradius}.} to the convergence behavior, namely the asymptotic relationship between $|x_t -x^*|$ and the initial distance $|x_0 - x^*|$. 
%Unfortunately given time-homogenous spectral radii $\sqrt{\mu}$, the spectral radius of the composition $\mat{A}_t\cdots\mat{A}_1$ does not always follow the asymptotics of $\sqrt{\mu}^t$.
However, a homogeneous spectral radius often yields the $\sqrt{\mu}^t$ rate empirically.
In other words, {\em this linear convergence rate is not guaranteed}.
Instead, we demonstrate examples to expose the {\bf robustness properties}: \emph{if the learning rate $\alpha$ and momentum $\mu$ are in the robust region, 
the homogeneity of spectral radii can empirically yield linear convergence with rate $\sqrt{\mu}$; this behavior is robust with respect to learning rate misspecification and to varying curvature}.
%Thus Lemma~\ref{lem:robustness} can imply the {\bf robustness properties} of momentum operator: \emph{time-homogeneous spectral radius $\sqrt{\mu}$ of $\mat{A}_t$ imply empirical linear convergence; this behavior is robust with respect to learning rate and to curvature variations}.
%%%%%%%%%%%%%%%%%%%%%%%%%%%%%%%%%%%%%%%%%%%%%%


%\begin{wrapfigure}[12]{r}{0.25\textwidth}
%\vspace{-2.5em}
%\begin{minipage}{1.0\linewidth}
%\begin{figure}[H]
%  \includegraphics[width=\linewidth]{figures/spectral_radii}
%%  \vspace{-0.75em}
%\caption{
%Momentum operator on scalar quadratic.
%}
%\label{fig:lr_robustness}
%\end{figure}
%\end{minipage}
%\end{wrapfigure}
%\vspace{-0.5em}
\paragraph{Momentum is robust to learning rate misspecification}
\label{sec:lr_robustness}
For a one-dimensional quadratic with curvature $h$,
we have generalized curvature $h(x)=h$ for all $x$. Lemma~\ref{lem:robustness} implies the spectral radius $\rho(\mat{A}_t)\!=\!\sqrt{\mu}$ if
\begin{align}
{(1-\sqrt{\mu})^2/h} &\leq \alpha \leq {(1+\sqrt{\mu})^2/h}.
\label{eqn:lr_robustness}
\end{align}

\begin{wrapfigure}[14]{R}{0.23\textwidth}
\vspace{-2.5em}
%\hspace{-0.5em}
\begin{minipage}{1.0\linewidth}
\begin{figure}[H]
  \includegraphics[width=\linewidth]{figures/spectral_radii}
  \vspace{-1.5em}
\caption{
Spectral radius of momentum operator on scalar quadratic
for varying $\alpha$.
}
\label{fig:lr_robustness}
\end{figure}
\end{minipage}
\end{wrapfigure}
In Figure~\ref{fig:lr_robustness}, we plot $\rho(\mat{A}_t)$ for different $\alpha$ and $\mu$ when $h\!=\!1$.
The solid line segments correspond to the robust region.
As we increase momentum, a linear rate of convergence, $\sqrt{\mu}$, is robustly achieved by an ever-widening range of learning rates:
higher values of momentum are more robust to learning rate mispecification.
%We also note that for objectives with large condition number, higher values of momentum are {\em both faster and more robust}.

{\bf This property influences the design of our tuner:}
\emph{more generally for a class of one-dimensional non-convex objectives},
as long as the learning rate $\alpha$ and momentum $\mu$ are in the {\em robust region}, i.e.\ satisfy \eqref{eqn:robust_region} at every step, then
{\em momentum operators at all steps $t$ have the same spectral radius}.
In the case of quadratics, this implies a convergence rate of $\sqrt{\mu}$, independent of the learning rate.
%%%%%%%%%%%%%%%%%% backup discussion %%%%%%%%%%%%%%%%%%%%%
%We also note that for objectives with large condition number, higher values of momentum can be {\em more robust to learning rate variations}.
%{\bf This property influences the design of our tuner:} as long as \eqref{eqn:lr_robustness} is satisfied, we are in the {\em robust region} and 
%expect the same convergence rate of $\sqrt{\mu}$ for quadratics, independent of the learning rate.
%%%%%%%%%%%%%%%%%%%%%%%%%%%%%%%%%%%%%%%%%%%%%%%%%%%%%%%%%%%
Having established that, we can just focus on optimally tuning momentum.


%\vspace{-0.5em}
\paragraph{Momentum is robust to varying curvature}
\label{sec:curvature_robustness}

As discussed in Section~\ref{sec:robust_preliminaries}, the intuition hidden in classic results
is that for certain strongly convex smooth objectives, 
%momentum larger than 
momentum at least as high as
the value in \eqref{eqn:optimal_momentum} can achieve the same rate of linear convergence along all axes with different curvatures. 
We extend this intuition to certain one-dimensional non-convex functions with varying curvatures along their domains; we discuss the generalization to multidimensional cases in Section~\ref{sec:tuner}.
%where steepness---and, as a result, contractivity---vary as we move along.
Lemma~\ref{lem:robustness} guarantees constant, time-homogeneous spectral radii for momentum operators $\mat{A}_t$ 
assuming \eqref{eqn:robust_region} is satisfied at every step. 
This assumption motivates a ``long-range'' extension of the condition number.
%%%%%%%%%%%%%%%%%%% backup discussion %%%%%%%%%%%%%%%%%%
%is that for a subset of strongly convex smooth objectives, momentum greater than the value in \eqref{eqn:optimal_momentum} guarantees the same rate of linear convergence along all axes with different curvatures. 
%We extend this intuition to certain one dimensional, non-convex functions where curvature varies along this slice.
%Lemma~\ref{lem:robustness} guarantees a constant, time-homogeneous spectral radius for the momentum operators $\mat{A}_t$ if 
%\eqref{eqn:robust_region} is satisfied at every step. 
%This motivates an extension of the condition number.
%%%%%%%%%%%%%%%%%%%%%%%%%%%%%%%%%%%%%%%%%%%%%%%%%%%%%%%%
\begin{definition}[Generalized condition number]
We define the generalized condition number (GCN) with respect to a local minimum $x^*$ of a scalar function, $f(x):\mathbb{R}\rightarrow \mathbb{R}$, to be the dynamic range of its generalized curvature $h(x)$:
\begin{equation}
	\nu = \frac{\sup_{x \in dom(f)} h(x)}{ \inf_{x \in dom(f)} h(x)}
\end{equation}
\end{definition}
The GCN captures variations in generalized curvature along a scalar slice.
From Lemma~\ref{lem:robustness} we get
\begin{equation}
%\begin{aligned}
\begin{gathered}
	\mu \geq \mu^* = \left(\frac{\sqrt{\nu}-1}{\sqrt{\nu}+1}\right)^2, \\
%	\quad
	\frac{(1-\sqrt{\mu})^2}{\inf_{x \in dom(f)}h(x)} \leq \alpha \leq \frac{(1+\sqrt{\mu})^2}{\sup_{x \in dom(f)}h(x)}
	\label{eqn:noiseless_tuning_rule}
\end{gathered}
%\end{aligned}
\end{equation}
as the description of the robust region. The momentum and learning rate satisfying~\eqref{eqn:noiseless_tuning_rule} guarantees a homogeneous spectral radius of $\sqrt{\mu}$ for all $\mat{A}_t$.
Specifically, $\mu^*$ is the smallest momentum value that allows for homogeneous spectral radii.
%The spectral radius of an operator describes its asymptotic convergence behavior. 
%However, the product of a sequence of operators $\mat{A}_t\cdots\mat{A}_1$ all with spectral radius $\sqrt{\mu}$ does not always follow the asymptotics of $\sqrt{\mu}^t$.
%In other words, {\em we do not provide a convergence rate guarantee}.
%Instead, we provide evidence in support of this intuition. 
We demonstrate with examples that \emph{homogeneous spectral radii suggest an empirical linear convergence behavior on a class of non-convex objectives}. In Figure~\ref{fig:curvature_robustness}(a), the non-convex objective,
composed of two quadratics with curvatures $1$ and $1000$, has a GCN of $1000$.
Using the tuning rule of \eqref{eqn:noiseless_tuning_rule}, and running the momentum algorithm (Figure~\ref{fig:curvature_robustness}(b)) practically yields the linear convergence predicted by Lemma~\ref{lem:robustness}.
%In Figures~\ref{fig:curvature_robustness}(c,d) we demonstrate that for an LSTM,
%the majority of model variables follow a $\sqrt{\mu}$ convergence rate.
In Figures~\ref{fig:curvature_robustness}(c,d), we demonstrate an LSTM as another example. As we increase the momentum value (the same value for all variables in the model), more model variables follow a $\sqrt{\mu}$ convergence rate.
In these examples, \emph{the linear convergence is robust to the varying curvature of the objectives}. \textbf{This property influences our tuner design:}
{in the next section, we extend the tuning rules of \eqref{eqn:noiseless_tuning_rule} to handle SGD noise; 
we generalize the extended rule to multidimensional cases as the tuning rule in \tuner}.
%the multidimensional generalization of the extended rule is the tuning rule in \tuner}.
%in the next section we use the tuning rules of \eqref{eqn:noiseless_tuning_rule} in \tuner,
%generalized appropriately to handle SGD noise.






\begin{figure*}[t]
\centering
\vspace{-0.5em}
\begin{tabular}{c c c c}
  \includegraphics[width=0.225\linewidth]{figures/non_convex_toy} &
  \includegraphics[width=0.235\linewidth]{figures/non_convex_constant_rate} &
  \includegraphics[width=0.235\linewidth]{figures/constant_rate_09} &
  \includegraphics[width=0.185\linewidth]{figures/constant_rate_099} \\
  (a) & (b) & (c) &(d)
\end{tabular}
\vspace{-0.5em}
\caption{(a) Non-convex toy example;
(b) linear convergence rate achieved empirically on the example in (a) tuned according to \eqref{eqn:noiseless_tuning_rule};
(c,d)
LSTM on MNIST: as momentum increases from $0.9$ to $0.99$, the global learning rate and momentum falls in robust regions of more model variables. The convergence behavior (shown in grey) of these variables follow the robust rate $\sqrt{\mu}$ (shown in red).}
\vspace{-0.25em}
\label{fig:curvature_robustness}
\end{figure*}











\vspace{-0.2em}
\section{The \tuner tuner}
\label{sec:sync_tuner}
\vspace{-0.2em}
%In this section we describe \tuner, our tuner for momentum SGD.
%We introduce a noisy quadratic model and work on a local quadratic approximation of $f(x)$ to apply the tuning rule of \eqref{eqn:noiseless_tuning_rule} to SGD on an arbitrary objective.
%\tuner is our implementation of that rule.

Here we describe our tuner for momentum SGD that uses the same learning rate for all variables.
We first introduce a noisy quadratic model $f(x)$ as the local approximation of an arbitrary one-dimensional objective. On this approximation, we extend the tuning rule of \eqref{eqn:noiseless_tuning_rule} to SGD. In section~\ref{sec:tuner}, \emph{we generalize the discussion to multidimensional objectives; it yields the \tuner tuning rule}.


\paragraph{Noisy quadratic model}
\label{sec:noisy_quadratics}

\newcommand{\oac}{origin-adjusted curvature }
\newcommand{\bx}{\bar{x}}


We consider a scalar quadratic 
\begin{equation}
	f(x) = \frac{h}{2} x^2 + C 
	= \sum_i \frac{h}{2n}(x-c_i)^2
	\triangleq \frac{1}{n} \sum_i f_i(x)
%	\quad \sum_i c_i = 0.
	\label{eqn:noise_quad_1d}
\end{equation}
with $\sum_i c_i = 0$. $f(x)$ is a quadratic approximation of the original objectives with $h$ and $C$ derived from measurement on the original objective. The function $f(x)$ is defined as the average of $n$ {\em component functions}, $f_i$.
This is a common model for SGD, where we use only a single data point (or a mini-batch) drawn uniformly at random, $S_t \sim \mathrm{Uni}([n])$ to compute a noisy gradient, $\nabla f_{S_t}(x)$, for step $t$.
Here, $C=\frac{1}{2n}\sum_i h c_i^2$ denotes the {\em gradient variance}.
As optimization on quadratics decomposes into scalar problems along the principal eigenvectors of the Hessian, the scalar model in~\eqref{eqn:noise_quad_1d} is sufficient to study local quadratic approximations of multidimensional objectives.
Next we get an {\em exact} expression for the mean square error after running momentum SGD on the scalar quadratic in~\eqref{eqn:noise_quad_1d} for $t$ steps.



\begin{lemma}
\label{lem:main_lemma}
Let $f(x)$ be defined as in \eqref{eqn:noise_quad_1d},
$x_1=x_0$ and $x_t$ follow the momentum update \eqref{eqn:momentum_gd} with stochastic gradients $\nabla f_{S_t}(x_{t-1})$ for $t \geq 2$.
Let $\mat{e}_1=[1, 0]^T$, the expectation of squared distance to the optimum $x^*$ is
	\begin{equation}
	\begin{aligned}
		\E (x_{t+1} - x^{*})^2 & = (\mat{e}^{\top}_1 \mat{A}^t [x_1 - x^{*}, x_0-x^{*}]^{\top})^2 \\
		& + \alpha^2 C \mat{e}^{\top}_1 (\mat{I} - \mat{B}^t)(\mat{I} - \mat{B})^{-1}\mat{e}_1	,
		\label{equ:squared_dist_exact}	
	\end{aligned}
	\end{equation}
where the first and second term correspond to squared bias 
and variance, and their corresponding momentum dynamics are captured by operators
	\begin{equation}
	\begin{gathered}
		\mat{A} = \begin{bmatrix}
		1-\alpha h + \mu & - \mu\\
		1 & 0 \\
		\end{bmatrix}, \\
%		\quad
		\mat{B} = 
		\begin{bmatrix}
		(1-\alpha h + \mu)^2 &  \mu^2 & -2\mu(1-\alpha h + \mu)\\
		1 & 0 & 0 \\
		1-\alpha h + \mu & 0 & - \mu
		\end{bmatrix}.
		\label{equ:mat_def}
	\end{gathered}
	\end{equation}
\end{lemma}

\yell{
Even though it is possible to numerically work on~\eqref{equ:squared_dist_exact} directly,
we use a scalar, asymptotic surrogate in~\eqref{eqn:asymptotic_surrogate} based on the spectral radii of operators to simplify analysis and expose insights.
This decision is supported by our findings in Section~\ref{sec:momentum_operator}: the spectral radii can capture empirical convergence rate.}
\begin{equation}
\begin{aligned}
%	\E ( x_{t+1} - x^{*} )^2
%	\approx  \rho(\mat{A})^{2t} ( x_0 - x_{*} )^2 
%		+ (1-\rho(\mat{B})^{t}) \frac{\alpha^2 C}{1-\rho(\mat{B})}
	&\E ( x_{t+1} - x^{*} )^2 \\
	\approx & \rho(\mat{A})^{2t} ( x_0 - x_{*} )^2 
		+ (1-\rho(\mat{B})^{t}) \frac{\alpha^2 C}{1-\rho(\mat{B})}
	\label{eqn:asymptotic_surrogate}
\end{aligned}
\end{equation}

One of our design decisions for \tuner 
is to always work in the robust region of Lemma~\ref{lem:robustness}.
We know that this implies a spectral radius $\sqrt{\mu}$ of the momentum operator, $\mat{A}$, for the bias. 
Lemma~\ref{lem:spectral_var_control} shows that under the exact same condition, the variance operator $\mat{B}$ has spectral radius $\mu$.
 

\begin{lemma}
\label{lem:spectral_var_control}
The spectral radius of the variance operator, $\mat{B}$ is $\mu$, if ${(1-\sqrt{\mu})^2} \leq  \alpha h \leq {(1+\sqrt{\mu})^2}$.
\end{lemma}

As a result, the surrogate objective of \eqref{eqn:asymptotic_surrogate}, takes the following form in the robust region.  
\begin{equation}
	\E ( x_{t+1} - x^{*} )^2 
	\approx \mu^t ( x_0 - x^{*} )^2
		+ (1-\mu^t) \frac{\alpha^2 C}{1-\mu}
	\label{eqn:noisy_square_dist}
\end{equation}
We extend this surrogate to multidimensional cases to extract a noisy tuning rule for \tuner.

%\vspace{-0.25em}
\subsection{Tuning rule}
\label{sec:tuner}
\vspace{-0.25em}

In this section, we present \textsc{SingleStep}, the tuning rule of YellowFin (Algorithm~\ref{alg:basic-algo}). Based on the surrogate in~\eqref{eqn:noisy_square_dist}, \textsc{SingleStep} is a multidimensional SGD version of the noiseless tuning rule in~\eqref{eqn:noiseless_tuning_rule}. We first generalize~\eqref{eqn:noiseless_tuning_rule} and~\eqref{eqn:noisy_square_dist} to multidimensional cases, and then discuss \textsc{SingleStep}.% in details.

As discussed in Section~\ref{sec:robust_properties}, GCN $\nu$ captures the dynamic range of generalized curvatures in a one-dimensional objective with varying curvature. The consequent robust region described by~\eqref{eqn:noiseless_tuning_rule} implies homogeneous spectral radii. 
%\emph{In multidimensional cases, we use a single learning rate and momentum for the entire model.} 
On a multidimensional non-convex objective, each one-dimensional slice passing a minimum $x^*$ can have \emph{varying curvature}. As we use \emph{a single $\mu$ and $\alpha$ for the entire model}, if $\nu$ simultaneously captures the dynamic range of generalized curvature over all these slices, $\mu$ and $\alpha$ in~\eqref{eqn:noiseless_tuning_rule} are in the robust region for all these slices. This implies homogeneous spectral radii $\sqrt{\mu}$ according to Lemma~\ref{lem:robustness}, empirically facilitating convergence at a common rate along all the directions. 

Given homogeneous spectral radii $\sqrt{\mu}$ along all directions, the surrogate in~\eqref{eqn:noisy_square_dist} generalizes on the local quadratic approximation of multiple dimensional objectives. On this approximation with minimum $x^*$, the expectation of squared distance to $x^*$, $\E \| x_0 - x^*\|^2$, decomposes into independent scalar components along the eigenvectors of the Hessian. We define gradient variance $C$ as the sum of gradient variance along these eigenvectors. The one-dimensional surrogates in~\eqref{eqn:noisy_square_dist} for the independent components sum to $\mu^t\| x_0 - x^* \|^2 + (1-\mu^t)\alpha^2 C / (1 - \mu)$, the \emph{multidimensional surrogate} corresponding to the one in~\eqref{eqn:noisy_square_dist}. 
%Given homogeneous spectral radii $\sqrt{\mu}$ along all directions, the surrogate in~\eqref{eqn:noisy_square_dist} generalizes via multidimensional local quadratic approximations. Assuming the quadratic approximation aligns with standard axes, the expectation of squared distance to the optimum of the approximation, $\E\|x_t - x^*\|^2$, decomposes along the standard axes. As the initial squared distance $\| x_0 - x^*\|^2$ and gradient variance $C$ also decompose along the axes, the one dimensional surrogates along axes sums to $\mu^t\| x_0 - x^* \|^2 + (1-\mu^t)\alpha^2 C / (1 - \mu)$, the surrogates corresponding to the one in~\eqref{eqn:noisy_square_dist}. 
%	Note for quadratic approximation not aligned with the axes, this \emph{multidimensional surrogates} is attained by decomposing along eigenvectors of the quadratic approximation's Hessian, instead of standard axes. 

%%\begin{minipage}{0.5\linewidth}
%%\vspace{-0.25em}
%\begin{equation}
%\begin{aligned}
%	\textsc{(SingleStep)} \notag \\
%	 \mu_t, \alpha_t & = && \arg \min_{\mu} \mu D^2
%		+ \alpha^2 C \\
%	s.t. & &&\mu \geq \left(\frac{\sqrt{h_{\max}/h_{\min} }-1}{\sqrt{h_{\max}/h_{\min}}+1}\right)^2 \\
%	& &&\alpha = \frac{(1-\sqrt{\mu})^2}{h_{\min}}
%\end{aligned}
%\label{equ:noisy_min}
%\end{equation}
%%\end{minipage}
%\begin{minipage}{0.025\linewidth}
%\ 
%\end{minipage}
%\begin{minipage}{0.45\linewidth}
%\vspace{-1.5em}
%\begin{algorithm}[H]
%	\footnotesize
%	\caption{\jianedits{\tuner}}
%	\begin{algorithmic}
%	\State \textbf{state: } $\alpha \gets 1.0$, $\mu \gets 0.0$%, $w\gets20$
%	\Function{\tuner}{$\text{gradient } g_t$, $\beta$}
%	\State $h_{\max}, h_{\min} \gets \Call{CurvatureRange}{g_t, \beta}$
%	\State $C \gets \Call{Variance}{g_t, \beta}$ 
%	\State $D \gets \Call{Distance}{g_t, \beta}$ 
%
%	\State $\mu_t, \alpha_t \gets \Call{SingleStep}{C, D, h_{\max}, h_{\min}}$
%	\State $\mu \gets \beta \cdot \mu + (1 - \beta) \cdot \mu_t$
%	\State $\alpha \gets \beta \cdot \alpha + (1 - \beta) \cdot \alpha_t$ %\Comment{Smoothing learning rate and momentum for stable control}
%	\Return $\mu, \alpha$
%	\EndFunction
%	\end{algorithmic}
%	\label{alg:basic-algo}
%\end{algorithm}
%\end{minipage}
%%%%%%%%%%%%%%%%%%%%%%%%% new algorithm %%%%%%%%%%%%%%%%%%%%%%%%
%%\begin{minipage}{0.475\linewidth}
%%\vspace{-0.25em}
%\begin{algorithm}[H]
%	\footnotesize
%	\caption{\jianedits{\tuner}}
%	\begin{algorithmic}
%%	\State \textbf{state: } $\alpha \gets 1.0$, $\mu \gets 0.0$%, $w\gets20$
%	\Function{\tuner}{$\text{gradient } g_t$, $\beta$}
%	\State $h_{\max}, h_{\min} \gets \Call{CurvatureRange}{g_t, \beta}$
%	\State $C \gets \Call{Variance}{g_t, \beta}$ 
%	\State $D \gets \Call{Distance}{g_t, \beta}$ 
%
%	\State $\mu_t, \alpha_t \gets \Call{SingleStep}{C, D, h_{\max}, h_{\min}}$
%%	\State $\mu_t, \alpha_t \gets \Call{SingleStep}{C, D, h_{\max}, h_{\min}}$
%%	\State $\mu \gets \beta \cdot \mu + (1 - \beta) \cdot \mu_t$
%%	\State $\alpha \gets \beta \cdot \alpha + (1 - \beta) \cdot \alpha_t$ %\Comment{Smoothing learning rate and momentum for stable control}
%	\Return $\mu_t, \alpha_t$
%	\EndFunction
%	\end{algorithmic}
%	\label{alg:basic-algo}
%\end{algorithm}
%%\end{minipage}
%\begin{minipage}{0.475\linewidth}
%\vspace{-0.25em}
\vspace{-0.25em}
\begin{algorithm}[h]
%	\footnotesize
	\caption{\jianedits{\tuner}}
	\begin{algorithmic}
%	\State \textbf{state: } $\alpha \gets 1.0$, $\mu \gets 0.0$%, $w\gets20$
	\Function{\tuner}{$\text{gradient } g_t$, $\beta$}
	\State $h_{\max}, h_{\min} \gets \Call{CurvatureRange}{g_t, \beta}$
	\State $C \gets \Call{Variance}{g_t, \beta}$ 
	\State $D \gets \Call{Distance}{g_t, \beta}$ 

	\State $\mu_t, \alpha_t \gets \Call{SingleStep}{C, D, h_{\max}, h_{\min}}$
%	\State $\mu_t, \alpha_t \gets \Call{SingleStep}{C, D, h_{\max}, h_{\min}}$
%	\State $\mu \gets \beta \cdot \mu + (1 - \beta) \cdot \mu_t$
%	\State $\alpha \gets \beta \cdot \alpha + (1 - \beta) \cdot \alpha_t$ %\Comment{Smoothing learning rate and momentum for stable control}
	\Return $\mu_t, \alpha_t$
	\EndFunction
	\end{algorithmic}
	\label{alg:basic-algo}
\end{algorithm}
\vspace{-0.25em}
%\end{minipage}
%%%%%%%%%%%%%%%%%%%%%%%%%%%%%%%%%%%%%%%%%%%%%%%%%%%%%%%%%%%%%%%%%%%%%%%
%%%%%%%%%%%%%%%%%%%%%%%%%%% new algorithm %%%%%%%%%%%%%%%%%%%%%%%%%%%%%%%%%%%%%%%%
\begin{table*}[t]
\begin{minipage}{0.37\textwidth}
\vspace{-1em}
\algrenewcommand\alglinenumber[1]{\scriptsize #1:}
	\begin{algorithm}[H]
	\small
	\setstretch{1.01}
	\caption{\small Curvature range}
	\begin{algorithmic}
		\State \textbf{state: } $h_{\max}$, $h_{\min}$, $h_i, \forall i \in\{1,2,3,...\}$
		\Function{CurvatureRange}{gradient $g_t$, $\beta$}
			\State $h_t \gets \| g_t \|^2$
			\State $h_{\max,t}\gets\!\!\!\max\limits_{t - w \leq i \leq t}\!h_i$, $h_{\min,t}\gets\!\!\!\min\limits_{t - w \leq i \leq t}\!h_i$
			\State $h_{\max} \gets \beta \cdot h_{\max} + (1 - \beta) \cdot h_{\max,t}$ %\hfill Smoothed largest curvature.
			\State $h_{\min} \gets \beta \cdot h_{\min} + (1 - \beta) \cdot h_{\min,t}$ %\hfill Smoothed smallest curvature.
%			\State $h_{\max} \gets \beta \  h_{\max} + (1 - \beta) \  h_{\max,t}$ %\hfill Smoothed largest curvature.
%			\State $h_{\min} \gets \beta \ h_{\min} + (1 - \beta) \ h_{\min,t}$ %\hfill Smoothed smallest curvature.
			\Return $h_{\max}$, $h_{\min}$
		\EndFunction
	\end{algorithmic}
	\label{alg:curv_func}
	\end{algorithm}
\end{minipage}
\begin{minipage}{0.315\textwidth}
\vspace{-1em}
\algrenewcommand\alglinenumber[1]{\scriptsize #1:}
	\begin{algorithm}[H]
	\small
	\setstretch{1.5}
	\caption{\small Gradient variance}
	\begin{algorithmic}
	\State \textbf{state: } $\overline{g^2}\gets0$, $\overline{g}\gets0$
	\Function{Variance}{gradient $g_t$, $\beta$}
		\State $\overline{g^2}\gets\beta \cdot \overline{g^2} + (1 - \beta) \cdot g_t \odot g_t$
		\State $\overline{g}\gets\beta \cdot \overline{g} + (1 - \beta) \cdot g_t$
%		\Return $\| \overline{g^2} - \overline{g}^2 \|_1$ %\hfill Sum of elements in the vect
		\Return $\bm{1}^T\!\!\cdot\left(\overline{g^2} - \overline{g}^2\right)$ %\hfill Sum of elements in the vect
	\EndFunction
	\end{algorithmic}
	\label{alg:var_func}
	\end{algorithm}
\end{minipage}
\begin{minipage}{0.3\textwidth}
\vspace{-1em}
\algrenewcommand\alglinenumber[1]{\scriptsize #1:}
	\begin{algorithm}[H]
	\small
	\setstretch{1.25}
	\caption{\small Distance to opt.}
	\begin{algorithmic}
	\State \textbf{state: } $\overline{\|g\|}\gets0$, $\overline{h}\gets0$
		\Function{Distance}{gradient $g_t$, $\beta$}
		\State $\overline{\|g\|}\gets \beta \cdot \overline{\|g\|} + (1 - \beta) \cdot \|g_t\|$
		\State $\overline{h} \gets \beta \cdot \overline{h} + (1 - \beta) \cdot \| g_t \|^2$
		\State $D \gets \beta \cdot D + (1 - \beta) \cdot \overline{\|g\|} /\overline{h}$
		\Return $D$
	\EndFunction
	\end{algorithmic}
	\label{alg:dist_func}
	\end{algorithm}
\end{minipage}
\end{table*}
%%%%%%%%%%%%%%%%%%%%%%%%%%%% end of new algorithm

\begin{wrapfigure}[9]{R}{0.55\linewidth}
\vspace{-2.5em}
\hspace{-1em}
\begin{minipage}{\linewidth}
	\begin{equation}
	\begin{aligned}
	&\textsc{(SingleStep)} \\
	 \mu_t, \alpha_t = & \arg \min_{\mu} \mu D^2
		+ \alpha^2 C \\
	s.t.\  \mu \geq & \left(\frac{\sqrt{h_{\max}/h_{\min} }-1}{\sqrt{h_{\max}/h_{\min}}+1}\right)^2 \\
	\alpha =& \frac{(1-\sqrt{\mu})^2}{h_{\min}}
	\end{aligned}
	\label{equ:noisy_min}
	\end{equation}
\end{minipage}
\end{wrapfigure}
Let $D$ be an estimate of the current model's distance to a local quadratic approximation's minimum, and $C$ denote an estimate for gradient variance.
\textsc{SingleStep} minimizes the \emph{multidimensional surrogate} after a single step (i.e. $t=1$) while ensuring $\mu$ and $\alpha$ in the robust region for all directions. \emph{A single instance of \textsc{SingleStep} solves a single momentum and learning rate for the entire model at each iteration.}
Specifically, the extremal curvatures $h_{min}$ and $h_{max}$ denote estimates for the largest and smallest generalized curvature respectively. They are meant to capture both generalized curvature variation along all different directions (like the classic condition number)
and also variation that occurs as the {\em landscape evolves}. The constraints keep the global learning rate and momentum in the robust region (defined in Lemma~\ref{lem:robustness}) 
for slices along all directions.
%along eigenvectors of the quadratic approximation's Hessian. 
%\textsc{SingleStep} can be solved in closed form; we refer to Appendix~\ref{sec:opt} for relevant details on the closed form solution. 

The problem in~\eqref{equ:noisy_min} does not need iterative solver but has an analytical solution. Substituting only the second constraint, the objective becomes $p(x)=x^2D^2 + (1-x)^4/h_{\min}^2C$ with $x=\sqrt{\mu} \in [0, 1)$. By setting the gradient of $p(x)$ to 0, we can get a cubic equation whose root $x=\sqrt{\mu_p}$ can be computed in closed form using Vieta's substitution. As $p(x)$ is uni-modal in $[0, 1)$, the optimizer for \eqref{equ:noisy_min} is exactly the maximum of $\mu_p$ and $(\sqrt{h_{\max}/h_{\min} }-1 )^2 / (\sqrt{h_{\max}/h_{\min}}+1)^2$, the right hand-side of the first constraint in~\eqref{equ:noisy_min}.

\tuner uses functions \textproc{CurvatureRange}, \textproc{Variance} and \textproc{Distance} to measure quantities $h_{\max}$, $h_{\min}$, $C$ and $D$ respectively. These measurement functions can be designed in different ways.
We present the implementations we used for our experiments,
based completely on gradients,  in Section~\ref{sec:oracles}.



%Let $D$ denote an estimate of the current model's distance to a local quadratic approximation's minimum, and $C$ denote an estimate for gradient variance. The extremal curvatures $h_{min}$ and $h_{max}$ denote estimates for the largest and smallest generalized curvature respectively. They are meant to capture both local curvature variation along all different directions (like the classic condition number)
%and also variation that occurs as the {\em landscape evolves}. Thus the constraints keep the global learning rate and momentum in the robust region (defined in Lemma~\ref{lem:robustness}) along all eigendirections of the quadratic approximation. According to Lemma~\ref{lem:robustness} and~\ref{lem:spectral_var_control}, these constraints support the decomposition of \textsc{SingleStep} objective, as the sum of ~\eqref{eqn:noisy_square_dist} (with $t=1$) along the eigendirections of the quadratic approximation.
%\textsc{SingleStep} can be solved in closed form; we refer to Appendix~\ref{sec:opt} for discussion on the closed form solution. 
%\tuner uses functions \textproc{CurvatureRange}, \textproc{Variance} and \textproc{Distance} to measure quantities $h_{\max}$, $h_{\min}$, $C$ and $D$ respectively. These measurement functions can be designed in different ways.
%We present the implementations we used for our experiments,
%based completely on gradients,  in Section~\ref{sec:oracles}.

%\textsc{SingleStep} minimizes the surrogate for the expected squared distance from the optimum of a local quadratic approximation  \eqref{eqn:noisy_square_dist} after a single step ($t=1$),
%while keeping all directions in the robust region \eqref{eqn:robust_region}.
%This is the SGD version of the noiseless tuning rule in \eqref{eqn:noiseless_tuning_rule}.
%It can be solved in closed form; we refer to Appendix~\ref{sec:opt} for discussion on the closed form solution. 
%\tuner uses functions \textproc{CurvatureRange}, \textproc{Variance} and \textproc{Distance} to measure quantities $h_{\max}$, $h_{\min}$, $C$ and $D$ respectively. These measurement functions can be designed in different ways.
%We present the implementations we used for our experiments,
%based completely on gradients,  in Section~\ref{sec:oracles}.




\subsection{Measurement functions in \tuner}
\label{sec:oracles}
This section describes our implementation of the measurement oracles used by \tuner: \textproc{CurvatureRange}, \textproc{Variance}, and \textproc{Distance}.
We design the measurement functions with the assumption of a negative log-probability objective; this is in line with typical losses in machine learning, e.g. cross-entropy for neural nets and maximum likelihood estimation in general.
Under this assumption, the Fisher information matrix---i.e.\ the expected outer product of noisy gradients---approximates the Hessian of the objective~\citep{johnfisherinfo2016,pascanu2013revisiting}. This allows for measurements purely being approximated from minibatch gradients with overhead linear to model dimensionality.
These implementations are not guaranteed to give accurate measurements.
Nonetheless, their use in our experiments in Section~\ref{sec:experiments} shows that they are sufficient for \tuner to outperform the state of the art on a variety of objectives. We also refer to Appendix~\ref{sec:practical_impl} for details on zero-debias~\citep{kingma2014adam}, slow start~\citep{schaul2013no} and smoothing for curvature range estimation.

%\begin{minipage}{0.37\textwidth}
%\algrenewcommand\alglinenumber[1]{\scriptsize #1:}
%	\begin{algorithm}[H]
%	\scriptsize
%	\caption{\small Curvature range}
%	\begin{algorithmic}
%		\State \textbf{state: } $h_{\max}$, $h_{\min}$, $h_i, \forall i \in\{1,2,3,...\}$
%		\Function{CurvatureRange}{gradient $g_t$, $\beta$}
%			\State $h_t \gets \| g_t \|^2$
%			\State $h_{\max,t}\!\!\gets\!\!\!\!\!\!\!\max\limits_{t - w \leq i \leq t}\!h_i$, $h_{\min,t}\!\!\gets\!\!\!\!\!\!\!\min\limits_{t - w \leq i \leq t}\!h_i$
%			\State $h_{\max} \gets \beta \cdot h_{\max} + (1 - \beta) \cdot h_{\max,t}$ %\hfill Smoothed largest curvature.
%			\State $h_{\min} \gets \beta \cdot h_{\min} + (1 - \beta) \cdot h_{\min,t}$ %\hfill Smoothed smallest curvature.
%			\Return $h_{\max}$, $h_{\min}$
%		\EndFunction
%	\end{algorithmic}
%	\label{alg:curv_func}
%	\end{algorithm}
%\end{minipage}
%\begin{minipage}{0.315\textwidth}
%\algrenewcommand\alglinenumber[1]{\scriptsize #1:}
%	\begin{algorithm}[H]
%	\scriptsize
%	\setstretch{1.5}
%	\caption{\small Gradient variance}
%	\begin{algorithmic}
%	\State \textbf{state: } $\overline{g^2}\gets0$, $\overline{g}\gets0$
%	\Function{Variance}{gradient $g_t$, $\beta$}
%		\State $\overline{g^2}\gets\beta \cdot \overline{g^2} + (1 - \beta) \cdot g_t \odot g_t$
%		\State $\overline{g}\gets\beta \cdot \overline{g} + (1 - \beta) \cdot g_t$
%		\Return $\| \overline{g^2} - \overline{g}^2 \|_1$ %\hfill Sum of elements in the vect
%	\EndFunction
%	\end{algorithmic}
%	\label{alg:var_func}
%	\end{algorithm}
%\end{minipage}
%\begin{minipage}{0.3\textwidth}
%\algrenewcommand\alglinenumber[1]{\scriptsize #1:}
%	\begin{algorithm}[H]
%	\scriptsize
%	\setstretch{1.25}
%	\caption{\small Distance to opt.}
%	\begin{algorithmic}
%	\State \textbf{state: } $\overline{\|g\|}\gets0$, $\overline{h}\gets0$
%		\Function{Distance}{gradient $g_t$, $\beta$}
%		\State $\overline{\|g\|}\gets \beta \cdot \overline{\|g\|} + (1 - \beta) \cdot \|g_t\|$
%		\State $\overline{h} \gets \beta \cdot \overline{h} + (1 - \beta) \cdot \| g_t \|^2$
%		\State $D \gets \beta \cdot D + (1 - \beta) \cdot \overline{\|g\|} /\overline{h}$
%		\Return $D$
%	\EndFunction
%	\end{algorithmic}
%	\label{alg:dist_func}
%	\end{algorithm}
%\end{minipage}

\paragraph{Curvature range}
Let $g_t$ be a noisy gradient, we estimate the curvatures range in Algorithm~\ref{alg:curv_func}. We notice that the outer product $g_tg_t^T$ has an eigenvalue $h_t=\| g_t \|^2$ with eigenvector $g_t$. Thus under our negative log-likelihood assumption, we use $h_t$ to approximate the curvature of Hessian along gradient direction $g_t$. Note here we use empirical Fisher $g_tg_t^T$ instead of Fisher information matrix. Empirical Fisher is typically used in practical natural gradient methods~\citep{martens2014new, roux2008topmoumoute, duchi2011adaptive}. For practically efficient measurement, we use the empirical Fisher as a coarse proxy of Fisher information matrix which approximates the Hessian of the objective. 
Specifically in Algorithm~\ref{alg:curv_func}, we maintain $h_{\min}$ and $h_{\max}$ as running averages of extreme curvature $h_{\min, t}$ and $h_{\max, t}$, from a sliding window of width 20\footnote{We use window width 20 across all the models and experiments in our paper. We refer to Section~\ref{sec:experiments} for details on selecting the window width}.
As gradient directions evolve, we estimate curvatures along different directions. Thus $h_{\min}$ and $h_{\max}$ capture the curvature variations.

%\vspace{-0.5em}
\paragraph{Gradient variance}
To estimate the gradient variance in Algorithm~\ref{alg:var_func}, 
we use running averages $\overline{g}$ and $\overline{g^2}$ to keep track of $g_t$ and $g_t \odot g_t$, the first and second order moment of the gradient. 
As $\Var(g_t) = \E{g_t^2} - \E{g_t} \odot \E{g_t}$, we estimate the gradient variance $C$ in \eqref{equ:noisy_min} using $C=\bm{1}^T\!\!\cdot(\overline{g^2} - \overline{g}^2)$. %To get stable estimates, we use $C$, the running average of $C_t$ as the quantity representing gradient variance.

%\vspace{-0.25em}
\paragraph{Distance to optimum}
In Algorithm~\ref{alg:dist_func}, we estimate the distance to the optimum of the local quadratic approximation.
Inspired by the fact that $\| \nabla f(\mat{x}) \| \leq \| \mat{H} \| \| \mat{x} - \mat{x}^{\star}\|$ for a quadratic $f(x)$ with Hessian $\mat{H}$ and minimizer $\mat{x}^{*}$,  
we first maintain $\overline{h}$ and $\overline{\|g\|}$ as running averages of curvature $h_t$ and gradient norm $\| g_t \|$. Then the distance is approximated using $\overline{\|g\|} / \overline{h}$. %according to inequality $\| \nabla f(\mat{x}) \| \leq \| \mat{H} \| \| \mat{x} - \mat{x}^{\star}\|$.

%\begin{minipage}{0.37\textwidth}
%\algrenewcommand\alglinenumber[1]{\scriptsize #1:}
%	\begin{algorithm}[H]
%	\scriptsize
%	\caption{\small Curvature range}
%	\begin{algorithmic}
%		\State \textbf{state: } $h_{\max}$, $h_{\min}$, $h_i, \forall i \in\{1,2,3,...\}$
%		\Function{CurvatureRange}{gradient $g_t$, $\beta$}
%			\State $h_t \gets \| g_t \|^2$
%			\State $h_{\max,t}\!\!\gets\!\!\!\!\!\!\!\max\limits_{t - w \leq i \leq t}\!h_i$, $h_{\min,t}\!\!\gets\!\!\!\!\!\!\!\min\limits_{t - w \leq i \leq t}\!h_i$
%			\State $h_{\max} \gets \beta \cdot h_{\max} + (1 - \beta) \cdot h_{\max,t}$ %\hfill Smoothed largest curvature.
%			\State $h_{\min} \gets \beta \cdot h_{\min} + (1 - \beta) \cdot h_{\min,t}$ %\hfill Smoothed smallest curvature.
%			\Return $h_{\max}$, $h_{\min}$
%		\EndFunction
%	\end{algorithmic}
%	\label{alg:curv_func}
%	\end{algorithm}
%\end{minipage}
%\begin{minipage}{0.315\textwidth}
%\algrenewcommand\alglinenumber[1]{\scriptsize #1:}
%	\begin{algorithm}[H]
%	\scriptsize
%	\setstretch{1.5}
%	\caption{\small Gradient variance}
%	\begin{algorithmic}
%	\State \textbf{state: } $\overline{g^2}\gets0$, $\overline{g}\gets0$
%	\Function{Variance}{gradient $g_t$, $\beta$}
%		\State $\overline{g^2}\gets\beta \cdot \overline{g^2} + (1 - \beta) \cdot g_t \odot g_t$
%		\State $\overline{g}\gets\beta \cdot \overline{g} + (1 - \beta) \cdot g_t$
%		\Return $\| \overline{g^2} - \overline{g}^2 \|_1$ %\hfill Sum of elements in the vect
%	\EndFunction
%	\end{algorithmic}
%	\label{alg:var_func}
%	\end{algorithm}
%\end{minipage}
%\begin{minipage}{0.3\textwidth}
%\algrenewcommand\alglinenumber[1]{\scriptsize #1:}
%	\begin{algorithm}[H]
%	\scriptsize
%	\setstretch{1.25}
%	\caption{\small Distance to opt.}
%	\begin{algorithmic}
%	\State \textbf{state: } $\overline{\|g\|}\gets0$, $\overline{h}\gets0$
%		\Function{Distance}{gradient $g_t$, $\beta$}
%		\State $\overline{\|g\|}\gets \beta \cdot \overline{\|g\|} + (1 - \beta) \cdot \|g_t\|$
%		\State $\overline{h} \gets \beta \cdot \overline{h} + (1 - \beta) \cdot \| g_t \|^2$
%		\State $D \gets \beta \cdot D + (1 - \beta) \cdot \overline{\|g\|} /\overline{h}$
%		\Return $D$
%	\EndFunction
%	\end{algorithmic}
%	\label{alg:dist_func}
%	\end{algorithm}
%\end{minipage}


























\subsection{Stability on non-smooth objectives}
\label{sec:stability}

\begin{figure}
%	%\begin{minipage}{0.61\textwidth}
\centering
%% \vspace{-2.25em}
  \includegraphics[width=\linewidth]{experiment_results/clipping_example.pdf} 
% \vspace{-0.75em}
\caption{A variation of the LSTM architecture in \citep{zhu2016trained} exhibits exploding gradients.
The proposed adaptive gradient clipping threshold (blue) stabilizes the training loss.}
\label{fig:stability}
%\end{minipage}
\end{figure}

The process of training neural networks is inherently non-stationary, with the landscape abruptly switching from flat to steep areas. 
In particular, the objective functions of RNNs with hidden units can exhibit occasional but very steep slopes \citep{pascanu2013difficulty,szegedy2013intriguing}.
To deal with this issue, gradient clipping has been established in literature as a standard tool to stabilize the training using such objectives \citep{pascanu2013difficulty,Goodfellow-et-al-2016,gehring2017convolutional}. 
%When that happens, the smoothed statistics from our measurement functions might not be representative.
%So, it is not always safe to assume that the smoothed statistics from our measurement functions so far will accurately represent the objective with abruptly large gradient in the next step.

We use \emph{adaptive gradient clipping} heuristics as a very natural addition to our basic tuner. 
However, the classic tradeoff between adaptivity and stability applies: 
setting a clipping threshold that is too low can hurt performance;
setting it to be high, can compromise stability.
\tuner, keeps running estimates of extremal gradient magnitude squares, $h_{max}$ and $h_{min}$ in order to estimate a generalized condition number.
We posit that $\sqrt{h_{max}}$ is an ideal gradient norm threshold for adaptive clipping.
In order to ensure robustness to extreme gradient spikes, like the ones in Figure~\ref{fig:stability}, we also limit the growth rate of the envelope $h_{max}$ in Algorithm~\ref{alg:curv_func} as follows:
\begin{equation}
 h_{max} 
 \leftarrow
 \beta \cdot h_{max}
 	+ (1-\beta) \cdot \textrm{min}\left\{
 		h_{max,t}, 100 \cdot h_{max}
 	\right\}
\end{equation}
Our heuristics follows along the lines of classic recipes like~\cite{pascanu2013difficulty}. However, instead of using the average gradient norm to clip, it uses a running estimate of the maximum norm $h_{\max}$. In Figure~\ref{fig:stability}, we demonstrate the mechanism of our heuristic by presenting an example of an LSTM that exhibits the 'exploding gradient' issue. The proposed adaptive clipping can stabilize the training process using \tuner and prevent large catastrophic loss spikes.  
\label{sec:infl_clip}
\begin{figure}
\centering
\begin{tabular}{c@{\hspace{0.0em}} c}
	\includegraphics[width=0.49\linewidth]{experiment_results/ptb/clip_cmp.pdf} &
	\includegraphics[width=0.49\linewidth]{experiment_results/resnet/cifar10_clip_cmp.pdf}
\end{tabular}
\caption{Training losses on PTB LSTM (left) and CIFAR10 ResNet (right) for YellowFin with and without adaptive clipping.}
\label{fig:infl_clip}
\end{figure}


%\begin{table*}
%\begin{minipage}{0.61\textwidth}
%\centering
%% \vspace{-2.25em}
%  \includegraphics[width=0.95\linewidth]{experiment_results/clipping_example.pdf} 
% \vspace{-0.75em}
%\captionof{figure}{A variation of the LSTM architecture in \citep{zhu2016trained} exhibits exploding gradients.
%The proposed adaptive gradient clipping threshold (blue) stabilizes the training loss.}
%\label{fig:stability}
%\end{minipage}
%\begin{minipage}{0.01\textwidth}
%	 \ 
%\end{minipage}
%\begin{minipage}{0.37\textwidth}
%\centering
%%\small
%\vspace{0.5em}
%\begin{tabular} {@{\hspace{0.2em}}c | c | c @{\hspace{0.2em}}}
%\toprule
%	& Loss & BLEU4 \\
%\midrule
%\midrule
%	Default w/o clip. & \multicolumn{2}{c}{diverge} \\ [0.3em]
%	Default w/ clip. & 2.86 & 30.75 \\ [0.3em]
%	YF & \textbf{2.75} & \textbf{31.59} \\
%\bottomrule
%\end{tabular}
%\vspace{2.25em}
%\captionof{table}{German-English translation validation performance using convolutional seq-to-seq learning.}
%\label{tab:conv_seq}
%\end{minipage}
%%\vspace{-0.75em}
%\end{table*}

%Section~\ref{sec:oracles} describes the core measurement functions for \tuner tuner. To support its tuning rule, \tuner calculates smoothed, rough approximations of curvature ranges, a distance from a local minimum and gradient variance.
%Neural network objectives can involve arbitrary non-linearities, and large Lipschitz constants \citep{szegedy2013intriguing}.
%Furthermore, the process of training them is inherently non-stationary, with the landscape abruptly switching from flat to steep areas. 
%The process of training neural networks is inherently non-stationary, with the landscape abruptly switching from flat to steep areas. 
%In particular, the objective functions of RNNs with hidden units can exhibit occasional but very steep slopes \citep{pascanu2013difficulty,szegedy2013intriguing}.
%%When that happens, the smoothed statistics from our measurement functions might not be representative.
%%So, it is not always safe to assume that the smoothed statistics from our measurement functions so far will accurately represent the objective with abruptly large gradient in the next step.
%To deal with this issue, we use \emph{adaptive gradient clipping} heuristics as a very natural addition to our basic tuner. It is discussed with extensive details in Appendix~\ref{sec:adapt_clip}.  
%In Figure~\ref{fig:stability} in Appendix~\ref{sec:adapt_clip}, we present an example of an LSTM that exhibits the 'exploding gradient' issue. The proposed adaptive clipping can stabilize the training process using \tuner and prevent large catastrophic loss spikes.
%\begin{table}
%\centering
%\begin{tabular} { c | c | c | c}
%\toprule
%	& Default w/o clip. & Default w/ clip. & YF \\
%\midrule
%\midrule
%	Validation loss & diverge & 2.86 & 2.75 \\
%	Validation BLEU4 & diverge & 30.75 & 31.59 \\ 
%\bottomrule
%\end{tabular}
%\caption{German-English translation performance using convolutional sequence to sequence learning.}
%\label{tab:conv_seq}
%\end{table}

\begin{wrapfigure}[10]{r}{0.48\linewidth}
%asdfad
\hspace{-0.75em}
\begin{minipage}{\linewidth}
%%\begin{table}[h]
%\centering
\small
%\hspace{-1em}
\vspace{-0.75em}
\begin{tabular} {@{\hspace{0.1em}}c@{\hspace{0.3em}} | @{\hspace{0.35em}}c@{\hspace{0.4em}}c @{\hspace{0.1em}}}
\toprule
	& Loss & BLEU4 \\
\midrule
\midrule
	Default w/o clip. & \multicolumn{2}{c}{diverge} \\ [0.3em]
	Default w/ clip. & 2.86 & 30.75 \\ [0.3em]
	YF & \textbf{2.75} & \textbf{31.59} \\
\bottomrule
\end{tabular}
%\hspace{-1em}
\captionof{table}{German-English translation validation metrics using convolutional seq-to-seq model.}
%\vspace{2.25em}
\label{tab:conv_seq}
\end{minipage}
\end{wrapfigure}
We validate the proposed adaptive clipping on the convolutional sequence to sequence learning model \citep{gehring2017convolutional} for IWSLT 2014 German-English translation. The default optimizer~\citep{gehring2017convolutional} uses learning rate $0.25$ and Nesterov's momentum $0.99$, diverging to loss overflow due to 'exploding gradient'. It requires, as in~\citet{gehring2017convolutional}, strict manually set gradient norm threshold $0.1$ to stabilize. 
%We train the model for 120 epochs and report the best validation loss, as well as the best validation BLEU4 score. 
%We follow the default optimizer setting in~\citep{gehring2017convolutional}, where manually set strict clipping is applied before performing SGD with learning rate $0.25$ and Nesterov's momentum set to $0.99$. 
%The optimizer diverges when the clipping is removed. 
In Table~\ref{tab:conv_seq}, we can see YellowFin, with adaptive clipping, outperforms the default optimizer using manually set clipping, with 0.84 higher validation BLEU4 after 120 epochs. 
To further demonstrate the practical applicability of our gradient clipping heuristics, in Figure~\ref{fig:infl_clip}, we demonstrate that the adaptive clipping does not hurt performance on models that do not exhibit instabilities without clipping. Specifically, for both PTB LSTM and CIFAR10 ResNet, the difference between \tuner with and without adaptive clipping diminishes quickly. 
%Thinking fast and slow approach:
%- slow layer, the basic tuner we described
%- fast layer: applies clipping based on the statistics estimated
%- *and* we don’t let the estimates grow too quickly


%%%%%%%%%%%%%%%%%%% latest backup version %%%%%%%%%%%%%%%%%%%%%%%%%%%%%%%%%%%%%

%%\begin{table*}
%%\begin{minipage}{0.61\textwidth}
%%\centering
%%% \vspace{-2.25em}
%%  \includegraphics[width=0.95\linewidth]{experiment_results/clipping_example.pdf} 
%% \vspace{-0.75em}
%%\captionof{figure}{A variation of the LSTM architecture in \citep{zhu2016trained} exhibits exploding gradients.
%%The proposed adaptive gradient clipping threshold (blue) stabilizes the training loss.}
%%\label{fig:stability}
%%\end{minipage}
%%\begin{minipage}{0.01\textwidth}
%%	 \ 
%%\end{minipage}
%%\begin{minipage}{0.37\textwidth}
%%\centering
%%%\small
%%\vspace{0.5em}
%%\begin{tabular} {@{\hspace{0.2em}}c | c | c @{\hspace{0.2em}}}
%%\toprule
%%	& Loss & BLEU4 \\
%%\midrule
%%\midrule
%%	Default w/o clip. & \multicolumn{2}{c}{diverge} \\ [0.3em]
%%	Default w/ clip. & 2.86 & 30.75 \\ [0.3em]
%%	YF & \textbf{2.75} & \textbf{31.59} \\
%%\bottomrule
%%\end{tabular}
%%\vspace{2.25em}
%%\captionof{table}{German-English translation validation performance using convolutional seq-to-seq learning.}
%%\label{tab:conv_seq}
%%\end{minipage}
%%%\vspace{-0.75em}
%%\end{table*}
%
%%Section~\ref{sec:oracles} describes the core measurement functions for \tuner tuner. To support its tuning rule, \tuner calculates smoothed, rough approximations of curvature ranges, a distance from a local minimum and gradient variance.
%%Neural network objectives can involve arbitrary non-linearities, and large Lipschitz constants \citep{szegedy2013intriguing}.
%%Furthermore, the process of training them is inherently non-stationary, with the landscape abruptly switching from flat to steep areas. 
%The process of training neural networks is inherently non-stationary, with the landscape abruptly switching from flat to steep areas. 
%In particular, the objective functions of RNNs with hidden units can exhibit occasional but very steep slopes \citep{pascanu2013difficulty,szegedy2013intriguing}.
%%When that happens, the smoothed statistics from our measurement functions might not be representative.
%%So, it is not always safe to assume that the smoothed statistics from our measurement functions so far will accurately represent the objective with abruptly large gradient in the next step.
%To deal with this issue, we use \emph{adaptive gradient clipping} heuristics as a very natural addition to our basic tuner. It is discussed with extensive details in Appendix~\ref{sec:adapt_clip}.  
%In Figure~\ref{fig:stability} in Appendix~\ref{sec:adapt_clip}, we present an example of an LSTM that exhibits the 'exploding gradient' issue. The proposed adaptive clipping can stabilize the training process using \tuner and prevent large catastrophic loss spikes.
%%\begin{table}
%%\centering
%%\begin{tabular} { c | c | c | c}
%%\toprule
%%	& Default w/o clip. & Default w/ clip. & YF \\
%%\midrule
%%\midrule
%%	Validation loss & diverge & 2.86 & 2.75 \\
%%	Validation BLEU4 & diverge & 30.75 & 31.59 \\ 
%%\bottomrule
%%\end{tabular}
%%\caption{German-English translation performance using convolutional sequence to sequence learning.}
%%\label{tab:conv_seq}
%%\end{table}
%
%\begin{wrapfigure}[10]{r}{0.48\linewidth}
%%asdfad
%\hspace{-0.75em}
%\begin{minipage}{\linewidth}
%%%\begin{table}[h]
%%\centering
%\small
%%\hspace{-1em}
%\vspace{-0.75em}
%\begin{tabular} {@{\hspace{0.1em}}c@{\hspace{0.3em}} | @{\hspace{0.35em}}c@{\hspace{0.4em}}c @{\hspace{0.1em}}}
%\toprule
%	& Loss & BLEU4 \\
%\midrule
%\midrule
%	Default w/o clip. & \multicolumn{2}{c}{diverge} \\ [0.3em]
%	Default w/ clip. & 2.86 & 30.75 \\ [0.3em]
%	YF & \textbf{2.75} & \textbf{31.59} \\
%\bottomrule
%\end{tabular}
%%\hspace{-1em}
%\captionof{table}{German-English translation validation metrics using convolutional seq-to-seq model.}
%%\vspace{2.25em}
%\label{tab:conv_seq}
%\end{minipage}
%\end{wrapfigure}
%We validate the proposed adaptive clipping on the convolutional sequence to sequence learning model \citep{gehring2017convolutional} for IWSLT 2014 German-English translation. The default optimizer~\citep{gehring2017convolutional} uses learning rate $0.25$ and Nesterov's momentum $0.99$, diverging to loss overflow due to 'exploding gradient'. It requires, as in~\citet{gehring2017convolutional}, strict manually set gradient norm threshold $0.1$ to stabilize. 
%%We train the model for 120 epochs and report the best validation loss, as well as the best validation BLEU4 score. 
%%We follow the default optimizer setting in~\citep{gehring2017convolutional}, where manually set strict clipping is applied before performing SGD with learning rate $0.25$ and Nesterov's momentum set to $0.99$. 
%%The optimizer diverges when the clipping is removed. 
%In Table~\ref{tab:conv_seq}, we can see YellowFin, with adaptive clipping, outperforms the default optimizer using manually set clipping, with 0.84 higher validation BLEU4 after 120 epochs.
%%Thinking fast and slow approach:
%%- slow layer, the basic tuner we described
%%- fast layer: applies clipping based on the statistics estimated
%%- *and* we don’t let the estimates grow too quickly
%%%%%%%%%%%%%%%%%%%%%%%%%%%%%%%%%%%%%%%%%%%%%%%%%%%%%%%%%%%%%%%%%%%%%%%%%%%%%%%%%


%\begin{table}[t]
%\begin{minipage}{0.62\textwidth}
%\centering
%  \includegraphics[width=\linewidth]{experiment_results/clipping_example.pdf} 
% \vspace{-1.5em}
%\captionof{figure}{A variation of the LSTM architecture in \citep{zhu2016trained} exhibits exploding gradients.
%The proposed adaptive gradient clipping threshold (blue) stabilizes the training loss.}
%\label{fig:stability}
%\end{minipage}
%\begin{minipage}{0.02\textwidth}
%	 \ 
%\end{minipage}
%\begin{minipage}{0.35\textwidth}
%\centering
%\small
%\vspace{1.5em}
%\begin{tabular} {@{\hspace{0.2em}}c | c | c @{\hspace{0.2em}}}
%\toprule
%	& Loss & BLEU4 \\
%\midrule
%\midrule
%	Default w/o clip. & \multicolumn{2}{c}{diverge} \\ [0.3em]
%	Default w/ clip. & 2.86 & 30.75 \\ [0.3em]
%	YF & \textbf{2.75} & \textbf{31.59} \\
%\bottomrule
%\end{tabular}
%\vspace{1em}
%\caption{German-English translation validation performance using convolutional seq-to-seq learning.}
%\label{tab:conv_seq}
%\end{minipage}
%\end{table}
%
%
%Section~\ref{sec:sync_tuner} describes the core of the \tuner tuner.
%It uses the basic tuning rules extracted from a noisy quadratic model.
%To engineer our implementation on arbitrary objectives, we calculate smoothed, rough approximations of curvature ranges, a distance from a local minimum and gradient variance.
%However, certain neural network objectives can involve arbitrary non-linearities, and large Lipschitz constants \citep{szegedy2013intriguing}.
%Furthermore, the process of training them is inherently non-stationary, with the landscape abruptly switching from flat to steep areas. 
%In particular, the objective functions associated with RNNs   with hidden units can exhibit occasional but very steep slopes \citep{pascanu2013difficulty}.
%So, it is not always safe to assume that the statistics we calculated so far will accurately represent the objective function in the next step.
%
%In Figure~\ref{fig:stability}, we present such an example of an LSTM that exhibits this 'exploding gradient' issue.
%To deal with this issue, we propose a very natural addition to our basic tuner, that performs {\em adaptive gradient clipping}. 
%Gradient clipping has been established in literature as a standard---almost necessary---tool for training such objectives \citep{pascanu2013difficulty,Goodfellow-et-al-2016,gehring2017convolutional}. 
%However, the classic tradeoff between adaptivity and stability applies: 
%setting a clipping threshold that is too low can hurt performance;
%setting it to be high, can compromise stability.
%\tuner, keeps running estimates of extremal gradient magnitude squares, $h_{max}$ and $h_{min}$ in order to estimate a generalized condition number.
%We posit that $\sqrt{h_{max}}$ is an ideal gradient norm threshold for adaptive clipping.
%In order to ensure robustness to extreme gradient spikes, like the ones in Figure~\ref{fig:stability}, we also limit the growth rate of the envelope $h_{max}$ in Algorithm~\ref{alg:curv_func} as follows:
%\begin{equation}
% h_{max} 
% \leftarrow
% \beta \cdot h_{max}
% 	+ (1-\beta) \cdot \textrm{min}\left\{
% 		h_{max,t}, 100 \cdot h_{max}
% 	\right\}
%\end{equation}
%%\begin{table}
%%\centering
%%\begin{tabular} { c | c | c | c}
%%\toprule
%%	& Default w/o clip. & Default w/ clip. & YF \\
%%\midrule
%%\midrule
%%	Validation loss & diverge & 2.86 & 2.75 \\
%%	Validation BLEU4 & diverge & 30.75 & 31.59 \\ 
%%\bottomrule
%%\end{tabular}
%%\caption{German-English translation performance using convolutional sequence to sequence learning.}
%%\label{tab:conv_seq}
%%\end{table}
%
%We demonstrate the performance of YellowFin with adaptive clipping on the IWSLT 2014 German-English translation task using the convolutional sequence to sequence learning model of \citep{gehring2017convolutional}. We train the model for 120 epochs and report the best validation loss, as well as the best validation BLEU4 score. We follow the default optimizer setting in~\citep{gehring2017convolutional}, where manually set strict clipping is applied before performing SGD with learning rate $0.25$ and Nesterov's momentum set to $0.99$. The optimizer diverges when the clipping is removed. In Table~\ref{tab:conv_seq}, we can see YellowFin, with adaptive clipping, can outperform the default optimizer with manually set clipping in both validation loss and BLEU4 score.
%Our proposed adaptive clipping helps stabilize difficult objectives, without sacrificing performance.
%In Appendix~\ref{sec:infl_clip} we demonstrate that on models that do not exhibit instabilities, our clipping does not hurt performance.
%
%%Thinking fast and slow approach:
%%- slow layer, the basic tuner we described
%%- fast layer: applies clipping based on the statistics estimated
%%- *and* we don’t let the estimates grow too quickly
%

%\section{Stability on non-smooth objectives}
%\label{sec:stability}
%
%\begin{figure}[t]
%\centering
%  \includegraphics[width=0.8\linewidth]{experiment_results/clipping_example.pdf} 
%\caption{An LSTM with hidden units exhibits exploding gradients.
%The proposed adaptive threshold for gradient clipping (blue line) keeps instabilities contained.
%The network is a small variation of the architecture presented in \citep{zhu2016trained}.}
%\label{fig:stability}
%\end{figure}
%
%
%Section~\ref{sec:sync_tuner} describes the core of the \tuner tuner.
%It uses the basic tuning rules extracted from a noisy quadratic model.
%To engineer our implementation on arbitrary objectives, we calculate smoothed, rough approximations of curvature ranges, a distance form a local minimum and a gradient variance.
%However, certain neural network objectives can involve arbitrary non-linearities, and large Lipschitz constants \citep{szegedy2013intriguing}.
%Furthermore, the process of training them is inherently non-stationary, with the landscape abruptly switching from flat to steep areas. 
%In particular, the objective functions associated with RNNs   with hidden units can exhibit occasional but very steep slopes \citep{pascanu2013difficulty}.
%So, it is not always safe to assume that the statistics we calculated so far will accurately represent the objective function in the next step.
%
%In Figure~\ref{fig:stability}, we present such an example of an LSTM that exhibits this 'exploding gradient' issue.
%To deal with this issue, we propose a very natural addition to our basic tuner, that performs {\em adaptive gradient clipping}. 
%Gradient clipping has been established in literature as a standard---almost necessary---tool for training such objectives \citep{pascanu2013difficulty,Goodfellow-et-al-2016,gehring2017convolutional}. 
%However, the classic tradeoff between adaptivity and stability applies: 
%setting a clipping threshold that is too low can hurt performance;
%setting it to be high, can compromise stability.
%\tuner, keeps running estimates of extremal gradient magnitude squares, $h_{max}$ and $h_{min}$ in order to estimate a generalized condition number.
%We posit that $\sqrt{h_{max}}$ is an ideal gradient norm threshold for adaptive clipping.
%In order to improve robustness to extreme gradient spikes, like the ones in Figure~\ref{fig:stability}, we also limit the growth rate of the envelope $h_{max}$ as follows:
%\begin{equation}
% h_{max} 
% \leftarrow
% \beta \cdot h_{max}
% 	+ (1-\beta) \cdot \textrm{min}\left\{
% 		h_{max,t}, 100 \cdot h_{max}
% 	\right\}
%\end{equation}
%\begin{table}
%\centering
%\begin{tabular} { c | c | c | c}
%\toprule
%	& Default w/o clip. & Default w/ clip. & YF \\
%\midrule
%\midrule
%	Validation loss & diverge & 2.86 & 2.75 \\
%	Validation BLEU4 & diverge & 30.75 & 31.59 \\ 
%\bottomrule
%\end{tabular}
%\caption{German-English translation performance using convolutional sequence to sequence learning.}
%\label{tab:conv_seq}
%\end{table}
%
%We demonstrate the performance of YellowFin with adaptive clipping on IWSLT 2014 German-English translation task using convolutional sequence to sequence learning~\citep{gehring2017convolutional}. We train the model for 120 epochs and report the best validation loss, as well as the best BLEU4 score. We follow the default optimizer setting in~\citep{gehring2017convolutional}, where manually set strict clipping is applied before performing SGD with learning rate 0.25 and momentum 0.99. The optimizer diverges when the clipping is removed. In Table~\ref{tab:conv_seq}, we can see YellowFin, with adaptive clipping, can outperform the default optimizer with manually set clipping in both validation loss and BLEU4 score.
%Our proposed adaptive clipping helps stabilize difficult objectives, without sacrificing performance.
%In Appendix~\ref{sec:infl_clip} we demonstrate that on models that do not exhibit instabilities, our clipping does not hurt performance.
%
%%Thinking fast and slow approach:
%%- slow layer, the basic tuner we described
%%- fast layer: applies clipping based on the statistics estimated
%%- *and* we don’t let the estimates grow too quickly
%



\begin{figure*}
%\vspace{-1em}
\centering
%\includegraphics[width=0.99\linewidth, trim={10cm 0 0 0},clip]{../yellowfin_iclr2018/manuscript_for_revision/experiment_results/resnet/mom_dynamic_3_annotated.pdf}
\includegraphics[width=0.99\linewidth]{../yellowfin_iclr2018/manuscript_for_revision/experiment_results/resnet/mom_dynamic_3_annotated.pdf}
	\vspace{-0.5em}
	\caption{
	When running \tuner, total momentum $\hat{\mu}_t$ equals algorithmic value in synchronous settings (left); $\hat{\mu}_t$ is greater than algorithmic value on 16 asynchronous workers (middle).
	\Asynctuner automatically lowers algorithmic momentum and brings total momentum to match the target value (right).
%Red dots are measured $\hat{\mu}_t$ at every step with red line as its running average.
	Red dots are total momentum estimates, $\hat{\mu}_T$, at each iteration. 
The solid red line is a running average of $\hat{\mu}_T$.
%	When running \tuner, total momentum $\hat{\mu}_t$ is greater than algorithmic value on 16 asynchronous workers (left).
%	\Asynctuner automatically lowers algorithmic momentum and matches total momentum to the target value (right).
%%Red dots are measured $\hat{\mu}_t$ at every step with red line as its running average.
%	Red dots are total momentum estimates, $\hat{\mu}_T$, at each iteration. 
%    The solid red line is a running average of $\hat{\mu}_T$.	
	}
	\label{fig:we-can-measure}
%\vspace{-0.35em}
\end{figure*}

\section{\Asynctuner}
\label{sec:async_tuner}

Asynchrony is a parallelization technique that avoids synchronization barriers \citep{recht2011hogwild}. 
In this section, we propose a {\em closed momentum loop} variant of \tuner to accelerate convergence in asynchronous training. 
%To handle the momentum dynamics of asynchronous parallelism, we propose a {\em closed momentum loop} variant of \tuner.
After some preliminaries, we show the mechanism of the extension: 
it measures the dynamics on a running system and controls momentum with a negative feedback loop.
\paragraph{Preliminaries}
%Asynchrony is a popular parallelization technique \citep{recht2011hogwild} that avoids synchronization barriers.
When training on $M$ asynchronous workers, staleness (the number of model updates between a worker's read and write operations) is on average $\tau=M-1$,
i.e., the gradient in the SGD update is delayed by $\tau$ iterations as $\nabla f_{S_{t - \tau}}(x_{t - \tau} )$.
Asynchrony yields faster steps, but can
increase the number of iterations to achieve the same solution,
a tradeoff between hardware and statistical 
efficiency~\citep{DBLP:journals/pvldb/ZhangR14}.
\citet{mitliagkas2016asynchrony} interpret asynchrony as added momentum dynamics.
Experiments in \citet{hadjis2016omnivore} support this finding, and demonstrate that reducing algorithmic momentum can compensate for asynchrony-induced momentum
and significantly reduce the number of iterations for convergence.
Motivated by that result, we use the model
in~\eqref{equ:exp_async_update_app}, where the total momentum, $\mu_T$, includes both asynchrony-induced and algorithmic  momentum, $\mu$, in~\eqref{eqn:momentum_gd}.
\begin{equation}
	\mathbb{E}[ x_{t+1} - x_t ] 
	= \mu_T \mathbb{E}[x_t - x_{t-1}] - \alpha \mathbb{E}\nabla f(x_{t})
\label{equ:exp_async_update_app}
\end{equation}
We will use this expression to design an estimator for the value of total momentum, $\hat{\mu}_T$.
This estimator is a basic building block of \asynctuner, that {\em removes the need to manually compensate for the effects of asynchrony}.



\paragraph{Measuring the momentum dynamics}
\Asynctuner estimates total momentum $\mu_{T}$ on a running system and uses a negative feedback loop to adjust algorithmic momentum accordingly.
Equation~\eqref{equ:exp_async_update} gives an estimate of $\hat{\mu}_T$ on a system with staleness $\tau$, based on \eqref{equ:exp_async_update}.
\begin{align}
\hat{\mu}_T
					= \mathop{\mathsf{median}}\left(
							\frac{x_{t - \tau} - x_{t - \tau-1} + \alpha \nabla_{S_{t-\tau -1}} f(x_{t - \tau - 1} )}
							{x_{t - \tau-1} - x_{t - \tau-2}}
					\right)
\label{eqn:momentum_measurement}
\end{align}
We use $\tau$-stale model values to match the staleness of the gradient,  and perform all operations in an elementwise fashion. 
This way we get a total momentum measurement from each variable; 
the median combines them into a more robust estimate.

\paragraph{Closing the asynchrony loop}
Given a reliable measurement of $\mu_{T}$, 
we can use it to adjust the value of algorithmic momentum so that the total momentum matches the \emph{target momentum} as decided by \tuner in Algorithm~\ref{alg:basic-algo}.
\Asynctuner in Algorithm~\ref{alg:async-algo} %(in Appendix~\ref{sec:async_yf}) 
uses a simple negative feedback loop to achieve the adjustment.
%Figure~\ref{fig:we-can-measure} demonstrates that under asynchrony the measured total momentum is strictly higher than the algorithmic momentum (middle plot), as expected from theory;
%closing the feedback loop (right plot) leads to total momentum matching the target momentum.
%Closing the loop, as we will see, improves performance significantly.
%Note for asynchronous-parallel training, as the estimates and parameter tuning is unstable in the beginning when there are only a small number of iterations, we use initial learning $\frac{1}{\tau + 1}$ instead of $1.0$ to prevent overflow in the beginning. 

%\begin{algorithm}[H]
%	\caption{\Asynctuner}
%	\begin{algorithmic}[1]
%%	\State Input: $\mu\gets0$, $\alpha \gets \frac{1}{\tau + 1}$, $\gamma\gets0.01, \tau$ (staleness)
%	\State Input: $\mu\gets0$, $\alpha \gets 0.0001$, $\gamma\gets0.01, \tau$ (staleness)
%	\For { $t\gets1$ to $T$}
%	\State $x_t\!\gets\!x_{t - 1} + \mu (x_{t - 1} - x_{t - 2} ) - \alpha \nabla_{S_t} f(x_{t - \tau - 1} )$
%	\State $\mu^*,\alpha \gets \Call{\tuner}{\nabla_{S_t} f(x_{t - \tau - 1} ), \beta}$ %(get momentum from the dynamic range)
%	\State $\hat{\mu_T} 
%					\gets \mathop{\mathsf{median}}\left(
%							\frac{x_{t - \tau} - x_{t - \tau-1} + \alpha \nabla_{S_{t-\tau-1}} f(x_{t - \tau - 1} )}
%							{x_{t - \tau-1} - x_{t - \tau-2}}
%					\right)$ \Comment{Measuring total momentum}
%	\State $\mu \leftarrow \mu + \gamma \cdot (\mu^* - \hat{\mu_T})$ \Comment{Closing the loop}
%	\EndFor
%\end{algorithmic}
%\label{alg:async-algo}
%\end{algorithm}




%In Section~\ref{sec:async_tuner}, we briefly discuss the mechanism of our designed \Asynctuner in asynchronous-parallel setting. In this appendix, we expand the details in total momentum estimator, $\hat{\mu_T}$, and present the full \Asynctuner in Algorithm~\ref{alg:async-algo} with extensive discussion.
%\paragraph{Measuring the momentum dynamics}
%Remember, we use the formula in~\eqref{equ:exp_async_update_app} to model the momentum dynamics in asynchronous-parallel systems
%\Asynctuner estimates total momentum $\mu_{T}$ on a running system and uses a negative feedback loop to adjust algorithmic momentum accordingly.
%\begin{equation}
%	\mathbb{E}[ x_{t+1} - x_t ] 
%	= \mu_T \mathbb{E}[x_t - x_{t-1}] - \alpha \mathbb{E}\nabla f(x_{t})
%\label{equ:exp_async_update_app}
%\end{equation}
%Equation~\eqref{eqn:momentum_measurement_app} gives an estimate of $\hat{\mu_T}$ on a system with staleness $\tau$, based on \eqref{equ:exp_async_update_app}.
%\begin{align}
%\hat{\mu_T}
%					= \mathop{\mathsf{median}}\left(
%							\frac{x_{t - \tau} - x_{t - \tau-1} + \alpha \nabla_{S_{t-\tau -1}} f(x_{t - \tau - 1} )}
%							{x_{t - \tau-1} - x_{t - \tau-2}}
%					\right)
%\label{eqn:momentum_measurement_app}
%\end{align}
%We use $\tau$-stale model values to match the staleness of the gradient,  and perform all operations in an elementwise fashion. 
%This way we get a total momentum measurement from each variable; 
%the median combines them into a more robust estimate.
%
%%\label{subsec:closed_loop_YF}
%%\begin{figure}
%%\centering
%%\includegraphics[width=0.95\linewidth]{experiment_results/resnet/mom_dynamic_3_annotated.pdf}
%%	\caption{
%%	Momentum dynamics on CIFAR100 ResNet.
%%	Running \tuner, total momentum is equal to algorithmic momentum in a synchronous setting (left). Total momentum is greater than algorithmic momentum on 16 asynchronous workers, due to asynchrony-induced momentum (middle).
%%	Using the momentum feedback mechanism of \asynctuner, lowers algorithmic momentum and brings total momentum to match the target value on 16 asynchronous workers (right).
%%	Red dots are individual total momentum estimates, $\hat{\mu}_T$, at each iteration. 
%%The solid red line is a running average of those estimates.	
%%	}
%%	\label{fig:we-can-measure}
%%\end{figure}
%
%\paragraph{Closing the asynchrony loop}
%Given a reliable measurement of $\mu_{T}$, 
%we can use it to adjust the value of algorithmic momentum so that the total momentum matches the \emph{target momentum} as decided by \tuner in Algorithm~\ref{alg:basic-algo}.
%\Asynctuner in Algorithm~\ref{alg:async-algo} %(in Appendix~\ref{sec:async_yf}) 
%uses a simple negative feedback loop to achieve the adjustment.
%%Figure~\ref{fig:we-can-measure} demonstrates that under asynchrony the measured total momentum is strictly higher than the algorithmic momentum (middle plot), as expected from theory;
%%closing the feedback loop (right plot) leads to total momentum matching the target momentum.
%%Closing the loop, as we will see, improves performance significantly.
%%%Note for asynchronous-parallel training, as the estimates and parameter tuning is unstable in the beginning when there are only a small number of iterations, we use initial learning $\frac{1}{\tau + 1}$ instead of $1.0$ to prevent overflow in the beginning. 
%%
%%%\begin{algorithm}[H]
%%%	\caption{\Asynctuner}
%%%	\begin{algorithmic}[1]
%%%%	\State Input: $\mu\gets0$, $\alpha \gets \frac{1}{\tau + 1}$, $\gamma\gets0.01, \tau$ (staleness)
%%%	\State Input: $\mu\gets0$, $\alpha \gets 0.0001$, $\gamma\gets0.01, \tau$ (staleness)
%%%	\For { $t\gets1$ to $T$}
%%%	\State $x_t\!\gets\!x_{t - 1} + \mu (x_{t - 1} - x_{t - 2} ) - \alpha \nabla_{S_t} f(x_{t - \tau - 1} )$
%%%	\State $\mu^*,\alpha \gets \Call{\tuner}{\nabla_{S_t} f(x_{t - \tau - 1} ), \beta}$ %(get momentum from the dynamic range)
%%%	\State $\hat{\mu_T} 
%%%					\gets \mathop{\mathsf{median}}\left(
%%%							\frac{x_{t - \tau} - x_{t - \tau-1} + \alpha \nabla_{S_{t-\tau-1}} f(x_{t - \tau - 1} )}
%%%							{x_{t - \tau-1} - x_{t - \tau-2}}
%%%					\right)$ \Comment{Measuring total momentum}
%%%	\State $\mu \leftarrow \mu + \gamma \cdot (\mu^* - \hat{\mu_T})$ \Comment{Closing the loop}
%%%	\EndFor
%%%\end{algorithmic}
%%%\label{alg:async-algo}
%%%\end{algorithm}
%%
%
%
%




\begin{algorithm}[h]
	\caption{\Asynctuner}
	\begin{algorithmic}[1]
%	\State Input: $\mu\gets0$, $\alpha \gets \frac{1}{\tau + 1}$, $\gamma\gets0.01, \tau$ (staleness)
	\State Input: $\mu\gets0$, $\alpha \gets 0.0001$, $\gamma\gets0.01, \tau$ (staleness)
	\For { $t\gets1$ to $T$}
	\State $x_t\!\gets\!x_{t - 1} + \mu (x_{t - 1} - x_{t - 2} ) - \alpha \nabla_{S_t} f(x_{t - \tau - 1} )$
	\State $\mu^*,\alpha \gets \Call{\tuner}{\nabla_{S_t} f(x_{t - \tau - 1} ), \beta}$ %(get momentum from the dynamic range)
	\State $\hat{\mu_T} 
					\gets \mathop{\mathsf{median}}\left(
							\frac{x_{t - \tau} - x_{t - \tau-1} + \alpha \nabla_{S_{t-\tau-1}} f(x_{t - \tau - 1} )}
							{x_{t - \tau-1} - x_{t - \tau-2}}
					\right)$ \Comment{Measuring total momentum}
	\State $\mu \leftarrow \mu + \gamma \cdot (\mu^* - \hat{\mu_T})$ \Comment{Closing the loop}
	\EndFor
\end{algorithmic}
\label{alg:async-algo}
\end{algorithm}


%%%%%%%%%%%%%%%%% latest backup version %%%%%%%%%%%%%%%%%%%%%%%%%%%%%%%%%%%%
%Asynchrony is a parallelization technique that avoids synchronization barriers \citep{recht2011hogwild}. 
%%In this section, we propose a {\em closed momentum loop} variant of \tuner to accelerate convergence in asynchronous training. 
%%To handle the momentum dynamics of asynchronous parallelism, we propose a {\em closed momentum loop} variant of \tuner.
%%After some preliminaries, we show the mechanism of the extension: 
%%it measures the dynamics on a running system and controls momentum with a negative feedback loop.
%%\paragraph{Preliminaries}
%%Asynchrony is a popular parallelization technique \citep{recht2011hogwild} that avoids synchronization barriers.
%%When training on $M$ asynchronous workers, staleness (the number of model updates between a worker's read and write operations) is on average $\tau=M-1$,
%%i.e., the gradient in the SGD update is delayed by $\tau$ iterations as $\nabla f_{S_{t - \tau}}(x_{t - \tau} )$.
%%It yields faster steps, but can
%%increase the number of iterations needed,
%%a tradeoff between hardware and statistical 
%%efficiency~\citep{DBLP:journals/pvldb/ZhangR14}.
%It yields better hardware efficiency, i.e. faster steps, but can
%increase the number of iterations to a given metric, i.e. statistical efficiency, as a tradeoff~\citep{DBLP:journals/pvldb/ZhangR14}.
%%a tradeoff between hardware and statistical 
%%efficiency~\citep{DBLP:journals/pvldb/ZhangR14}.
%%In this section, we propose a {\em closed momentum loop} variant of \tuner to reduce the number of iterations it needs to converge in asynchronous training. 
%%In this section, we propose a {\em closed momentum loop} variant of \tuner to reduce the number of iterations for convergence in asynchronous training.
%%\paragraph{\Asynctuner}
%\citet{mitliagkas2016asynchrony} interpret asynchrony as added momentum dynamics.
%%It is empirically supported in \citet{hadjis2016omnivore} that manually reducing algorithmic momentum can compensate for asynchrony-induced momentum
%%and significantly reduce the number of iterations to converge.
%We design \asynctuner, a variant of \tuner to automatically control algorithmic momentum, compensate for asynchrony and accelerate convergence.
%We use the formula in~\eqref{equ:exp_async_update} to model the dynamics in the system, where the total momentum, $\mu_T$, includes both asynchrony-induced and algorithmic  momentum, $\mu$, in~\eqref{eqn:momentum_gd}.
%\begin{equation}
%	\mathbb{E}[ x_{t+1} - x_t ] 
%	= \mu_T \mathbb{E}[x_t - x_{t-1}] - \alpha \mathbb{E}\nabla f(x_{t})
%\label{equ:exp_async_update}
%\end{equation}
%We first use~\eqref{equ:exp_async_update} to design an robust estimator $\hat{\mu}_T$ for the value of total momentum at every iteration.
%%This estimator is a basic building block of \asynctuner, that {\em removes the need to manually compensate for the effects of asynchrony}. 
%Then we use a simple negative feedback control loop to adjust the value of algorithmic momentum so that $\hat{\mu}_T$ matches the \emph{target momentum} decided by \tuner in Algorithm~\ref{alg:basic-algo}. 
%%We refer to Appendix~\ref{sec:async_app} for details on estimator $\hat{\mu}_T$ and \Asynctuner in Algorithm~\ref{alg:async-algo}.
%%\Asynctuner in Algorithm~\ref{alg:async-algo} (in Appendix~\ref{sec:async_app}) %(in Appendix~\ref{sec:async_yf}) 
%%uses a simple negative feedback loop to achieve the adjustment.
%In Figure~\ref{fig:we-can-measure}, 
%we demonstrate momentum dynamics in an asynchronous training system. 
%As directly using the target value as algorithmic momentum, \tuner (middle) presents total momentum $\hat{\mu}_T$ strictly larger than the target momentum, due to asynchrony-induced momentum. \Asynctuner (right) automatically brings down algorithmic momentum, match measured total momentum $\hat{\mu}_T$ to target value and, as we will see, speeds up convergence comparing to \tuner. We refer to Appendix~\ref{sec:async_app} for details on estimator $\hat{\mu}_T$ and \Asynctuner in Algorithm~\ref{alg:async-algo}.
%%
%%so that visually demonstrates the mechanism of \Asynctuner in handling the momentum dynamics under asynchrony. In asynchronous-parallel setting, the measured total momentum is strictly higher than the algorithmic momentum (middle plot), as expected from theory.
%%Closing the feedback loop (right plot) leads to total momentum matching the target momentum and, as we will see, improves performance significantly.
%
%%\begin{figure*}
%%%\vspace{-2.5em}
%%\centering
%%\includegraphics[width=\linewidth]{experiment_results/resnet/mom_dynamic_3_annotated.pdf}
%%	\caption{
%%	When running \tuner, total momentum $\hat{\mu}_t$ equals algorithmic value in synchronous settings (left); $\hat{\mu}_t$ is greater than algorithmic value on 16 asynchronous workers (middle).
%%	\Asynctuner automatically lowers algorithmic momentum and brings total momentum to match the target value (right).
%%Red dots are measured $\hat{\mu}_t$ at every step with red line as its running average.
%%%	Red dots are total momentum estimates, $\hat{\mu}_T$, at each iteration. 
%%%The solid red line is a running average of $\hat{\mu}_T$.	
%%	}
%%	\vspace{-0.25em}
%%	\label{fig:we-can-measure}
%%\end{figure*}


%%%%%%%%%%%%%%%%%%%%%% latest backup versions %%%%%%%%%%%%%%%%%%%%%%%%%%%%%%%%%%%%%%%%%
%%\begin{figure}
%%\centering
%%\includegraphics[width=0.95\linewidth]{experiment_results/resnet/mom_dynamic_3_annotated.pdf}
%%	\caption{
%%%	Momentum dynamics on CIFAR100 ResNet.
%%	Running \tuner on a ResNet, total momentum equals algorithmic value in a synchronous setting (left). Total momentum is greater than algorithmic value on 16 asynchronous workers, due to asynchrony-induced momentum (middle).
%%	\asynctuner automatically lowers algorithmic momentum and brings total momentum to match the target value (right).
%%	Red dots are total momentum estimates, $\hat{\mu}_T$, at each iteration. 
%%The solid red line is a running average of $\hat{\mu}_T$.	
%%	}
%%	\label{fig:we-can-measure}
%%\end{figure}
%Asynchrony is a parallelization technique that avoids synchronization barriers \citep{recht2011hogwild}. 
%%In this section, we propose a {\em closed momentum loop} variant of \tuner to accelerate convergence in asynchronous training. 
%%To handle the momentum dynamics of asynchronous parallelism, we propose a {\em closed momentum loop} variant of \tuner.
%%After some preliminaries, we show the mechanism of the extension: 
%%it measures the dynamics on a running system and controls momentum with a negative feedback loop.
%%\paragraph{Preliminaries}
%%Asynchrony is a popular parallelization technique \citep{recht2011hogwild} that avoids synchronization barriers.
%%When training on $M$ asynchronous workers, staleness (the number of model updates between a worker's read and write operations) is on average $\tau=M-1$,
%%i.e., the gradient in the SGD update is delayed by $\tau$ iterations as $\nabla f_{S_{t - \tau}}(x_{t - \tau} )$.
%%It yields faster steps, but can
%%increase the number of iterations needed,
%%a tradeoff between hardware and statistical 
%%efficiency~\citep{DBLP:journals/pvldb/ZhangR14}.
%It yields better hardware efficiency, i.e. faster steps, but can
%increase the number of iterations to a given metric, i.e. statistical efficiency, as a tradeoff~\citep{DBLP:journals/pvldb/ZhangR14}.
%%a tradeoff between hardware and statistical 
%%efficiency~\citep{DBLP:journals/pvldb/ZhangR14}.
%%In this section, we propose a {\em closed momentum loop} variant of \tuner to reduce the number of iterations it needs to converge in asynchronous training. 
%%In this section, we propose a {\em closed momentum loop} variant of \tuner to reduce the number of iterations for convergence in asynchronous training.
%%\paragraph{\Asynctuner}
%\citet{mitliagkas2016asynchrony} interpret asynchrony as added momentum dynamics.
%%It is empirically supported in \citet{hadjis2016omnivore} that manually reducing algorithmic momentum can compensate for asynchrony-induced momentum
%%and significantly reduce the number of iterations to converge.
%We design a {\em closed momentum loop} variant of \tuner to control algorithmic momentum, compensate for asynchrony and accelerate convergence.
%We use the formula in~\eqref{equ:exp_async_update} to model the dynamics in the system, where the total momentum, $\mu_T$, includes both asynchrony-induced and algorithmic  momentum, $\mu$, in~\eqref{eqn:momentum_gd}.
%\begin{equation}
%	\mathbb{E}[ x_{t+1} - x_t ] 
%	= \mu_T \mathbb{E}[x_t - x_{t-1}] - \alpha \mathbb{E}\nabla f(x_{t})
%\label{equ:exp_async_update}
%\end{equation}
%We first use this expression to design an robust estimator $\hat{\mu}_T$ for the value of total momentum.
%%This estimator is a basic building block of \asynctuner, that {\em removes the need to manually compensate for the effects of asynchrony}.
%Given $\hat{\mu}_T$,  
%we use it to adjust the value of algorithmic momentum so that the total momentum matches the \emph{target momentum} decided by \tuner in Algorithm~\ref{alg:basic-algo}. Specifically, we %\asynctuner  %(in Appendix~\ref{sec:async_yf}) 
%uses a simple negative feedback loop to achieve the adjustment. We refer to Appendix~\ref{sec:async_app} for details on estimator $\hat{\mu}_T$ and \Asynctuner in Algorithm~\ref{alg:async-algo}.
%%\Asynctuner in Algorithm~\ref{alg:async-algo} (in Appendix~\ref{sec:async_app}) %(in Appendix~\ref{sec:async_yf}) 
%%uses a simple negative feedback loop to achieve the adjustment.
%Figure~\ref{fig:we-can-measure} visually demonstrates the mechanism of \Asynctuner in handling the momentum dynamics under asynchrony. In asynchronous-parallel setting, the measured total momentum is strictly higher than the algorithmic momentum (middle plot), as expected from theory.
%Closing the feedback loop (right plot) leads to total momentum matching the target momentum and, as we will see, improves performance significantly.
%
%\begin{figure}
%\centering
%\includegraphics[width=0.95\linewidth]{experiment_results/resnet/mom_dynamic_3_annotated.pdf}
%	\caption{
%%	Momentum dynamics on CIFAR100 ResNet.
%	Running \tuner on a ResNet, total momentum equals algorithmic value in a synchronous setting (left). Total momentum is greater than algorithmic value on 16 asynchronous workers, due to asynchrony-induced momentum (middle).
%	\asynctuner automatically lowers algorithmic momentum and brings total momentum to match the target value (right).
%	Red dots are total momentum estimates, $\hat{\mu}_T$, at each iteration. 
%The solid red line is a running average of $\hat{\mu}_T$.	
%	}
%	\label{fig:we-can-measure}
%\end{figure}
%

%%%%%%%%%%%%%%%%%%%%%% below are old backup versions %%%%%%%%%%%%%%%%%%%%%%%%%%%%%%%%%%

%\paragraph{Measuring the momentum dynamics}
%\Asynctuner estimates total momentum $\mu_{T}$ on a running system and uses a negative feedback loop to adjust algorithmic momentum accordingly.
%Equation~\eqref{equ:exp_async_update} gives an estimate of $\hat{\mu_T}$ on a system with staleness $\tau$, based on \eqref{equ:exp_async_update}.
%\begin{align}
%\hat{\mu_T}
%					= \mathop{\mathsf{median}}\left(
%							\frac{x_{t - \tau} - x_{t - \tau-1} + \alpha \nabla_{S_{t-\tau -1}} f(x_{t - \tau - 1} )}
%							{x_{t - \tau-1} - x_{t - \tau-2}}
%					\right)
%\label{eqn:momentum_measurement}
%\end{align}
%We use $\tau$-stale model values to match the staleness of the gradient,  and perform all operations in an elementwise fashion. 
%This way we get a total momentum measurement from each variable; 
%the median combines them into a more robust estimate.

%\label{subsec:closed_loop_YF}
%\begin{figure}
%\centering
%\includegraphics[width=0.95\linewidth]{experiment_results/resnet/mom_dynamic_3_annotated.pdf}
%	\caption{
%	Momentum dynamics on CIFAR100 ResNet.
%	Running \tuner, total momentum is equal to algorithmic momentum in a synchronous setting (left). Total momentum is greater than algorithmic momentum on 16 asynchronous workers, due to asynchrony-induced momentum (middle).
%	Additionally applying the momentum feedback loop of \asynctuner, lowers algorithmic momentum and matches total momentum to target value (right).
%	Red dots are individual total momentum estimates, $\hat{\mu}_T$, at each iteration, with
%the solid red line as its running average.	
%	}
%	\label{fig:we-can-measure}
%\end{figure}

%\paragraph{Closing the asynchrony loop}
%Given a reliable measurement of $\mu_{T}$, 
%we can use it to adjust the value of algorithmic momentum so that the total momentum matches the \emph{target momentum} as decided by \tuner in Algorithm~\ref{alg:basic-algo}.
%\Asynctuner in Algorithm~\ref{alg:async-algo} (in Appendix~\ref{sec:async_app}) %(in Appendix~\ref{sec:async_yf}) 
%uses a simple negative feedback loop to achieve the adjustment.
%Figure~\ref{fig:we-can-measure} demonstrates that under asynchrony the measured total momentum is strictly higher than the algorithmic momentum (middle plot), as expected from theory;
%closing the feedback loop (right plot) leads to total momentum matching the target momentum.
%Closing the loop, as we will see, improves performance significantly.
%





%To handle the momentum dynamics of asynchronous parallelism, we propose a {\em closed momentum loop} variant of \tuner.
%After some preliminaries, we show the mechanism of the extension: 
%it measures the dynamics on a running system and controls momentum with a negative feedback loop.
%\paragraph{Preliminaries}
%Asynchrony is a popular parallelization technique \citep{recht2011hogwild} that avoids synchronization barriers.
%When training on $M$ asynchronous workers, staleness (the number of model updates between a worker's read and write operations) is on average $\tau=M-1$,
%i.e., the gradient in the SGD update is delayed by $\tau$ iterations as $\nabla f_{S_{t - \tau}}(x_{t - \tau} )$.
%Asynchrony yields faster steps, but can
%increase the number of iterations to achieve the same solution,
%a tradeoff between hardware and statistical 
%efficiency~\citep{DBLP:journals/pvldb/ZhangR14}.
%\citet{mitliagkas2016asynchrony} interpret asynchrony as added momentum dynamics.
%Experiments in \citet{hadjis2016omnivore} support this finding, and demonstrate that reducing algorithmic momentum can compensate for asynchrony-induced momentum
%and significantly reduce the number of iterations for convergence.
%Motivated by that result, we use the model
%in~\eqref{equ:exp_async_update}, where the total momentum, $\mu_T$, includes both asynchrony-induced and algorithmic  momentum, $\mu$, in~\eqref{eqn:momentum_gd}.
%\begin{equation}
%	\mathbb{E}[ x_{t+1} - x_t ] 
%	= \mu_T \mathbb{E}[x_t - x_{t-1}] - \alpha \mathbb{E}\nabla f(x_{t})
%\label{equ:exp_async_update}
%\end{equation}
%We will use this expression to design an estimator for the value of total momentum, $\hat{\mu_T}$.
%This estimator is a basic building block of \asynctuner, that {\em removes the need to manually compensate for the effects of asynchrony}.
%
%
%
%\paragraph{Measuring the momentum dynamics}
%\Asynctuner estimates total momentum $\mu_{T}$ on a running system and uses a negative feedback loop to adjust algorithmic momentum accordingly.
%Equation~\eqref{equ:exp_async_update} gives an estimate of $\hat{\mu_T}$ on a system with staleness $\tau$, based on \eqref{equ:exp_async_update}.
%\begin{align}
%\hat{\mu_T}
%					= \mathop{\mathsf{median}}\left(
%							\frac{x_{t - \tau} - x_{t - \tau-1} + \alpha \nabla_{S_{t-\tau -1}} f(x_{t - \tau - 1} )}
%							{x_{t - \tau-1} - x_{t - \tau-2}}
%					\right)
%\label{eqn:momentum_measurement}
%\end{align}
%We use $\tau$-stale model values to match the staleness of the gradient,  and perform all operations in an elementwise fashion. 
%This way we get a total momentum measurement from each variable; 
%the median combines them into a more robust estimate.
%
%\label{subsec:closed_loop_YF}
%\begin{figure}
%\centering
%\includegraphics[width=0.95\linewidth]{experiment_results/resnet/mom_dynamic_3_annotated.pdf}
%	\caption{
%	Momentum dynamics on CIFAR100 ResNet.
%	Running \tuner, total momentum is equal to algorithmic momentum in a synchronous setting (left). Total momentum is greater than algorithmic momentum on 16 asynchronous workers, due to asynchrony-induced momentum (middle).
%	Using the momentum feedback mechanism of \asynctuner, lowers algorithmic momentum and brings total momentum to match the target value on 16 asynchronous workers (right).
%	Red dots are individual total momentum estimates, $\hat{\mu}_T$, at each iteration. 
%The solid red line is a running average of those estimates.	
%	}
%	\label{fig:we-can-measure}
%\end{figure}
%
%\paragraph{Closing the asynchrony loop}
%Given a reliable measurement of $\mu_{T}$, 
%we can use it to adjust the value of algorithmic momentum so that the total momentum matches the \emph{target momentum} as decided by \tuner in Algorithm~\ref{alg:basic-algo}.
%\Asynctuner in Algorithm~\ref{alg:async-algo} (in Appendix~\ref{sec:async_app}) %(in Appendix~\ref{sec:async_yf}) 
%uses a simple negative feedback loop to achieve the adjustment.
%Figure~\ref{fig:we-can-measure} demonstrates that under asynchrony the measured total momentum is strictly higher than the algorithmic momentum (middle plot), as expected from theory;
%closing the feedback loop (right plot) leads to total momentum matching the target momentum.
%Closing the loop, as we will see, improves performance significantly.
%%Note for asynchronous-parallel training, as the estimates and parameter tuning is unstable in the beginning when there are only a small number of iterations, we use initial learning $\frac{1}{\tau + 1}$ instead of $1.0$ to prevent overflow in the beginning. 
%
%%\begin{algorithm}[H]
%%	\caption{\Asynctuner}
%%	\begin{algorithmic}[1]
%%%	\State Input: $\mu\gets0$, $\alpha \gets \frac{1}{\tau + 1}$, $\gamma\gets0.01, \tau$ (staleness)
%%	\State Input: $\mu\gets0$, $\alpha \gets 0.0001$, $\gamma\gets0.01, \tau$ (staleness)
%%	\For { $t\gets1$ to $T$}
%%	\State $x_t\!\gets\!x_{t - 1} + \mu (x_{t - 1} - x_{t - 2} ) - \alpha \nabla_{S_t} f(x_{t - \tau - 1} )$
%%	\State $\mu^*,\alpha \gets \Call{\tuner}{\nabla_{S_t} f(x_{t - \tau - 1} ), \beta}$ %(get momentum from the dynamic range)
%%	\State $\hat{\mu_T} 
%%					\gets \mathop{\mathsf{median}}\left(
%%							\frac{x_{t - \tau} - x_{t - \tau-1} + \alpha \nabla_{S_{t-\tau-1}} f(x_{t - \tau - 1} )}
%%							{x_{t - \tau-1} - x_{t - \tau-2}}
%%					\right)$ \Comment{Measuring total momentum}
%%	\State $\mu \leftarrow \mu + \gamma \cdot (\mu^* - \hat{\mu_T})$ \Comment{Closing the loop}
%%	\EndFor
%%\end{algorithmic}
%%\label{alg:async-algo}
%%\end{algorithm}
%



\vspace{-0.25em}
\section{Experiments}
\label{sec:experiments}
\vspace{-0.25em}
We empirically validate the importance of momentum tuning and evaluate \tuner in both synchronous (single-node) and asynchronous settings.
In synchronous settings, we first demonstrate that, with hand-tuning, momentum SGD is competitive with Adam, a state-of-the-art adaptive method.
Then, we evaluate \tuner \emph{without any hand tuning} in comparison to hand-tuned Adam and momentum SGD.
In asynchronous settings, we show that \asynctuner accelerates with momentum closed-loop control, significantly outperforming Adam.
%\emph{To eliminate influences of a specific random seed, in our synchronous and asynchronous experiments, the training loss and validation metrics are averaged from 3 runs using different random seeds.}

We evaluate on convolutional neural networks (CNN) and recurrent neural networks (RNN). For CNN, we train ResNet~\citep{he2016deep} for image recognition on CIFAR10 and CIFAR100~\citep{krizhevsky2014cifar}.
%, with regular and bottleneck building units respectively.
%We conduct experiments on both convolutional neural networks and recurrent neural networks. For convolutional neural networks, we evaluate on image recognition using a 110-layer ResNet~\citep{he2016deep} on CIFAR10~\citep{krizhevsky2014cifar} and a 164-layer ResNet on CIFAR100.
%For diversity of the models, we use regular and bottleneck building units respectively for CIFAR10 and CIFAR100.
For RNN, we train LSTMs for character-level language modeling with 
the TinyShakespeare (TS) dataset~\citep{karpathy2015visualizing}, word-level language modeling with the Penn TreeBank (PTB) ~\citep{marcus1993building}, and constituency parsing on the Wall Street Journal (WSJ) dataset~\citep{charniakparsing}.
We refer to Table~\ref{tab:model_specification} in Appendix~\ref{sec:model_spec} for model specifications. 
\emph{To eliminate influences of a specific random seed, in our synchronous and asynchronous experiments, the training loss and validation metrics are averaged from 3 runs using different random seeds.}
Across all the eight models and all experiments, we use sliding window width 20 for estimating the extreme curvature $h_max$ and $h_min$ in Algorithm~\ref{alg:curv_func}. It is selected based on the performance on PTB LSTM and CIFAR10 ResNet model. The selected sliding window width is directly applied to the other 6 models, including the convolutional sequence to sequence model in Section~\ref{sec:stability}, as well as the ResNext and Tied LSTM model in Appendix~\ref{sec:boost_exp}.

%We conduct experiments on both convolutional neural networks (CNN) and recurrent neural networks (RNN). For CNN, we evaluate on ResNet~\citep{he2016deep} for image recognition on CIFAR10~\citep{krizhevsky2014cifar} and CIFAR100, with regular and bottleneck building units respectively.
%%We conduct experiments on both convolutional neural networks and recurrent neural networks. For convolutional neural networks, we evaluate on image recognition using a 110-layer ResNet~\citep{he2016deep} on CIFAR10~\citep{krizhevsky2014cifar} and a 164-layer ResNet on CIFAR100.
%%For diversity of the models, we use regular and bottleneck building units respectively for CIFAR10 and CIFAR100.
%For RNN, we evaluate with LSTMs in 3 tasks: character-level language modeling with 
%the TinyShakespeare (TS) dataset~\citep{karpathy2015visualizing}, word-level language modeling with the Penn TreeBank (PTB) ~\citep{marcus1993building}, and constituency parsing on the Wall Street Journal (WSJ) dataset~\citep{charniakparsing}.
%We refer to Table~\ref{tab:model_specification} in Appendix~\ref{sec:model_spec} for model architecture details. 

%\vspace{-0.25em}
\subsection{Synchronous experiments}
\label{subsec:sync_exp}
\begin{figure*}[t]
\vspace{-0.75em}
\centering
	\begin{tabular}{c}
		\includegraphics[width=0.925\linewidth]{experiment_results/lstm_loss_all.pdf} \\[-0.5em]
		\includegraphics[width=0.925\linewidth]{experiment_results/lstm_test_all.pdf} \\[-0.5em]
	\end{tabular}
%	\vspace{-0.5em}
	\caption{
	Training loss and validation metrics on (left to right) word-level language modeling with PTB, char-level language modeling with TS and constituency parsing on WSJ. The valid. metrics are monotonic as we report the best values up to each number of iterations.}
%	\caption{
%	Training loss and validation metrics on word-level language modeling with PTB (left), char-level language modeling with TS (middle) and constituency parsing on WSJ (right). Note the validation metrics are monotonic as we report the best values up to each specific number of iterations.}
%	\vspace{-0.75em}
	\label{fig:loss_result_ptb}
%	\vspace{-0.25em}
\end{figure*}


%\begin{wrapfigure}[8]{r}{0.42\textwidth}
%\begin{minipage}{1.0\linewidth}
%\vspace{-0.3in}
%%\begin{table}[H]
%%\centering
%%\caption{
%%	Speedup of \tuner and tuned momentum SGD over tuned Adam.
%%	% The speedup is with respect to the number of iterations.
%%	%We compare our Algorithm~\ref{alg:basic-algo}) to the best configuration in tuning grid for Adam and momentum SGD. For tuned Adam and tuned momentum SGD, we take the lowest value of the smoothed training loss curve	 and report the speedup of our  to achieve the same loss. The speedup is demonstrated with respect to the number of iteraions. We run Resnet for 40k iterations, bottleneck resnet for 70k iterations and PTB LSTM for 30k Iterations.
%%	}
%%	\begin{tabular}[t]{c@{\hskip 0.6em}|c@{\hskip 0.6em}c@{\hskip 0.6em}c}
%%		\toprule
%%		 & Adam & mom.SGD & YF \\
%%		\midrule
%%		\midrule
%%		CIFAR10 & 1x & 1.71x & 1.93x  \\
%%		CIFAR100 & 1x & 1.83x & 1.35x \\
%%		PTB & 1x & 0.88x & 0.77x \\
%%		TS & 1x & 5.66x & 6.83x \\
%%		WSJ & 1x & 1.33x & 2.33x \\
%%		\bottomrule
%%	\end{tabular}
%%	\label{tab:iters_to_loss}
%%\end{table}
%% grid search version on Adam
%\begin{table}[H]
%\centering
%\caption{
%	Speedup of \tuner and tuned momentum SGD over tuned Adam.
%	% The speedup is with respect to the number of iterations.
%	%We compare our Algorithm~\ref{alg:basic-algo}) to the best configuration in tuning grid for Adam and momentum SGD. For tuned Adam and tuned momentum SGD, we take the lowest value of the smoothed training loss curve	 and report the speedup of our  to achieve the same loss. The speedup is demonstrated with respect to the number of iteraions. We run Resnet for 40k iterations, bottleneck resnet for 70k iterations and PTB LSTM for 30k Iterations.
%	}
%	\begin{tabular}[t]{c@{\hskip 0.6em}|c@{\hskip 0.6em}c@{\hskip 0.6em}c}
%		\toprule
%		 & Adam & mom.SGD & YF \\
%		\midrule
%		\midrule
%		CIFAR10 & 1x & 1.71x & 1.93x \\
%		CIFAR100 & 1x & 1.87x & 1.38x \\
%		PTB & 1x & 0.88x & 0.77x \\
%		TS & 1x & 2.49x & 3.28x \\
%		WSJ & 1x & 1.33x & 2.33x \\
%		\bottomrule
%	\end{tabular}
%	\label{tab:iters_to_loss}
%\end{table}
%\end{minipage}
%\end{wrapfigure}
We tune Adam and  momentum SGD on learning rate grids with prescribed momentum $0.9$ for SGD. We fix the parameters of Algorithm~\ref{alg:basic-algo} in all experiments, i.e.\ \tuner runs {\em without any hand tuning}.
We provide full specifications, including the learning rate (grid) and the number of iterations we train on each model in Appendix~\ref{sec:exp_spec}.
For visualization purposes, we smooth training losses with a uniform window of width $1000$. 
%\emph{We average the smoothed losses from 3 different random seeds}.
For Adam and momentum SGD on each model, we pick the configuration achieving the lowest averaged smoothed loss.
%in corresponding grids.
To compare two algorithms, we record the lowest smoothed loss achieved by both. Then the speedup is reported as the ratio of iterations to achieve this loss.
We use this setup to validate our claims.
%We tune momentum SGD and Adam on learning rate grids with prescribed momentum $0.9$ for SGD. We fix the parameters of Algorithm~\ref{alg:basic-algo} in all experiments, i.e.\ \tuner runs without any hand tuning. We provide full specifications, including the learning rate (grid) and the number of iterations we train on each model in Appendix~\ref{sec:exp_spec}.
%For visualization purposes, we smooth training losses with a uniform window of width $1000$. 
%%\emph{We average the smoothed losses from 3 different random seeds}.
%For Adam and momentum SGD on each model, we pick the configuration achieving the lowest averaged smoothed loss.
%%in corresponding grids.
%To compare two algorithms, we record the lowest smoothed loss achieved by both. Then the speedup is reported as the ratio of iterations to achieve this loss.
%We use this setup to validate our claims.
%

%%\begin{figure}[t]
%%\centering
%%	\begin{tabular}{c c}
%%		\includegraphics[width=0.45\linewidth]{experiment_results/resnet/resnet_loss.pdf} &
%%		\includegraphics[width=0.45\linewidth]{experiment_results/resnet/resnet_bottleneck_loss.pdf}
%%	\end{tabular}
%%	\caption{
%%	Training loss for ResNet on CIFAR10 (left) and CIFAR100 (right, still running). }
%%	\label{fig:loss_result_cifar}
%%\end{figure}
%
%\begin{wrapfigure}[11]{r}{0.525\textwidth}
%\begin{minipage}{1.0\linewidth}
%\vspace{-0.3in}
%\begin{table}[H]
%\centering
%\caption{
%	The speedup of \tuner and tuned mom. SGD comparing to tuned Adam.
%	% The speedup is with respect to the number of iterations.
%	%We compare our Algorithm~\ref{alg:basic-algo}) to the best configuration in tuning grid for Adam and momentum SGD. For tuned Adam and tuned momentum SGD, we take the lowest value of the smoothed training loss curve	 and report the speedup of our  to achieve the same loss. The speedup is demonstrated with respect to the number of iteraions. We run Resnet for 40k iterations, bottleneck resnet for 70k iterations and PTB LSTM for 30k Iterations.
%	}
%	\begin{tabular}[t]{c@{\hskip 0.6em}|c@{\hskip 0.6em}c@{\hskip 0.6em}c}
%		\toprule
%		 & Adam & mom. SGD & \tuner \\
%		\midrule
%		\midrule
%		CIFAR10 & 1x & 1.71x & 1.93x \\
%		CIFAR100 & 1x & 1.87x & 1.38x \\
%		PTB & 1x & 0.88x & 0.77x \\
%		TS & 1x & 2.49x & 3.28x \\
%		WSJ & 1x & 1.33x & 2.33x \\
%		\bottomrule
%	\end{tabular}
%	\label{tab:iters_to_loss}
%\end{table}
%\end{minipage}
%\end{wrapfigure}
%\begin{wrapfigure}[11]{r}{0.39\textwidth}
%\begin{minipage}{1.0\linewidth}
%\vspace{-0.15in}
%%\begin{table}[H]
%%\centering
%%\caption{
%%	Speedup of \tuner and tuned momentum SGD over tuned Adam.
%%	% The speedup is with respect to the number of iterations.
%%	%We compare our Algorithm~\ref{alg:basic-algo}) to the best configuration in tuning grid for Adam and momentum SGD. For tuned Adam and tuned momentum SGD, we take the lowest value of the smoothed training loss curve	 and report the speedup of our  to achieve the same loss. The speedup is demonstrated with respect to the number of iteraions. We run Resnet for 40k iterations, bottleneck resnet for 70k iterations and PTB LSTM for 30k Iterations.
%%	}
%%	\begin{tabular}[t]{c@{\hskip 0.6em}|c@{\hskip 0.6em}c@{\hskip 0.6em}c}
%%		\toprule
%%		 & Adam & mom.SGD & YF \\
%%		\midrule
%%		\midrule
%%		CIFAR10 & 1x & 1.71x & 1.93x  \\
%%		CIFAR100 & 1x & 1.83x & 1.35x \\
%%		PTB & 1x & 0.88x & 0.77x \\
%%		TS & 1x & 5.66x & 6.83x \\
%%		WSJ & 1x & 1.33x & 2.33x \\
%%		\bottomrule
%%	\end{tabular}
%%	\label{tab:iters_to_loss}
%%\end{table}
%% grid search version on Adam
%\begin{table}[H]
%\centering
%\caption{
%	Speedup of \tuner and tuned mom. SGD over tuned Adam.
%	% The speedup is with respect to the number of iterations.
%	%We compare our Algorithm~\ref{alg:basic-algo}) to the best configuration in tuning grid for Adam and momentum SGD. For tuned Adam and tuned momentum SGD, we take the lowest value of the smoothed training loss curve	 and report the speedup of our  to achieve the same loss. The speedup is demonstrated with respect to the number of iteraions. We run Resnet for 40k iterations, bottleneck resnet for 70k iterations and PTB LSTM for 30k Iterations.
%	}
%	\begin{tabular}[t]{@{\hskip 0.3em}c@{\hskip 0.3em}|c@{\hskip 0.3em}c@{\hskip 0.3em}c@{\hskip 0.3em}}
%		\toprule
%		 & Adam & mom.SGD & YF \\
%		\midrule
%		\midrule
%		CIFAR10 & $1\times$ & $1.71\times$ & $1.93\times$ \\
%		CIFAR100 & $1\times$ & $1.87\times$ & $1.38\times$ \\
%		PTB & $1\times$ & $0.88\times$ & $0.77\times$ \\
%		TS & $1\times$ & $2.49\times$ & $3.28\times$ \\
%		WSJ & $1\times$ & $1.33\times$ & $2.33\times$ \\
%		\bottomrule
%	\end{tabular}
%	\label{tab:iters_to_loss}
%\end{table}
%\end{minipage}
%\end{wrapfigure}

%\begin{wrapfigure}[10]{r}{0.375\textwidth}
%\begin{minipage}{1.0\linewidth}
%\vspace{-0.35in}
%\begin{table}[H]
%\centering
%\caption{
%	Speedup of \tuner and tuned momentum SGD over tuned Adam.
%	% The speedup is with respect to the number of iterations.
%	%We compare our Algorithm~\ref{alg:basic-algo}) to the best configuration in tuning grid for Adam and momentum SGD. For tuned Adam and tuned momentum SGD, we take the lowest value of the smoothed training loss curve	 and report the speedup of our  to achieve the same loss. The speedup is demonstrated with respect to the number of iteraions. We run Resnet for 40k iterations, bottleneck resnet for 70k iterations and PTB LSTM for 30k Iterations.
%	}
%	\begin{tabular}[t]{c@{\hskip 0.6em}|c@{\hskip 0.6em}c@{\hskip 0.6em}c}
%		\toprule
%		 & Adam & mom.SGD & YF \\
%		\midrule
%		\midrule
%		CIFAR10 & 1x & 1.71x & 1.93x  \\
%		CIFAR100 & 1x & 1.83x & 1.35x \\
%		PTB & 1x & 0.88x & 0.77x \\
%		TS & 1x & 5.66x & 6.83x \\
%		WSJ & 1x & 1.33x & 2.33x \\
%		\bottomrule
%	\end{tabular}
%	\label{tab:iters_to_loss}
%\end{table}
% grid search version on Adam
%%%%%%%%%%%%%%% old version table %%%%%%%%%%%%%%%%%%%
%\begin{table}[H]
%\centering
%\caption{
%	Speedup of \tuner and tuned mom. SGD over tuned Adam.
%	% The speedup is with respect to the number of iterations.
%	%We compare our Algorithm~\ref{alg:basic-algo}) to the best configuration in tuning grid for Adam and momentum SGD. For tuned Adam and tuned momentum SGD, we take the lowest value of the smoothed training loss curve	 and report the speedup of our  to achieve the same loss. The speedup is demonstrated with respect to the number of iteraions. We run Resnet for 40k iterations, bottleneck resnet for 70k iterations and PTB LSTM for 30k Iterations.
%	}
%%	\vspace{-0.075in}
%%	\begin{tabular}[t]{@{\hskip 0.15em}c@{\hskip 0.15em}|c@{\hskip 0.3em}c@{\hskip 0.3em}c@{\hskip 0.15em}}
%	\begin{tabular}{c | c c c}
%		 & Adam & mom.SGD & YF \\
%		\midrule
%		\midrule
%%		CIFAR10 & $1\times$ & $1.71\times$ & $1.93\times$ \\
%%		CIFAR100 & $1\times$ & $1.87\times$ & $1.38\times$ \\
%%		PTB & $1\times$ & $0.88\times$ & $0.77\times$ \\
%%		TS & $1\times$ & $2.49\times$ & $3.28\times$ \\
%%		WSJ & $1\times$ & $1.33\times$ & $2.33\times$ \\
%		CIFAR10 & 1x & 1.71x & 1.93x \\
%		CIFAR100 & 1x & 1.87x & 1.38x \\
%		PTB & 1x & 0.88x & 0.77x \\
%		TS & 1x & 2.49x & 3.28x \\
%		WSJ & 1x & 1.33x & 2.33x \\
%		\bottomrule
%	\end{tabular}
%	\label{tab:iters_to_loss}
%\end{table}
%%%%%%%%%%%%%%%%%%%%%%%%%%%%%%%%%%%%%%%%%%%%%%%%%%%%%
%%%%%%%%%%%%%%% new version table %%%%%%%%%%%%%%%%%%%
\vspace{-0.25em}
\begin{table}[h]
\centering
\small
%	\begin{tabular}[t]{@{\hskip 0.3em}c@{\hskip 0.5em}|c@{\hskip 0.45em}c@{\hskip 0.45em}c@{\hskip 0.75em}c@{\hskip 0.75em}c@{\hskip 0.15em}}
	\begin{tabular}[t]{@{\hskip 0.5em}c@{\hskip 0.5em}|c@{\hskip 1em}c@{\hskip 1em}c@{\hskip 1em}c@{\hskip 1em}c@{\hskip 0.5em}}
%	\begin{tabular}{c | c c c c c}
		\toprule
		 & CIFAR10 & CIFAR100 & PTB & TS & WSJ \\
		\midrule
		\midrule
		Adam & 1x & 1x & 1x & 1x & 1x \\
		mom. SGD & 1.71x & 1.87x & 0.88x & 2.49x & 1.33x \\
		YF & 1.93x & 1.38x & 0.77x & 3.28x & 2.33x \\
%		CIFAR10 & 1x & 1.71x & 1.93x \\
%		CIFAR100 & 1x & 1.87x & 1.38x \\
%		PTB & 1x & 0.88x & 0.77x \\
%		TS & 1x & 2.49x & 3.28x \\
%		WSJ & 1x & 1.33x & 2.33x \\
		\bottomrule
	\end{tabular}
	\caption{
	The speedup of \tuner and tuned momentum SGD over tuned Adam on ResNet and LSTM models.
	% The speedup is with respect to the number of iterations.
	%We compare our Algorithm~\ref{alg:basic-algo}) to the best configuration in tuning grid for Adam and momentum SGD. For tuned Adam and tuned momentum SGD, we take the lowest value of the smoothed training loss curve	 and report the speedup of our  to achieve the same loss. The speedup is demonstrated with respect to the number of iteraions. We run Resnet for 40k iterations, bottleneck resnet for 70k iterations and PTB LSTM for 30k Iterations.
	}
	\label{tab:iters_to_loss}
%	\vspace{-0.85em}
\end{table}
\vspace{-0.5em}

%%%%%%%%%%%%%%%%%%%%%%%%%%%%%%%%%%%%%%%%%%%%%%%%%%%%%
%\end{minipage}
%\end{wrapfigure}
\paragraph{Momentum SGD is competitive with adaptive methods}
In Table~\ref{tab:iters_to_loss}, we compare tuned momentum SGD and tuned Adam on ResNets with training losses shown in Figure~\ref{fig:loss_result_cifar} in Appendix~\ref{sec:add_exp}. We can observe that momentum SGD
achieves $1.71$x and $1.87$x speedup to tuned Adam on CIFAR10 and CIFAR100 respectively. In Figure~\ref{fig:loss_result_ptb} and Table~\ref{tab:iters_to_loss}, 
with the exception of PTB LSTM, momentum SGD also produces better training loss, as well as better validation perplexity in language modeling and validation F1 in parsing.
For the parsing task, we also compare with tuned Vanilla SGD and AdaGrad, which are used in the NLP community.
Figure~\ref{fig:loss_result_ptb} (right) shows that \emph{fixed momentum 0.9 can already speedup Vanilla SGD by $2.73$x, achieving observably better validation F1}. %,
%a performance matched by \tuner. 
 We refer to Appendix~\ref{sec:importance_momentum} for further discussion on the importance of momentum adaptivity in \tuner.
%These observations show that momentum is critical to acceleration, and momentum SGD can be better than the state-of-the-art adaptive method in a variety of models.% 

%\begin{figure*}[t]
%%\vspace{-2.25em}
%\centering
%	\begin{tabular}{c}
%		\includegraphics[width=0.925\linewidth]{experiment_results/lstm_loss_all.pdf} \\[-0.75em]
%		\includegraphics[width=0.925\linewidth]{experiment_results/lstm_test_all.pdf} \\[-0.5em]
%	\end{tabular}
%	\caption{
%	Training loss and test metrics on word-level language modeling with PTB (left), character-level language modeling with TS (middle) and constituency parsing on WSJ (right). Note the validation metrics are monotonic as we report the best values up to each specific number of iterations.}
%	\vspace{-0.75em}
%	\label{fig:loss_result_ptb}
%\end{figure*}


%\begin{wrapfigure}[10]{r}{0.375\textwidth}
%\begin{minipage}{1.0\linewidth}
%\vspace{-0.35in}
%%\begin{table}[H]
%%\centering
%%\caption{
%%	Speedup of \tuner and tuned momentum SGD over tuned Adam.
%%	% The speedup is with respect to the number of iterations.
%%	%We compare our Algorithm~\ref{alg:basic-algo}) to the best configuration in tuning grid for Adam and momentum SGD. For tuned Adam and tuned momentum SGD, we take the lowest value of the smoothed training loss curve	 and report the speedup of our  to achieve the same loss. The speedup is demonstrated with respect to the number of iteraions. We run Resnet for 40k iterations, bottleneck resnet for 70k iterations and PTB LSTM for 30k Iterations.
%%	}
%%	\begin{tabular}[t]{c@{\hskip 0.6em}|c@{\hskip 0.6em}c@{\hskip 0.6em}c}
%%		\toprule
%%		 & Adam & mom.SGD & YF \\
%%		\midrule
%%		\midrule
%%		CIFAR10 & 1x & 1.71x & 1.93x  \\
%%		CIFAR100 & 1x & 1.83x & 1.35x \\
%%		PTB & 1x & 0.88x & 0.77x \\
%%		TS & 1x & 5.66x & 6.83x \\
%%		WSJ & 1x & 1.33x & 2.33x \\
%%		\bottomrule
%%	\end{tabular}
%%	\label{tab:iters_to_loss}
%%\end{table}
%% grid search version on Adam
%\begin{table}[H]
%\centering
%\caption{
%	Speedup of \tuner and tuned mom. SGD over tuned Adam.
%	% The speedup is with respect to the number of iterations.
%	%We compare our Algorithm~\ref{alg:basic-algo}) to the best configuration in tuning grid for Adam and momentum SGD. For tuned Adam and tuned momentum SGD, we take the lowest value of the smoothed training loss curve	 and report the speedup of our  to achieve the same loss. The speedup is demonstrated with respect to the number of iteraions. We run Resnet for 40k iterations, bottleneck resnet for 70k iterations and PTB LSTM for 30k Iterations.
%	}
%	\vspace{-0.075in}
%	\begin{tabular}[t]{@{\hskip 0.15em}c@{\hskip 0.15em}|c@{\hskip 0.3em}c@{\hskip 0.3em}c@{\hskip 0.15em}}
%		\toprule
%		 & Adam & mom.SGD & YF \\
%		\midrule
%		\midrule
%%		CIFAR10 & $1\times$ & $1.71\times$ & $1.93\times$ \\
%%		CIFAR100 & $1\times$ & $1.87\times$ & $1.38\times$ \\
%%		PTB & $1\times$ & $0.88\times$ & $0.77\times$ \\
%%		TS & $1\times$ & $2.49\times$ & $3.28\times$ \\
%%		WSJ & $1\times$ & $1.33\times$ & $2.33\times$ \\
%		CIFAR10 & 1x & 1.71x & 1.93x \\
%		CIFAR100 & 1x & 1.87x & 1.38x \\
%		PTB & 1x & 0.88x & 0.77x \\
%		TS & 1x & 2.49x & 3.28x \\
%		WSJ & 1x & 1.33x & 2.33x \\
%		\bottomrule
%	\end{tabular}
%	\label{tab:iters_to_loss}
%\end{table}
%\end{minipage}
%\end{wrapfigure}
\paragraph{\tuner can match hand-tuned momentum SGD and can outperform hand-tuned Adam}% on ResNets and LSTMs}
In our experiments, 
\tuner, without any hand-tuning, yields training loss matching hand-tuned momentum SGD for all the ResNet and LSTM models in Figure~\ref{fig:loss_result_ptb} and~\ref{fig:loss_result_cifar}.  
When comparing to tuned Adam in Table~\ref{tab:iters_to_loss}, except being slightly slower on PTB LSTM, \tuner achieves $1.38$x to $3.28$x speedups in training losses on the other four models. \emph{More importantly, \tuner consistently shows better validation metrics than tuned Adam in Figure~\ref{fig:loss_result_ptb}}. It demonstrates that \tuner can match tuned momentum SGD and outperform tuned state-of-the-art adaptive optimizers. % in training deep neural networks.% in a large class of models.
In Appendix~\ref{sec:boost_exp}, we show \tuner further speeding up with finer-grain manual learning rate tuning.
%As careful optimizer tuning can further improve model performance in fixed training time, we refer to Appendix~\ref{sec:boost_exp} on auxiliary manual learning rate tuning for \tuner.

%%In our experiments, 
%%\tuner, without any hand-tuning, yields speedups from to 1.32x to 3.28x on ResNets and LSTMs in comparison to tuned momentum SGD.  
%When comparing to tuned Adam in Table~\ref{tab:iters_to_loss}, \tuner achieves speedups from 1.32x to 3.28x in training losses. \emph{More importantly, \tuner consistently shows better test metrics than tuned momentum SGD and Adam}. It demonstrates that \tuner can outperform hand-tuned state-of-the-art adaptive and non-adaptive optimizers.% in a large class of models.
\begin{figure*}
\centering	
\begin{tabular}{c c c}
	\includegraphics[width=0.31\linewidth]{experiment_results/tf_charrnn_train_loss_fix_mom_and_lr_rescaling_cmp.pdf} &
	\includegraphics[width=0.31\linewidth]{experiment_results/resnet/resnet_bottleneck_loss_fix_mom_and_lr_rescaling_cmp} &
	\includegraphics[width=0.31\linewidth]{experiment_results/tf_charrnn_train_loss_mom_vanilla_yf.pdf}
\end{tabular}
\caption{The importance of adaptive momentum: Training loss comparison between \tuner with adaptive momentum and \tuner with fixed momentum value; this comparison is conducted on TS LSTM (left) and CIFAR100 ResNet (right). Learning rate scaling based on \tuner tuned momentum can match the performance of full \tuner (right) on the TS LSTM. However without the \tuner tuned momentum, hand-tuned Vanilla SGD demonstrates observably larger training loss than momentum based methods, including full \tuner, \tuner learning rate rescaling and hand-tuned momentum SGD (with the same learning rate search grid as with Vanilla SGD.)}
\label{fig:cmp_fix_mom}
\end{figure*}

\paragraph{The importance of adaptive momentum in \tuner}
In Definition~\ref{def:GCN}, we noticed that the optimally tuned $\mu^*$ is highly objective-dependent. Empirically, We indeed observe a wide range of tuned momentum $\mu$ from YF; it ranges from smaller than 0.03 in the PTM LSTM to 0.89 for ResNext. To further validate the importance of momentum adaptivity in \tuner, we perform an ablation study to demonstrate the importance of objective-dependent momentum adaptivity in \tuner with CIFAR100 ResNet and TS LSTM. In the experiments, \tuner tunes the learning rate. Instead of also using the momentum tuned by YF, we continuously feed objective-agnostic prescribed momentum value $0.0$ and $0.9$ to the underlying momentum SGD optimizer which YF is tuning. In Figure~\ref{fig:cmp_fix_mom}, when comparing to \tuner with prescribed momentum 0.0 or 0.9, \tuner with adaptively tuned momentum achieves observably faster convergence on both TS LSTM and CIFAR100 ResNet. From a more practical perspective, in Figure~\ref{fig:loss_result_ptb} (bottom right) and Figure~\ref{fig:cmp_fix_mom} (right), we also observe that hand-tuned optimizer without momentum, i.e. Vanilla SGD, typically can not match the performance of momentum based methods, including \tuner and momentum SGD hand-tuned using the same learning rate grid as with Vanilla SGD. However in \tuner, we can rescale the learning rate based on the \tuner tuned momentum $\mu_t$, and use 0 momentum in the model updates to match the performance of momentum based methods. Specifically, we rescale the \tuner tuned learning rate $\alpha_t$ with $1/(1 - \mu_t)$ \footnote{Let $v_t = x_t - x_{t - 1}$ be the model update, this rescaling is motivated with the fact that $v_{t+1} = \mu_t v_{t} - \alpha_t \nabla f(x_t)$. Assuming the $v_t$ evolves smoothly, we have $v_t \approx \alpha_t/(1-\mu_t) \nabla f(x_t)$.}. Model updates with this rescaled learning rate and 0 momentum can demonstrate training loss closely matching those of \tuner and hand-tuned momentum SGD for WSJ LSTM in Figure~\ref{fig:loss_result_ptb} (bottom right) and TS LSTM in Figure~\ref{fig:cmp_fix_mom} (right).



%\begin{wrapfigure}[9]{r}{0.34\linewidth}
%\vspace{-0.6in}
%\begin{minipage}{1.0\linewidth}
%	\begin{figure}[H]
%		\includegraphics[width=0.975\linewidth]{experiment_results/ptb/adam_stale_15_tuning.pdf}
%			\vspace{-1.5em}
%		\caption{Hand-tuning Adam's momentum under asynchrony.}
%		\label{fig:adam_async_mom}
%	\end{figure}
%%	\vspace{-1.5em}
%%	\begin{figure}[H]
%%		\includegraphics[width=0.975\linewidth]{experiment_results/resnet/resnet_bottleneck_cmp_tuner_adam.pdf}
%%%		\caption{Adam, \tuner and \asynctuner on CIFAR100 with 16 async. workers. Sync. baseline uses \tuner.}
%%	\vspace{-1.5em}
%%		\caption{Asynchronous performance on CIFAR100 ResNet.}
%%		\label{fig:full_async_cmp}
%%	\end{figure}
%\end{minipage}	
%\end{wrapfigure}

%\begin{wrapfigure}[9]{R}{0.5\linewidth}
%\vspace{-0.575in}
%%\vspace{-0.75in}
%\begin{minipage}{\linewidth}
%%	\begin{figure}[H]
%%		\includegraphics[width=\linewidth]{experiment_results/ptb/adam_stale_15_tuning.pdf}
%%			\vspace{-1.5em}
%%		\caption{Hand-tuning Adam's momentum under asynchrony.}
%%		\label{fig:adam_async_mom}
%%	\end{figure}
%%	\vspace{-1.5em}
%	\begin{figure}[H]
%		\includegraphics[width=1.05\linewidth]{experiment_results/resnet/resnet_bottleneck_cmp_tuner_adam.pdf}
%%		\includegraphics[width=0.975\linewidth]{experiment_results/resnet/resnet_bottleneck_cmp_tuner_adam_default_adam.pdf}
%%		\caption{Adam, \tuner and \asynctuner on CIFAR100 with 16 async. workers. Sync. baseline uses \tuner.}
%	\vspace{-1.75em}
%		\caption{Asynchronous performance on CIFAR100 ResNet.}
%		\label{fig:full_async_cmp}
%	\end{figure}
%\end{minipage}	
%\end{wrapfigure}
%%\vspace*{5.0em}
\subsection{Asynchronous experiments}
\label{sec:async_exp}
%\vspace{-1em}
In this section, we evaluate \asynctuner with focus on the number of iterations to reach a certain solution. 
To that end, we run $16$ asynchronous workers on a single machine and force them to update the model in a round-robin fashion,
i.e. the gradient is delayed for $15$ iterations.
%We demonstrate
%%(1) Adam suffers a convergence speed penalty due to not tuning momentum in asynchronous settings;
%(1) \asynctuner (cf.\ Section~\ref{sec:async_tuner}) improves the convergence of \tuner dramatically, which leads to
%(3) \asynctuner having much faster convergence than Adam. 
%     
%\paragraph{State-of-the-art adaptive methods suffer from lack of momentum tuning} We conduct experiments on PTB LSTM with 16 asynchronous workers using Adam.
%Fixing the learning rate to the value achieving the lowest smoothed loss in Section~\ref{subsec:sync_exp}, we sweep the smoothing parameter $\beta_1$~\citep{kingma2014adam} of the first order moment estimate in grid $\{-0.2, 0.0, 0.3, 0.5, 0.7, 0.9\}$. $\beta_1$ serves the same role as momentum in SGD and we call it the momentum in Adam. Figure~\ref{fig:adam_async_mom} shows tuning momentum for Adam under asynchrony gives measurably better training loss. 
%This result emphasizes the importance of momentum tuning in asynchronous settings and suggests that state-of-the-art adaptive methods pay a penalty for using prescribed momentum.
%
%\begin{wrapfigure}[10]{r}{0.34\linewidth}
%\vspace{-0.35in}
%\begin{minipage}{1.0\linewidth}
%%	\begin{figure}[H]
%%		\includegraphics[width=\linewidth]{experiment_results/ptb/adam_stale_15_tuning.pdf}
%%			\vspace{-1.5em}
%%		\caption{Hand-tuning Adam's momentum under asynchrony.}
%%		\label{fig:adam_async_mom}
%%	\end{figure}
%%	\vspace{-1.5em}
%	\begin{figure}[H]
%		\includegraphics[width=0.975\linewidth]{experiment_results/resnet/resnet_bottleneck_cmp_tuner_adam.pdf}
%%		\caption{Adam, \tuner and \asynctuner on CIFAR100 with 16 async. workers. Sync. baseline uses \tuner.}
%	\vspace{-1.5em}
%		\caption{Asynchronous performance on CIFAR100 ResNet.}
%		\label{fig:full_async_cmp}
%	\end{figure}
%\end{minipage}	
%\end{wrapfigure}
%\paragraph{Closing the loop improves convergence under staleness}
%We compare the performance of \tuner in Algorithm~\ref{alg:basic-algo} and \asynctuner in Algorithm~\ref{alg:async-algo} on the 164-layer bottleneck ResNet. We conduct experiments using 16 asynchronous workers.
%In Figure~\ref{fig:adam_under_staleness} (right), 
%Figure~\ref{fig:full_async_cmp} 
Figure~\ref{fig:spotlight} (right) 
presents training losses on the CIFAR100 ResNet, using \tuner in Algorithm~\ref{alg:basic-algo}, \asynctuner in Algorithm~\ref{alg:async-algo} and Adam with the learning rate achieving the best smoothed loss in Section~\ref{subsec:sync_exp}.
We can observe closed-loop \tuner achieves $20.1$x speedup to \tuner, 
%Consequently, closed-loop \tuner achieves 5.09x speedup over Adam.
and consequently a $2.69$x speedup to Adam.
This demonstrates that (1) \asynctuner accelerates by reducing algorithmic momentum to compensate for asynchrony and (2) can converge in less iterations than Adam in asynchronous-parallel training. 
%We notice that \asynctuner achieves very similar loss as the synchronous baseline near the end.
%It suggests an almost $16$x wall-clock time speedup by parallelizing asynchronously.
%\vspace{-0.5em}
%\paragraph{Closed-loop \tuner outperforms Adam in asynchrony} 
%We run Adam with 16 workers on CIFAR100 ResNet using the learning rate achieving the lowest smoothed loss in Section~\ref{subsec:sync_exp}.
%Shown in Figure~\ref{fig:full_async_cmp}, Adam slows down dramatically due to asynchrony, while closed-loop \tuner, with feedback gain $\gamma=0.01$, demonstrates up to 2.69x speedup  over Adam.
%It shows \asynctuner can significantly outperform the state-of-the-art in asynchronous settings.

%\begin{figure}[t]
%\centering
%	\begin{tabular}{c c}
%		\includegraphics[width=0.45\linewidth]{experiment_results/ptb/adam_stale_15_tuning.pdf} &
%		\includegraphics[width=0.45\linewidth]{experiment_results/resnet/resnet_bottleneck_cmp_tuner_adam.pdf}
%	\end{tabular}
%	\caption{
%	Hand-tuning Adam's momentum under asynchrony (left) on PTB LSTM. Asynchronous performance comparison (right) on CIFAR100 ResNet.}
%	\label{fig:adam_under_staleness}
%\end{figure}



%% async experiment backup version
%\begin{wrapfigure}[9]{r}{0.34\linewidth}
%\vspace{-0.6in}
%\begin{minipage}{1.0\linewidth}
%	\begin{figure}[H]
%		\includegraphics[width=0.975\linewidth]{experiment_results/ptb/adam_stale_15_tuning.pdf}
%			\vspace{-1.5em}
%		\caption{Hand-tuning Adam's momentum under asynchrony.}
%		\label{fig:adam_async_mom}
%	\end{figure}
%%	\vspace{-1.5em}
%%	\begin{figure}[H]
%%		\includegraphics[width=0.975\linewidth]{experiment_results/resnet/resnet_bottleneck_cmp_tuner_adam.pdf}
%%%		\caption{Adam, \tuner and \asynctuner on CIFAR100 with 16 async. workers. Sync. baseline uses \tuner.}
%%	\vspace{-1.5em}
%%		\caption{Asynchronous performance on CIFAR100 ResNet.}
%%		\label{fig:full_async_cmp}
%%	\end{figure}
%\end{minipage}	
%\end{wrapfigure}
%\subsection{Asynchronous experiments}
%In this section, we evaulate \tuner in an asynchronous-parallel setting,
%where we focus on {\em statistical efficiency}: the number of iterations to reach a certain solution. 
%To that end, we run $M$ asynchronous workers on a single machine and force them to update the model in a round-robin fashion,
%i.e. the staled gradient is delayed for $(M-1)$ iterations.
%We demonstrate
%(1) Adam suffers a convergence speed penalty due to not tuning momentum in asynchronous settings;
%(2) \asynctuner (cf.\ Section~\ref{sec:async_tuner}) improves the convergence of \tuner dramatically, which leads to
%(3) \asynctuner having much faster convergence than Adam. 
%     
%
%\paragraph{State-of-the-art adaptive methods suffer from lack of momentum tuning} We conduct experiments on PTB LSTM with 16 asynchronous workers using Adam.
%Fixing the learning rate to the value achieving the lowest smoothed loss in Section~\ref{subsec:sync_exp}, we sweep the smoothing parameter $\beta_1$~\citep{kingma2014adam} of the first order moment estimate in grid $\{-0.2, 0.0, 0.3, 0.5, 0.7, 0.9\}$. $\beta_1$ serves the same role as momentum in SGD and we call it the momentum in Adam. Figure~\ref{fig:adam_async_mom} shows tuning momentum for Adam under asynchrony gives measurably better training loss. 
%This result emphasizes the importance of momentum tuning in asynchronous settings and suggests that state-of-the-art adaptive methods pay a penalty for using prescribed momentum.
%
%\begin{wrapfigure}[10]{r}{0.34\linewidth}
%\vspace{-0.35in}
%\begin{minipage}{1.0\linewidth}
%%	\begin{figure}[H]
%%		\includegraphics[width=\linewidth]{experiment_results/ptb/adam_stale_15_tuning.pdf}
%%			\vspace{-1.5em}
%%		\caption{Hand-tuning Adam's momentum under asynchrony.}
%%		\label{fig:adam_async_mom}
%%	\end{figure}
%%	\vspace{-1.5em}
%	\begin{figure}[H]
%		\includegraphics[width=0.975\linewidth]{experiment_results/resnet/resnet_bottleneck_cmp_tuner_adam.pdf}
%%		\caption{Adam, \tuner and \asynctuner on CIFAR100 with 16 async. workers. Sync. baseline uses \tuner.}
%	\vspace{-1.5em}
%		\caption{Asynchronous performance on CIFAR100 ResNet.}
%		\label{fig:full_async_cmp}
%	\end{figure}
%\end{minipage}	
%\end{wrapfigure}
%\paragraph{Closing the loop improves convergence under staleness}
%We compare the performance of \tuner in Algorithm~\ref{alg:basic-algo} and \asynctuner in Algorithm~\ref{alg:async-algo} on the 164-layer bottleneck ResNet. We conduct experiments using 16 asynchronous workers.
%%In Figure~\ref{fig:adam_under_staleness} (right), 
%In Figure~\ref{fig:full_async_cmp}, 
%we observe \tuner without closed-loop momentum control decrease slowly before 90k iterations. Consequently, the closed-loop \tuner achieves more than 20.1x speedup to reach the lowest loss from \tuner after 120k iterations.
%This demonstrates that \asynctuner accelerates by effectively reducing algorithmic momentum to compensate for asynchrony. 
%%We notice that \asynctuner achieves very similar loss as the synchronous baseline near the end.
%%It suggests an almost $16$x wall-clock time speedup by parallelizing asynchronously.
%\vspace{-0.5em}
%\paragraph{Closed-loop \tuner outperforms Adam in asynchrony} We run Adam with 16 workers on CIFAR100 ResNet using the learning rate achieving the lowest smoothed loss in Section~\ref{subsec:sync_exp}.
%Shown in Figure~\ref{fig:full_async_cmp}, Adam slows down dramatically due to asynchrony, while closed-loop \tuner, with feedback gain $\gamma=0.01$, demonstrates up to 2.69x speedup  over Adam.
%%It shows \asynctuner can significantly outperform the state-of-the-art in asynchronous settings.
%
%%\begin{figure}[t]
%%\centering
%%	\begin{tabular}{c c}
%%		\includegraphics[width=0.45\linewidth]{experiment_results/ptb/adam_stale_15_tuning.pdf} &
%%		\includegraphics[width=0.45\linewidth]{experiment_results/resnet/resnet_bottleneck_cmp_tuner_adam.pdf}
%%	\end{tabular}
%%	\caption{
%%	Hand-tuning Adam's momentum under asynchrony (left) on PTB LSTM. Asynchronous performance comparison (right) on CIFAR100 ResNet.}
%%	\label{fig:adam_under_staleness}
%%\end{figure}




 







 




\vspace{-0.5em}
\section{Related work}
\label{sec:related}
\vspace{-0.45em}
%Many techniques have been proposed on tuning hyperparameters for optimizers.~\citet{bergstra2012random} investigate random search for general tuning of  hyperparameters. 
%Bayesian approaches~\citep{snoek2012practical} model evaluation metrics as samples from a Gaussian process guiding optimal hyperparameter search. 
%Another trend is the adaptive methods which require less manual tuning than SGD:
%Adagrad~\citep{duchi2011adaptive} is one of the first method with per-dimension learning rate, followed by RMSProp~\citep{tieleman2012lecture} and Adam~\citep{chilimbi2014project} using different learning rate rules. 
%\citet{schaul2013no} use a noisy quadratic model similar to ours to extract learning rate tuning rule in Vanilla SGD.
%However their approach does not use momentum which is essential in training modern neural networks. Existing adaptive momentum approach either only consider the non-stochastic setting~\citep{graepel2002stable,rehman2011effect,hameed2016back,swanston1994simple,ampazis2000levenberg,qiu1992accelerated} or only analyze stochasticity with $O(1/t)$ learning rate. In the contrast, we aim at practical momentum adaptivity for stochastically  training of modern neural networks, as well as simultaneously presenting learning rate adaptivity.  
Many techniques have been proposed on tuning hyperparameters for optimizers. General hyperparameter tuning approaches, such as random search~\citep{bergstra2012random} and Bayesian approaches~\citep{snoek2012practical, hutter2011sequential}, can directly tune optimizers.  
As another trend, adaptive methods, including AdaGrad~\citep{duchi2011adaptive}, RMSProp~\citep{tieleman2012lecture} and Adam~\citep{kingma2014adam}, uses per-dimension learning rate. 
%They typically require less manual tuning than SGD. 
\citet{schaul2013no} use a noisy quadratic model similar to ours to tune the learning rate in Vanilla SGD.
However they do not use momentum which is essential in training modern neural nets. Existing adaptive momentum approach either consider the deterministic setting~\citep{graepel2002stable,rehman2011effect,hameed2016back,swanston1994simple,ampazis2000levenberg,qiu1992accelerated} or only analyze stochasticity with $O(1/t)$ learning rate~\citep{leen1994optimal}. In contrast, we aim at practical momentum adaptivity for stochastically training neural nets.  

\vspace{-0.5em}
\section{Discussion}
\label{sec:discussion}
\vspace{-0.45em}
We presented \tuner, the first optimization method that automatically tunes momentum as well as the learning rate of momentum SGD. 
\tuner outperforms the state-of-the-art adaptive optimizers on a large class of models both in synchronous and asynchronous settings.
It estimates statistics purely from the gradients of a running system,
and then tunes the hyperparameters of momentum SGD based on noisy, local quadratic approximations.
As future work, we believe that more accurate curvature estimation methods,
like the $bbprop$ method~\citep{martens2012estimating} can further improve \tuner.
We also believe that our closed-loop momentum control mechanism in Section~\ref{sec:async_tuner} 
could accelerate other adaptive methods in asynchronous-parallel settings.


% In the unusual situation where you want a paper to appear in the
% references without citing it in the main text, use \nocite
\nocite{langley00}

\bibliography{arxiv,iclr2018_conference}
\bibliographystyle{icml2018}


\appendix 
\newpage
\onecolumn
\section{Proof of Lemma~\ref{lem:robustness}}
\label{sec:proof_robustness}
To prove Lemma~\ref{lem:robustness}, we first prove a more generalized version in Lemma~\ref{lem:robustness_general}. By restricting $f$ to be a one dimensional quadratics function, the generalized curvature $h_t$ itself is the only eigenvalue. We can prove Lemma~\ref{lem:robustness} as a straight-forward corollary. Lemma~\ref{lem:robustness_general} also implies, in the multiple dimensional correspondence of~\eqref{equ:one_dim_22_rec}, the spectral radius $\rho(\mat{A}_t)=\sqrt{\mu}$ if the curvature on all eigenvector directions (eigenvalue) satisfies~\eqref{eqn:robust_region}.

\begin{lemma}
\label{lem:robustness_general}
Let the gradients of a function $f$ be described by
\begin{equation}
	\nabla f(\mat{x}_t) = \mat{H}(\mat{x}_t) (\mat{x}_t - \mat{x}^*),
\end{equation}
with $\mat{H}(\bm{x}_t) \in \mathbb{R}^n \mapsto \mathbb{R}^{n\times n}$.
Then the momentum update can be expressed as a linear operator:
\begin{align}
{\begin{pmatrix}
\mat{y}_{t+1}\\
\mat{y}_t \\
\end{pmatrix}}
=
{\begin{pmatrix}
\mat{I}-\alpha \mat{H}(\mat{x}_t) + \mu \mat{I} & - \mu \mat{I} \\
\mat{I} & \mat{0} \\
\end{pmatrix}}
{\begin{pmatrix}
\mat{y}_t \\
\mat{y}_{t-1} \\
\end{pmatrix}}
=\mat{A}_t
{\begin{pmatrix}
\mat{y}_t \\
\mat{y}_{t-1} \\
\end{pmatrix}},
\end{align}
where $\mat{y}_t\triangleq \mat{x}_t - \mat{x}^*$.
Now, assume that the following condition holds for all eigenvalues $\lambda(\mat{H}(\bm{x}_t))$ of $\mat{H}(\bm{x}_t)$:
\begin{align}
{(1-\sqrt{\mu})^2\over \alpha} &\leq \lambda(\mat{H}(\bm{x}_t)) \leq {(1+\sqrt{\mu})^2\over \alpha}.
\label{equ:control_condition}
\end{align}
then the spectral radius of $\mat{A}_t$ is controlled by momentum with
$	\rho(\mat{A}_t) = \sqrt{\mu}.$

\begin{proof}
Let $\lambda_t$ be an eigenvalue of matrix $\mat{A}_t$, it gives 
$\det\left(\mat{A}_t - \lambda_t \mat{I} \right) = 0$. 
We define the blocks in $\mat{A}_t$ as $\mat{C} = \mat{I} - \alpha \mat{H}_t + \mu \mat{I} - \lambda_t \mat{I}$, $\mat{D} = -\mu \mat{I}$,
$\mat{E} = \mat{I}$ and $\mat{F} = -\lambda_t \mat{I}$ which gives
\[
\det \left( \mat{A}_t - \lambda_t \mat{I}\right) = \det{\mat{F}} \det{\left(\mat{C} - \mat{D} \mat{F}^{-1}
\mat{E} \right)} = 0
\]
assuming generally $\mat{F}$ is invertible. Note we use $\mat{H}_t\triangleq\mat{H}(\mat{x}_t)$ for simplicity in writing. The equation $\det{\left(\mat{C} - \mat{D} \mat{F}^{-1}
\mat{E} \right)} = 0$ implies that
\begin{equation}
\det \left( \lambda_t^2\mat{I} - \lambda_t \mat{M}_t + \mu \mat{I} \right) = 0
\label{equ:control_condition_2}
\end{equation}
with $\mat{M}_t = \left( \mat{I} - \alpha \mat{H}_t + \mu \mat{I} \right)$. In other words, $\lambda_t$ satisfied that $\lambda_t^2 - \lambda_t \lambda(\mat{M}_t) + \mu = 0$ with $\lambda(\mat{M}_t)$ being one eigenvalue of $\mat{M_t}$. I.e.
\begin{equation}
	\lambda_t = \frac{\lambda(\mat{M}_t) \pm \sqrt{\lambda(\mat{M}_t)^2 - 4\mu}}{2}
\end{equation}

On the other hand,~\eqref{equ:control_condition} guarantees that $(1 - \alpha \lambda(\mat{H}_t) + \mu)^2 \leq 4\mu$. We know both $\mat{H}_t$ and $\mat{I} - \alpha \mat{H}_t + \mu \mat{I}$ are symmetric. Thus for all eigenvalues $\lambda(\mat{M}_t)$ of $\mat{M}_t$, we have $\lambda(\mat{M}_t)^2 = (1 - \alpha \lambda(\mat{H}_t) + \mu)^2 \leq 4\mu$ which guarantees $| \lambda_t | = \sqrt{\mu}$ for all $\lambda_t$. As the spectral radius is equal to the magnitude of the largest eigenvalue of $\mat{A}_t$, we have the spectral radius of $\mat{A}_t$ being $\sqrt{\mu}$.


\end{proof}
	
\end{lemma}


\section{Proof of Lemma~\ref{lem:main_lemma}}
We first prove Lemma~\ref{lem:bias_rec} and Lemma~\ref{lem:var_rec} as preparation for the proof of Lemma~\ref{lem:main_lemma}. After the proof for one dimensional case, we discuss the trivial generalization to multiple dimensional case.
\begin{lemma}
\label{lem:bias_rec}
	Let the $h$ be the curvature of a one dimensional quadratic function $f$ and $\overline{x}_t = \E x_t$. We assume, without loss of generality, the optimum point of $f$ is $x^{\star}=0$. Then we have the following recurrence
	\begin{equation} 
		\begin{pmatrix}
			\overline{x}_{t + 1} \\
			\overline{x}_t
		\end{pmatrix} = 
		\begin{pmatrix}
			1-\alpha h + \mu & - \mu\\
			1 & 0 \\
		\end{pmatrix}^{t}
		\begin{pmatrix}
			x_1 \\
			x_0
		\end{pmatrix}
		\label{equ:bias_rec}
	\end{equation} 
	\begin{proof}
		From the recurrence of momentum SGD, % in~\eqref{equ:momentum_sgd}, 
		we have
		\begin{equation*}
			\begin{aligned}
				\E x_{t + 1} = & \E [ x_{t} - \alpha \nabla f_{S_t} (x_t) + \mu (x_t - x_{t - 1} ) ]\\
							= & \E_{x_{t}}	[ x_{t} - \alpha \E_{S_t} \nabla f_{S_t} (x_t) + \mu (x_t - x_{t - 1} ) ] \\
							= & \E_{x_{t}}	[ x_{t} - \alpha h x_t + \mu (x_t - x_{t - 1} ) ] \\
							= & (1 - \alpha h + \mu)\overline{x}_t - \mu\overline{x}_{t - 1}
			\end{aligned}
		\end{equation*}
		By putting the equation in to matrix form,~\eqref{equ:bias_rec} is a straight-forward result from unrolling the recurrence for $t$ times. Note as we set $x_1 = x_0$ with no uncertainty in momentum SGD, we have $[\overline{x}_0, \overline{x}_1] = [x_0, x_1]$.
	\end{proof}
\end{lemma}

\begin{lemma}
\label{lem:var_rec}
	Let $U_t=\E ( x_t - \overline{x}_t ) ^2$ and $V_t= \E (x_t - \overline{x}_t)(x_{t-1} - \overline{x}_{t-1})$ with $\overline{x}_t$ being the expectation of $x_t$. For quadratic function $f(x)$ with curvature $h \in \mathbb{R}$, We have the following recurrence
		\begin{equation} 
		\begin{pmatrix}
			U_{t+1} \\
			U_t \\
			V_{t + 1}
		\end{pmatrix} = 
		(\mat{I} - \mat{B}^{\top})(\mat{I} - \mat{B})^{-1}
		\begin{pmatrix}
			\alpha^2 C \\
			0 \\
			0
		\end{pmatrix}
	\end{equation}
	where 
	\begin{equation}
		\mat{B} = 
		\begin{pmatrix}
		(1-\alpha h + \mu)^2 &  \mu^2 & -2\mu(1-\alpha h + \mu)\\
		1 & 0 & 0 \\
		1-\alpha h + \mu & 0 & - \mu
		\end{pmatrix}
	\end{equation}
	and $C = \E ( \nabla f_{S_t}(x_t) - \nabla f(x_t) )^2$ is the variance of gradient on minibatch $S_t$.
	
	\begin{proof}
		We prove by first deriving the recurrence for $U_t$ and $V_t$ respectively and combining them in to a matrix form. For $U_t$, we have
		\begin{equation}
		\begin{aligned}
			U_{t + 1} = & \E ( x_{t+1} - \overline{x}_{t + 1} )^2\\
			 = & \E ( x_{t} - \alpha \nabla f_{S_t}(x_t) + \mu (x_{t} - x_{t - 1} ) - (1 - \alpha h + \mu) \overline{x}_t + \mu \overline{x}_{t - 1} )^2 \\
			 = & \E ( x_{t} - \alpha \nabla f(x_t) + \mu (x_{t} - x_{t - 1} ) - (1 - \alpha h + \mu) \overline{x}_t + \mu \overline{x}_{t - 1}  + \alpha (\nabla f(x_t) - \nabla f_{S_t}(x_t)) )^2 \\
			 = & \E ( (1 - \alpha h + \mu) (x_t - \overline{x}_t)  - \mu(x_{t - 1} - \overline{x}_{t - 1} ) )^2 + \alpha^2 \E ( \nabla f(x_t) - \nabla f_{S_t}(x_t) )^2 \\
			 = & (1 - \alpha h + \mu)^2 \E ( x_t - \overline{x}_t )^2 -2 \mu (1 - \alpha h + \mu) \E (x_t - \overline{x}_t)(x_{t - 1} - \overline{x}_{t - 1} ) \\
			 & + \mu^2\E ( x_{t-1} - \overline{x}_{t-1} )^2+ \alpha^2 C
		\end{aligned}
		\label{equ:U_term}
		\end{equation}
		where the cross terms cancels due to the fact $\E_{S_t} [\nabla f(x_t) - \nabla f_{S_t}(x_t)]=0$ in the third equality. 
		
		For $V_t$, we can similarly derive
		\begin{equation}
		\begin{aligned}			
			V_t = & \E (x_t - \overline{x}_t) (x_{t-1} - \overline{x}_{t-1} ) \\
			= & \E ( (1 - \alpha h + \mu) (x_{t-1} - \overline{x}_{t-1} ) - \mu (x_{t-2} - \overline{x}_{t-2}) + \alpha (\nabla f(x_t) - \nabla f_{S_t}(x_t) ) ) (x_{t-1} - \overline{x}_{t-1} ) \\
			= & (1 - \alpha h + \mu)\E ( x_{t-1} - \overline{x}_{t-1} )^2 - \mu \E (x_{t-1} - \overline{x}_{t-1})(x_{t-2} - \overline{x}_{t-2})
		\end{aligned}
		\label{equ:V_term}
		\end{equation}
		Again, the term involving $\nabla f(x_t) - \nabla f_{S_t}(x_t)$ cancels in the third equality as a results of $\E_{S_t} [\nabla f(x_t) - \nabla f_{S_t}(x_t)]=0$.~\eqref{equ:U_term} and~\eqref{equ:V_term} can be jointly expressed in the following matrix form
		\begin{equation}
		\begin{aligned}
			\begin{pmatrix}
			U_{t+1} \\
			U_t \\
			V_{t + 1}
		\end{pmatrix}= \mat{B} 
		\begin{pmatrix}
			U_t \\
			U_{t-1} \\
			V_t
		\end{pmatrix} + 
		\begin{pmatrix}
			\alpha^2 C \\
			0 \\
			0
		\end{pmatrix}
		=\sum\limits_{i = 0}^{t-1} \mat{B}^{i} \begin{pmatrix}
			\alpha^2 C \\
			0 \\
			0
		\end{pmatrix} + \mat{B}^t \begin{pmatrix}
			U_1 \\
			U_0 \\
			V_1
		\end{pmatrix}
		= (\mat{I} - \mat{B}^t)(\mat{I} - \mat{B})^{-1}
		\begin{pmatrix}
			\alpha^2 C \\
			0 \\
			0
		\end{pmatrix}.
		\end{aligned}
		\end{equation}
		Note the second term in the second equality is zero because $x_0$ and $x_1$ are deterministic. Thus $U_1\!=\!U_0\!=\!V_1\!=\!0$.
	\end{proof}
\end{lemma}

According to Lemma~\ref{lem:bias_rec} and~\ref{lem:var_rec}, we have $\E ( \overline{x}_t - x^{*} )^2 = (\mat{e}^{\top}_1 \mat{A}^t [x_1, x_0]^{\top})^2$ and $\E ( x_t - \overline{x}_t )^2=\alpha^2 C \mat{e}^{\top}_1 (\mat{I} - \mat{B}^t)(\mat{I} - \mat{B})^{-1}\mat{e}_1$ where $\mat{e}_1 \in \mathbb{R}^n$ has all zero entries but the first dimension. Combining these two terms, we prove Lemma~\ref{lem:main_lemma}. Though the proof here is for one dimensional quadratics, it trivially generalizes to multiple dimensional quadratics. Specifically, we can decompose the quadratics along the eigenvector directions, and then apply Lemma~\ref{lem:main_lemma} to each eigenvector direction using the corresponding curvature $h$ (eigenvalue). By summing quantities in~\eqref{equ:squared_dist_exact} for all eigenvector directions, we can achieve the multiple dimensional correspondence of~\eqref{equ:squared_dist_exact}.




\section{Proof of Lemma~\ref{lem:spectral_var_control}}
Again we first present a proof of a multiple dimensional generalized version of Lemma~\ref{lem:spectral_var_control}. The proof of Lemma~\ref{lem:spectral_var_control} is a one dimensional special case of Lemma~\ref{lem:spectral_var_control_multi}. Lemma~\ref{lem:spectral_var_control_multi} also implies that for multiple dimension quadratics, the corresponding spectral radius $\rho(\mat{B})=\mu$ if ${(1-\sqrt{\mu})^2\over \alpha} \leq h \leq {(1+\sqrt{\mu})^2\over \alpha}$ on all the eigenvector directions with $h$ being the eigenvalue (curvature).
\begin{lemma}
\label{lem:spectral_var_control_multi}
Let $\mat{H}\in\mathbb{R}^{n\times n}$ be a symmetric matrix and $\rho(\mat{B})$ be the spectral radius of matrix 
\begin{equation}
	%\rho_V(\alpha, \mu) = \rho \left(
\mat{B} = {\begin{pmatrix}
(\mat{I}-\alpha \mat{H} + \mu \mat{I})^{\top}(\mat{I}-\alpha \mat{H} + \mu \mat{I}) &  \mu^2 \mat{I} & -2\mu(\mat{I}-\alpha \mat{H} + \mu \mat{I})\\
\mat{I} & \mat{0} & \mat{0} \\
\mat{I}-\alpha \mat{H} + \mu \mat{I} & \mat{0} & - \mu \mat{I} 
\end{pmatrix}}
	%\right)
\end{equation}
We have $\rho(\mat{B})=\mu$ if all eigenvalues $\lambda(\mat{H})$ of $\mat{H}$ satisfies
\begin{equation}
{(1-\sqrt{\mu})^2\over \alpha} \leq \lambda(\mat{H}) \leq {(1+\sqrt{\mu})^2\over \alpha}.
\label{equ:control_condition_var}
\end{equation}

\begin{proof}
	Let $\lambda$ be an eigenvalue of matrix $\mat{B}$, it gives 
$\det\left(\mat{B} - \lambda \mat{I} \right) = 0$ which can be alternatively expressed as
\begin{equation}	
\det \left( \mat{B} - \lambda \mat{I}\right) = \det{\mat{F}} \det{\left(\mat{C} - \mat{D} \mat{F}^{-1}
\mat{E} \right)} = 0
\label{equ:control_condition_var_1}
\end{equation}
assuming $\mat{F}$ is invertible, i.e. $\lambda + \mu \neq 0$, where the blocks in $\mat{B}$ 
\begin{equation*}
		\mat{C} = \left( { \begin{array}{c c}
 			\mat{M}^{\top}\mat{M} - \lambda \mat{I} &  \mu^2 \mat{I} \\
 			\mat{I} & - \lambda \mat{I}
 		\end{array} } \right), 
 		\mat{D} = \left( { \begin{array}{c}
 			-2\mu \mat{M} \\
 			\mat{0}
 		\end{array}}\right),
 		\mat{E} = \left( {\begin{array}{c}
 			\mat{M} \\
 			\mat{0}
 		\end{array}} \right)^{\top},
 		\mat{F} = -\mu \mat{I} - \lambda \mat{I}
	\end{equation*}
	with $\mat{M}=\mat{I}-\alpha \mat{H} + \mu \mat{I}$.~\eqref{equ:control_condition_var_1} can be transformed using straight-forward algebra as
	\begin{equation}
		\det \left( \begin{array}{c c}
 			(\lambda - \mu) \mat{M}^{\top}\mat{M} - (\lambda + \mu) \lambda \mat{I} & (\lambda + \mu)\mu^2 \mat{I} \\
 			(\lambda + \mu) \mat{I} & -(\lambda + \mu)\lambda \mat{I}
 		\end{array} \right) = 0
		\label{equ:control_condition_var_2}	
	\end{equation}
	Using similar simplification technique as in~\eqref{equ:control_condition_var_1}, we can further simplify into
	\begin{equation}
		(\lambda - \mu)\det \left( (\lambda + \mu)^2 \mat{I} - \lambda \mat{M}^{\top}\mat{M} \right) = 0
	\end{equation}
	if $\lambda \neq \mu$, as $(\lambda + \mu)^2 \mat{I} - \lambda \mat{M}^{\top}\mat{M}$ is diagonalizable, we have $(\lambda + \mu)^2 - \lambda \lambda(\mat{M})^2 = 0$ with $\lambda(\mat{M})$ being an eigenvalue of symmetric $\mat{M}$. The analytic solution to the equation can be explicitly expressed as
	\begin{equation}
		\lambda = \frac{\lambda(\mat{M})^2 - 2\mu \pm \sqrt{(\lambda(\mat{M})^2 - 2\mu)^2 - 4\mu^2}}{2}.
		\label{equ:control_condition_var_3}	
	\end{equation}
	
	When the condition in~\eqref{equ:control_condition_var} holds, we have $\lambda(M)^2=(1 - \alpha \lambda(\mat{H}) + \mu)^2 \leq 4\mu$. One can verify that 
	
	\begin{equation}
		\begin{aligned}
			(\lambda(\mat{M})^2 - 2\mu)^2 - 4\mu^2 & = && (\lambda(\mat{M})^2 - 4\mu)\lambda(\mat{M})^2 \\
			& = &&\left( (1 - \alpha \rho(\mat{H} ) + \mu)^2 - 4\mu\right)\lambda(\mat{M})^2 \\
			& \leq && 0
		\end{aligned}
	\end{equation}
	Thus the roots in~\eqref{equ:control_condition_var_3} are conjugate with $| \lambda | = \mu$. In conclusion, the condition in~\eqref{equ:control_condition_var} can guarantee all the eigenvalues of $\mat{B}$ has magnitude $\mu$. Thus the spectral radius of $\mat{B}$ is controlled by $\mu$.
\end{proof}

\end{lemma}



%\input{multi_dim}
%\input{dist_small_lstm}
\section{Analytical solution to~\eqref{equ:noisy_min}}
\label{sec:opt}
The problem in~\eqref{equ:noisy_min} does not need iterative solver but has an analytical solution. Substituting only the second constraint, the objective becomes $p(x)=x^2D^2 + (1-x)^4/h_{\min}^2C$ with $x=\sqrt{\mu} \in [0, 1)$. By setting the gradient of $p(x)$ to 0, we can get a cubic equation whose root $x=\sqrt{\mu_p}$ can be computed in closed form using Vieta's substitution. As $p(x)$ is uni-modal in $[0, 1)$, the optimizer for \eqref{equ:noisy_min} is exactly the maximum of $\mu_p$ and $(\sqrt{h_{\max}/h_{\min} }-1 )^2 / (\sqrt{h_{\max}/h_{\min}}+1)^2$, the right hand-side of the first constraint in~\eqref{equ:noisy_min}.

%\begin{figure*}
%\vspace{-1em}
\centering
%\includegraphics[width=0.99\linewidth, trim={10cm 0 0 0},clip]{../yellowfin_iclr2018/manuscript_for_revision/experiment_results/resnet/mom_dynamic_3_annotated.pdf}
\includegraphics[width=0.99\linewidth]{../yellowfin_iclr2018/manuscript_for_revision/experiment_results/resnet/mom_dynamic_3_annotated.pdf}
	\vspace{-0.5em}
	\caption{
	When running \tuner, total momentum $\hat{\mu}_t$ equals algorithmic value in synchronous settings (left); $\hat{\mu}_t$ is greater than algorithmic value on 16 asynchronous workers (middle).
	\Asynctuner automatically lowers algorithmic momentum and brings total momentum to match the target value (right).
%Red dots are measured $\hat{\mu}_t$ at every step with red line as its running average.
	Red dots are total momentum estimates, $\hat{\mu}_T$, at each iteration. 
The solid red line is a running average of $\hat{\mu}_T$.
%	When running \tuner, total momentum $\hat{\mu}_t$ is greater than algorithmic value on 16 asynchronous workers (left).
%	\Asynctuner automatically lowers algorithmic momentum and matches total momentum to the target value (right).
%%Red dots are measured $\hat{\mu}_t$ at every step with red line as its running average.
%	Red dots are total momentum estimates, $\hat{\mu}_T$, at each iteration. 
%    The solid red line is a running average of $\hat{\mu}_T$.	
	}
	\label{fig:we-can-measure}
%\vspace{-0.35em}
\end{figure*}

\section{\Asynctuner}
\label{sec:async_tuner}

Asynchrony is a parallelization technique that avoids synchronization barriers \citep{recht2011hogwild}. 
In this section, we propose a {\em closed momentum loop} variant of \tuner to accelerate convergence in asynchronous training. 
%To handle the momentum dynamics of asynchronous parallelism, we propose a {\em closed momentum loop} variant of \tuner.
After some preliminaries, we show the mechanism of the extension: 
it measures the dynamics on a running system and controls momentum with a negative feedback loop.
\paragraph{Preliminaries}
%Asynchrony is a popular parallelization technique \citep{recht2011hogwild} that avoids synchronization barriers.
When training on $M$ asynchronous workers, staleness (the number of model updates between a worker's read and write operations) is on average $\tau=M-1$,
i.e., the gradient in the SGD update is delayed by $\tau$ iterations as $\nabla f_{S_{t - \tau}}(x_{t - \tau} )$.
Asynchrony yields faster steps, but can
increase the number of iterations to achieve the same solution,
a tradeoff between hardware and statistical 
efficiency~\citep{DBLP:journals/pvldb/ZhangR14}.
\citet{mitliagkas2016asynchrony} interpret asynchrony as added momentum dynamics.
Experiments in \citet{hadjis2016omnivore} support this finding, and demonstrate that reducing algorithmic momentum can compensate for asynchrony-induced momentum
and significantly reduce the number of iterations for convergence.
Motivated by that result, we use the model
in~\eqref{equ:exp_async_update_app}, where the total momentum, $\mu_T$, includes both asynchrony-induced and algorithmic  momentum, $\mu$, in~\eqref{eqn:momentum_gd}.
\begin{equation}
	\mathbb{E}[ x_{t+1} - x_t ] 
	= \mu_T \mathbb{E}[x_t - x_{t-1}] - \alpha \mathbb{E}\nabla f(x_{t})
\label{equ:exp_async_update_app}
\end{equation}
We will use this expression to design an estimator for the value of total momentum, $\hat{\mu}_T$.
This estimator is a basic building block of \asynctuner, that {\em removes the need to manually compensate for the effects of asynchrony}.



\paragraph{Measuring the momentum dynamics}
\Asynctuner estimates total momentum $\mu_{T}$ on a running system and uses a negative feedback loop to adjust algorithmic momentum accordingly.
Equation~\eqref{equ:exp_async_update} gives an estimate of $\hat{\mu}_T$ on a system with staleness $\tau$, based on \eqref{equ:exp_async_update}.
\begin{align}
\hat{\mu}_T
					= \mathop{\mathsf{median}}\left(
							\frac{x_{t - \tau} - x_{t - \tau-1} + \alpha \nabla_{S_{t-\tau -1}} f(x_{t - \tau - 1} )}
							{x_{t - \tau-1} - x_{t - \tau-2}}
					\right)
\label{eqn:momentum_measurement}
\end{align}
We use $\tau$-stale model values to match the staleness of the gradient,  and perform all operations in an elementwise fashion. 
This way we get a total momentum measurement from each variable; 
the median combines them into a more robust estimate.

\paragraph{Closing the asynchrony loop}
Given a reliable measurement of $\mu_{T}$, 
we can use it to adjust the value of algorithmic momentum so that the total momentum matches the \emph{target momentum} as decided by \tuner in Algorithm~\ref{alg:basic-algo}.
\Asynctuner in Algorithm~\ref{alg:async-algo} %(in Appendix~\ref{sec:async_yf}) 
uses a simple negative feedback loop to achieve the adjustment.
%Figure~\ref{fig:we-can-measure} demonstrates that under asynchrony the measured total momentum is strictly higher than the algorithmic momentum (middle plot), as expected from theory;
%closing the feedback loop (right plot) leads to total momentum matching the target momentum.
%Closing the loop, as we will see, improves performance significantly.
%Note for asynchronous-parallel training, as the estimates and parameter tuning is unstable in the beginning when there are only a small number of iterations, we use initial learning $\frac{1}{\tau + 1}$ instead of $1.0$ to prevent overflow in the beginning. 

%\begin{algorithm}[H]
%	\caption{\Asynctuner}
%	\begin{algorithmic}[1]
%%	\State Input: $\mu\gets0$, $\alpha \gets \frac{1}{\tau + 1}$, $\gamma\gets0.01, \tau$ (staleness)
%	\State Input: $\mu\gets0$, $\alpha \gets 0.0001$, $\gamma\gets0.01, \tau$ (staleness)
%	\For { $t\gets1$ to $T$}
%	\State $x_t\!\gets\!x_{t - 1} + \mu (x_{t - 1} - x_{t - 2} ) - \alpha \nabla_{S_t} f(x_{t - \tau - 1} )$
%	\State $\mu^*,\alpha \gets \Call{\tuner}{\nabla_{S_t} f(x_{t - \tau - 1} ), \beta}$ %(get momentum from the dynamic range)
%	\State $\hat{\mu_T} 
%					\gets \mathop{\mathsf{median}}\left(
%							\frac{x_{t - \tau} - x_{t - \tau-1} + \alpha \nabla_{S_{t-\tau-1}} f(x_{t - \tau - 1} )}
%							{x_{t - \tau-1} - x_{t - \tau-2}}
%					\right)$ \Comment{Measuring total momentum}
%	\State $\mu \leftarrow \mu + \gamma \cdot (\mu^* - \hat{\mu_T})$ \Comment{Closing the loop}
%	\EndFor
%\end{algorithmic}
%\label{alg:async-algo}
%\end{algorithm}




%In Section~\ref{sec:async_tuner}, we briefly discuss the mechanism of our designed \Asynctuner in asynchronous-parallel setting. In this appendix, we expand the details in total momentum estimator, $\hat{\mu_T}$, and present the full \Asynctuner in Algorithm~\ref{alg:async-algo} with extensive discussion.
%\paragraph{Measuring the momentum dynamics}
%Remember, we use the formula in~\eqref{equ:exp_async_update_app} to model the momentum dynamics in asynchronous-parallel systems
%\Asynctuner estimates total momentum $\mu_{T}$ on a running system and uses a negative feedback loop to adjust algorithmic momentum accordingly.
%\begin{equation}
%	\mathbb{E}[ x_{t+1} - x_t ] 
%	= \mu_T \mathbb{E}[x_t - x_{t-1}] - \alpha \mathbb{E}\nabla f(x_{t})
%\label{equ:exp_async_update_app}
%\end{equation}
%Equation~\eqref{eqn:momentum_measurement_app} gives an estimate of $\hat{\mu_T}$ on a system with staleness $\tau$, based on \eqref{equ:exp_async_update_app}.
%\begin{align}
%\hat{\mu_T}
%					= \mathop{\mathsf{median}}\left(
%							\frac{x_{t - \tau} - x_{t - \tau-1} + \alpha \nabla_{S_{t-\tau -1}} f(x_{t - \tau - 1} )}
%							{x_{t - \tau-1} - x_{t - \tau-2}}
%					\right)
%\label{eqn:momentum_measurement_app}
%\end{align}
%We use $\tau$-stale model values to match the staleness of the gradient,  and perform all operations in an elementwise fashion. 
%This way we get a total momentum measurement from each variable; 
%the median combines them into a more robust estimate.
%
%%\label{subsec:closed_loop_YF}
%%\begin{figure}
%%\centering
%%\includegraphics[width=0.95\linewidth]{experiment_results/resnet/mom_dynamic_3_annotated.pdf}
%%	\caption{
%%	Momentum dynamics on CIFAR100 ResNet.
%%	Running \tuner, total momentum is equal to algorithmic momentum in a synchronous setting (left). Total momentum is greater than algorithmic momentum on 16 asynchronous workers, due to asynchrony-induced momentum (middle).
%%	Using the momentum feedback mechanism of \asynctuner, lowers algorithmic momentum and brings total momentum to match the target value on 16 asynchronous workers (right).
%%	Red dots are individual total momentum estimates, $\hat{\mu}_T$, at each iteration. 
%%The solid red line is a running average of those estimates.	
%%	}
%%	\label{fig:we-can-measure}
%%\end{figure}
%
%\paragraph{Closing the asynchrony loop}
%Given a reliable measurement of $\mu_{T}$, 
%we can use it to adjust the value of algorithmic momentum so that the total momentum matches the \emph{target momentum} as decided by \tuner in Algorithm~\ref{alg:basic-algo}.
%\Asynctuner in Algorithm~\ref{alg:async-algo} %(in Appendix~\ref{sec:async_yf}) 
%uses a simple negative feedback loop to achieve the adjustment.
%%Figure~\ref{fig:we-can-measure} demonstrates that under asynchrony the measured total momentum is strictly higher than the algorithmic momentum (middle plot), as expected from theory;
%%closing the feedback loop (right plot) leads to total momentum matching the target momentum.
%%Closing the loop, as we will see, improves performance significantly.
%%%Note for asynchronous-parallel training, as the estimates and parameter tuning is unstable in the beginning when there are only a small number of iterations, we use initial learning $\frac{1}{\tau + 1}$ instead of $1.0$ to prevent overflow in the beginning. 
%%
%%%\begin{algorithm}[H]
%%%	\caption{\Asynctuner}
%%%	\begin{algorithmic}[1]
%%%%	\State Input: $\mu\gets0$, $\alpha \gets \frac{1}{\tau + 1}$, $\gamma\gets0.01, \tau$ (staleness)
%%%	\State Input: $\mu\gets0$, $\alpha \gets 0.0001$, $\gamma\gets0.01, \tau$ (staleness)
%%%	\For { $t\gets1$ to $T$}
%%%	\State $x_t\!\gets\!x_{t - 1} + \mu (x_{t - 1} - x_{t - 2} ) - \alpha \nabla_{S_t} f(x_{t - \tau - 1} )$
%%%	\State $\mu^*,\alpha \gets \Call{\tuner}{\nabla_{S_t} f(x_{t - \tau - 1} ), \beta}$ %(get momentum from the dynamic range)
%%%	\State $\hat{\mu_T} 
%%%					\gets \mathop{\mathsf{median}}\left(
%%%							\frac{x_{t - \tau} - x_{t - \tau-1} + \alpha \nabla_{S_{t-\tau-1}} f(x_{t - \tau - 1} )}
%%%							{x_{t - \tau-1} - x_{t - \tau-2}}
%%%					\right)$ \Comment{Measuring total momentum}
%%%	\State $\mu \leftarrow \mu + \gamma \cdot (\mu^* - \hat{\mu_T})$ \Comment{Closing the loop}
%%%	\EndFor
%%%\end{algorithmic}
%%%\label{alg:async-algo}
%%%\end{algorithm}
%%
%
%
%




\begin{algorithm}[h]
	\caption{\Asynctuner}
	\begin{algorithmic}[1]
%	\State Input: $\mu\gets0$, $\alpha \gets \frac{1}{\tau + 1}$, $\gamma\gets0.01, \tau$ (staleness)
	\State Input: $\mu\gets0$, $\alpha \gets 0.0001$, $\gamma\gets0.01, \tau$ (staleness)
	\For { $t\gets1$ to $T$}
	\State $x_t\!\gets\!x_{t - 1} + \mu (x_{t - 1} - x_{t - 2} ) - \alpha \nabla_{S_t} f(x_{t - \tau - 1} )$
	\State $\mu^*,\alpha \gets \Call{\tuner}{\nabla_{S_t} f(x_{t - \tau - 1} ), \beta}$ %(get momentum from the dynamic range)
	\State $\hat{\mu_T} 
					\gets \mathop{\mathsf{median}}\left(
							\frac{x_{t - \tau} - x_{t - \tau-1} + \alpha \nabla_{S_{t-\tau-1}} f(x_{t - \tau - 1} )}
							{x_{t - \tau-1} - x_{t - \tau-2}}
					\right)$ \Comment{Measuring total momentum}
	\State $\mu \leftarrow \mu + \gamma \cdot (\mu^* - \hat{\mu_T})$ \Comment{Closing the loop}
	\EndFor
\end{algorithmic}
\label{alg:async-algo}
\end{algorithm}


%%%%%%%%%%%%%%%%% latest backup version %%%%%%%%%%%%%%%%%%%%%%%%%%%%%%%%%%%%
%Asynchrony is a parallelization technique that avoids synchronization barriers \citep{recht2011hogwild}. 
%%In this section, we propose a {\em closed momentum loop} variant of \tuner to accelerate convergence in asynchronous training. 
%%To handle the momentum dynamics of asynchronous parallelism, we propose a {\em closed momentum loop} variant of \tuner.
%%After some preliminaries, we show the mechanism of the extension: 
%%it measures the dynamics on a running system and controls momentum with a negative feedback loop.
%%\paragraph{Preliminaries}
%%Asynchrony is a popular parallelization technique \citep{recht2011hogwild} that avoids synchronization barriers.
%%When training on $M$ asynchronous workers, staleness (the number of model updates between a worker's read and write operations) is on average $\tau=M-1$,
%%i.e., the gradient in the SGD update is delayed by $\tau$ iterations as $\nabla f_{S_{t - \tau}}(x_{t - \tau} )$.
%%It yields faster steps, but can
%%increase the number of iterations needed,
%%a tradeoff between hardware and statistical 
%%efficiency~\citep{DBLP:journals/pvldb/ZhangR14}.
%It yields better hardware efficiency, i.e. faster steps, but can
%increase the number of iterations to a given metric, i.e. statistical efficiency, as a tradeoff~\citep{DBLP:journals/pvldb/ZhangR14}.
%%a tradeoff between hardware and statistical 
%%efficiency~\citep{DBLP:journals/pvldb/ZhangR14}.
%%In this section, we propose a {\em closed momentum loop} variant of \tuner to reduce the number of iterations it needs to converge in asynchronous training. 
%%In this section, we propose a {\em closed momentum loop} variant of \tuner to reduce the number of iterations for convergence in asynchronous training.
%%\paragraph{\Asynctuner}
%\citet{mitliagkas2016asynchrony} interpret asynchrony as added momentum dynamics.
%%It is empirically supported in \citet{hadjis2016omnivore} that manually reducing algorithmic momentum can compensate for asynchrony-induced momentum
%%and significantly reduce the number of iterations to converge.
%We design \asynctuner, a variant of \tuner to automatically control algorithmic momentum, compensate for asynchrony and accelerate convergence.
%We use the formula in~\eqref{equ:exp_async_update} to model the dynamics in the system, where the total momentum, $\mu_T$, includes both asynchrony-induced and algorithmic  momentum, $\mu$, in~\eqref{eqn:momentum_gd}.
%\begin{equation}
%	\mathbb{E}[ x_{t+1} - x_t ] 
%	= \mu_T \mathbb{E}[x_t - x_{t-1}] - \alpha \mathbb{E}\nabla f(x_{t})
%\label{equ:exp_async_update}
%\end{equation}
%We first use~\eqref{equ:exp_async_update} to design an robust estimator $\hat{\mu}_T$ for the value of total momentum at every iteration.
%%This estimator is a basic building block of \asynctuner, that {\em removes the need to manually compensate for the effects of asynchrony}. 
%Then we use a simple negative feedback control loop to adjust the value of algorithmic momentum so that $\hat{\mu}_T$ matches the \emph{target momentum} decided by \tuner in Algorithm~\ref{alg:basic-algo}. 
%%We refer to Appendix~\ref{sec:async_app} for details on estimator $\hat{\mu}_T$ and \Asynctuner in Algorithm~\ref{alg:async-algo}.
%%\Asynctuner in Algorithm~\ref{alg:async-algo} (in Appendix~\ref{sec:async_app}) %(in Appendix~\ref{sec:async_yf}) 
%%uses a simple negative feedback loop to achieve the adjustment.
%In Figure~\ref{fig:we-can-measure}, 
%we demonstrate momentum dynamics in an asynchronous training system. 
%As directly using the target value as algorithmic momentum, \tuner (middle) presents total momentum $\hat{\mu}_T$ strictly larger than the target momentum, due to asynchrony-induced momentum. \Asynctuner (right) automatically brings down algorithmic momentum, match measured total momentum $\hat{\mu}_T$ to target value and, as we will see, speeds up convergence comparing to \tuner. We refer to Appendix~\ref{sec:async_app} for details on estimator $\hat{\mu}_T$ and \Asynctuner in Algorithm~\ref{alg:async-algo}.
%%
%%so that visually demonstrates the mechanism of \Asynctuner in handling the momentum dynamics under asynchrony. In asynchronous-parallel setting, the measured total momentum is strictly higher than the algorithmic momentum (middle plot), as expected from theory.
%%Closing the feedback loop (right plot) leads to total momentum matching the target momentum and, as we will see, improves performance significantly.
%
%%\begin{figure*}
%%%\vspace{-2.5em}
%%\centering
%%\includegraphics[width=\linewidth]{experiment_results/resnet/mom_dynamic_3_annotated.pdf}
%%	\caption{
%%	When running \tuner, total momentum $\hat{\mu}_t$ equals algorithmic value in synchronous settings (left); $\hat{\mu}_t$ is greater than algorithmic value on 16 asynchronous workers (middle).
%%	\Asynctuner automatically lowers algorithmic momentum and brings total momentum to match the target value (right).
%%Red dots are measured $\hat{\mu}_t$ at every step with red line as its running average.
%%%	Red dots are total momentum estimates, $\hat{\mu}_T$, at each iteration. 
%%%The solid red line is a running average of $\hat{\mu}_T$.	
%%	}
%%	\vspace{-0.25em}
%%	\label{fig:we-can-measure}
%%\end{figure*}


%%%%%%%%%%%%%%%%%%%%%% latest backup versions %%%%%%%%%%%%%%%%%%%%%%%%%%%%%%%%%%%%%%%%%
%%\begin{figure}
%%\centering
%%\includegraphics[width=0.95\linewidth]{experiment_results/resnet/mom_dynamic_3_annotated.pdf}
%%	\caption{
%%%	Momentum dynamics on CIFAR100 ResNet.
%%	Running \tuner on a ResNet, total momentum equals algorithmic value in a synchronous setting (left). Total momentum is greater than algorithmic value on 16 asynchronous workers, due to asynchrony-induced momentum (middle).
%%	\asynctuner automatically lowers algorithmic momentum and brings total momentum to match the target value (right).
%%	Red dots are total momentum estimates, $\hat{\mu}_T$, at each iteration. 
%%The solid red line is a running average of $\hat{\mu}_T$.	
%%	}
%%	\label{fig:we-can-measure}
%%\end{figure}
%Asynchrony is a parallelization technique that avoids synchronization barriers \citep{recht2011hogwild}. 
%%In this section, we propose a {\em closed momentum loop} variant of \tuner to accelerate convergence in asynchronous training. 
%%To handle the momentum dynamics of asynchronous parallelism, we propose a {\em closed momentum loop} variant of \tuner.
%%After some preliminaries, we show the mechanism of the extension: 
%%it measures the dynamics on a running system and controls momentum with a negative feedback loop.
%%\paragraph{Preliminaries}
%%Asynchrony is a popular parallelization technique \citep{recht2011hogwild} that avoids synchronization barriers.
%%When training on $M$ asynchronous workers, staleness (the number of model updates between a worker's read and write operations) is on average $\tau=M-1$,
%%i.e., the gradient in the SGD update is delayed by $\tau$ iterations as $\nabla f_{S_{t - \tau}}(x_{t - \tau} )$.
%%It yields faster steps, but can
%%increase the number of iterations needed,
%%a tradeoff between hardware and statistical 
%%efficiency~\citep{DBLP:journals/pvldb/ZhangR14}.
%It yields better hardware efficiency, i.e. faster steps, but can
%increase the number of iterations to a given metric, i.e. statistical efficiency, as a tradeoff~\citep{DBLP:journals/pvldb/ZhangR14}.
%%a tradeoff between hardware and statistical 
%%efficiency~\citep{DBLP:journals/pvldb/ZhangR14}.
%%In this section, we propose a {\em closed momentum loop} variant of \tuner to reduce the number of iterations it needs to converge in asynchronous training. 
%%In this section, we propose a {\em closed momentum loop} variant of \tuner to reduce the number of iterations for convergence in asynchronous training.
%%\paragraph{\Asynctuner}
%\citet{mitliagkas2016asynchrony} interpret asynchrony as added momentum dynamics.
%%It is empirically supported in \citet{hadjis2016omnivore} that manually reducing algorithmic momentum can compensate for asynchrony-induced momentum
%%and significantly reduce the number of iterations to converge.
%We design a {\em closed momentum loop} variant of \tuner to control algorithmic momentum, compensate for asynchrony and accelerate convergence.
%We use the formula in~\eqref{equ:exp_async_update} to model the dynamics in the system, where the total momentum, $\mu_T$, includes both asynchrony-induced and algorithmic  momentum, $\mu$, in~\eqref{eqn:momentum_gd}.
%\begin{equation}
%	\mathbb{E}[ x_{t+1} - x_t ] 
%	= \mu_T \mathbb{E}[x_t - x_{t-1}] - \alpha \mathbb{E}\nabla f(x_{t})
%\label{equ:exp_async_update}
%\end{equation}
%We first use this expression to design an robust estimator $\hat{\mu}_T$ for the value of total momentum.
%%This estimator is a basic building block of \asynctuner, that {\em removes the need to manually compensate for the effects of asynchrony}.
%Given $\hat{\mu}_T$,  
%we use it to adjust the value of algorithmic momentum so that the total momentum matches the \emph{target momentum} decided by \tuner in Algorithm~\ref{alg:basic-algo}. Specifically, we %\asynctuner  %(in Appendix~\ref{sec:async_yf}) 
%uses a simple negative feedback loop to achieve the adjustment. We refer to Appendix~\ref{sec:async_app} for details on estimator $\hat{\mu}_T$ and \Asynctuner in Algorithm~\ref{alg:async-algo}.
%%\Asynctuner in Algorithm~\ref{alg:async-algo} (in Appendix~\ref{sec:async_app}) %(in Appendix~\ref{sec:async_yf}) 
%%uses a simple negative feedback loop to achieve the adjustment.
%Figure~\ref{fig:we-can-measure} visually demonstrates the mechanism of \Asynctuner in handling the momentum dynamics under asynchrony. In asynchronous-parallel setting, the measured total momentum is strictly higher than the algorithmic momentum (middle plot), as expected from theory.
%Closing the feedback loop (right plot) leads to total momentum matching the target momentum and, as we will see, improves performance significantly.
%
%\begin{figure}
%\centering
%\includegraphics[width=0.95\linewidth]{experiment_results/resnet/mom_dynamic_3_annotated.pdf}
%	\caption{
%%	Momentum dynamics on CIFAR100 ResNet.
%	Running \tuner on a ResNet, total momentum equals algorithmic value in a synchronous setting (left). Total momentum is greater than algorithmic value on 16 asynchronous workers, due to asynchrony-induced momentum (middle).
%	\asynctuner automatically lowers algorithmic momentum and brings total momentum to match the target value (right).
%	Red dots are total momentum estimates, $\hat{\mu}_T$, at each iteration. 
%The solid red line is a running average of $\hat{\mu}_T$.	
%	}
%	\label{fig:we-can-measure}
%\end{figure}
%

%%%%%%%%%%%%%%%%%%%%%% below are old backup versions %%%%%%%%%%%%%%%%%%%%%%%%%%%%%%%%%%

%\paragraph{Measuring the momentum dynamics}
%\Asynctuner estimates total momentum $\mu_{T}$ on a running system and uses a negative feedback loop to adjust algorithmic momentum accordingly.
%Equation~\eqref{equ:exp_async_update} gives an estimate of $\hat{\mu_T}$ on a system with staleness $\tau$, based on \eqref{equ:exp_async_update}.
%\begin{align}
%\hat{\mu_T}
%					= \mathop{\mathsf{median}}\left(
%							\frac{x_{t - \tau} - x_{t - \tau-1} + \alpha \nabla_{S_{t-\tau -1}} f(x_{t - \tau - 1} )}
%							{x_{t - \tau-1} - x_{t - \tau-2}}
%					\right)
%\label{eqn:momentum_measurement}
%\end{align}
%We use $\tau$-stale model values to match the staleness of the gradient,  and perform all operations in an elementwise fashion. 
%This way we get a total momentum measurement from each variable; 
%the median combines them into a more robust estimate.

%\label{subsec:closed_loop_YF}
%\begin{figure}
%\centering
%\includegraphics[width=0.95\linewidth]{experiment_results/resnet/mom_dynamic_3_annotated.pdf}
%	\caption{
%	Momentum dynamics on CIFAR100 ResNet.
%	Running \tuner, total momentum is equal to algorithmic momentum in a synchronous setting (left). Total momentum is greater than algorithmic momentum on 16 asynchronous workers, due to asynchrony-induced momentum (middle).
%	Additionally applying the momentum feedback loop of \asynctuner, lowers algorithmic momentum and matches total momentum to target value (right).
%	Red dots are individual total momentum estimates, $\hat{\mu}_T$, at each iteration, with
%the solid red line as its running average.	
%	}
%	\label{fig:we-can-measure}
%\end{figure}

%\paragraph{Closing the asynchrony loop}
%Given a reliable measurement of $\mu_{T}$, 
%we can use it to adjust the value of algorithmic momentum so that the total momentum matches the \emph{target momentum} as decided by \tuner in Algorithm~\ref{alg:basic-algo}.
%\Asynctuner in Algorithm~\ref{alg:async-algo} (in Appendix~\ref{sec:async_app}) %(in Appendix~\ref{sec:async_yf}) 
%uses a simple negative feedback loop to achieve the adjustment.
%Figure~\ref{fig:we-can-measure} demonstrates that under asynchrony the measured total momentum is strictly higher than the algorithmic momentum (middle plot), as expected from theory;
%closing the feedback loop (right plot) leads to total momentum matching the target momentum.
%Closing the loop, as we will see, improves performance significantly.
%





%To handle the momentum dynamics of asynchronous parallelism, we propose a {\em closed momentum loop} variant of \tuner.
%After some preliminaries, we show the mechanism of the extension: 
%it measures the dynamics on a running system and controls momentum with a negative feedback loop.
%\paragraph{Preliminaries}
%Asynchrony is a popular parallelization technique \citep{recht2011hogwild} that avoids synchronization barriers.
%When training on $M$ asynchronous workers, staleness (the number of model updates between a worker's read and write operations) is on average $\tau=M-1$,
%i.e., the gradient in the SGD update is delayed by $\tau$ iterations as $\nabla f_{S_{t - \tau}}(x_{t - \tau} )$.
%Asynchrony yields faster steps, but can
%increase the number of iterations to achieve the same solution,
%a tradeoff between hardware and statistical 
%efficiency~\citep{DBLP:journals/pvldb/ZhangR14}.
%\citet{mitliagkas2016asynchrony} interpret asynchrony as added momentum dynamics.
%Experiments in \citet{hadjis2016omnivore} support this finding, and demonstrate that reducing algorithmic momentum can compensate for asynchrony-induced momentum
%and significantly reduce the number of iterations for convergence.
%Motivated by that result, we use the model
%in~\eqref{equ:exp_async_update}, where the total momentum, $\mu_T$, includes both asynchrony-induced and algorithmic  momentum, $\mu$, in~\eqref{eqn:momentum_gd}.
%\begin{equation}
%	\mathbb{E}[ x_{t+1} - x_t ] 
%	= \mu_T \mathbb{E}[x_t - x_{t-1}] - \alpha \mathbb{E}\nabla f(x_{t})
%\label{equ:exp_async_update}
%\end{equation}
%We will use this expression to design an estimator for the value of total momentum, $\hat{\mu_T}$.
%This estimator is a basic building block of \asynctuner, that {\em removes the need to manually compensate for the effects of asynchrony}.
%
%
%
%\paragraph{Measuring the momentum dynamics}
%\Asynctuner estimates total momentum $\mu_{T}$ on a running system and uses a negative feedback loop to adjust algorithmic momentum accordingly.
%Equation~\eqref{equ:exp_async_update} gives an estimate of $\hat{\mu_T}$ on a system with staleness $\tau$, based on \eqref{equ:exp_async_update}.
%\begin{align}
%\hat{\mu_T}
%					= \mathop{\mathsf{median}}\left(
%							\frac{x_{t - \tau} - x_{t - \tau-1} + \alpha \nabla_{S_{t-\tau -1}} f(x_{t - \tau - 1} )}
%							{x_{t - \tau-1} - x_{t - \tau-2}}
%					\right)
%\label{eqn:momentum_measurement}
%\end{align}
%We use $\tau$-stale model values to match the staleness of the gradient,  and perform all operations in an elementwise fashion. 
%This way we get a total momentum measurement from each variable; 
%the median combines them into a more robust estimate.
%
%\label{subsec:closed_loop_YF}
%\begin{figure}
%\centering
%\includegraphics[width=0.95\linewidth]{experiment_results/resnet/mom_dynamic_3_annotated.pdf}
%	\caption{
%	Momentum dynamics on CIFAR100 ResNet.
%	Running \tuner, total momentum is equal to algorithmic momentum in a synchronous setting (left). Total momentum is greater than algorithmic momentum on 16 asynchronous workers, due to asynchrony-induced momentum (middle).
%	Using the momentum feedback mechanism of \asynctuner, lowers algorithmic momentum and brings total momentum to match the target value on 16 asynchronous workers (right).
%	Red dots are individual total momentum estimates, $\hat{\mu}_T$, at each iteration. 
%The solid red line is a running average of those estimates.	
%	}
%	\label{fig:we-can-measure}
%\end{figure}
%
%\paragraph{Closing the asynchrony loop}
%Given a reliable measurement of $\mu_{T}$, 
%we can use it to adjust the value of algorithmic momentum so that the total momentum matches the \emph{target momentum} as decided by \tuner in Algorithm~\ref{alg:basic-algo}.
%\Asynctuner in Algorithm~\ref{alg:async-algo} (in Appendix~\ref{sec:async_app}) %(in Appendix~\ref{sec:async_yf}) 
%uses a simple negative feedback loop to achieve the adjustment.
%Figure~\ref{fig:we-can-measure} demonstrates that under asynchrony the measured total momentum is strictly higher than the algorithmic momentum (middle plot), as expected from theory;
%closing the feedback loop (right plot) leads to total momentum matching the target momentum.
%Closing the loop, as we will see, improves performance significantly.
%%Note for asynchronous-parallel training, as the estimates and parameter tuning is unstable in the beginning when there are only a small number of iterations, we use initial learning $\frac{1}{\tau + 1}$ instead of $1.0$ to prevent overflow in the beginning. 
%
%%\begin{algorithm}[H]
%%	\caption{\Asynctuner}
%%	\begin{algorithmic}[1]
%%%	\State Input: $\mu\gets0$, $\alpha \gets \frac{1}{\tau + 1}$, $\gamma\gets0.01, \tau$ (staleness)
%%	\State Input: $\mu\gets0$, $\alpha \gets 0.0001$, $\gamma\gets0.01, \tau$ (staleness)
%%	\For { $t\gets1$ to $T$}
%%	\State $x_t\!\gets\!x_{t - 1} + \mu (x_{t - 1} - x_{t - 2} ) - \alpha \nabla_{S_t} f(x_{t - \tau - 1} )$
%%	\State $\mu^*,\alpha \gets \Call{\tuner}{\nabla_{S_t} f(x_{t - \tau - 1} ), \beta}$ %(get momentum from the dynamic range)
%%	\State $\hat{\mu_T} 
%%					\gets \mathop{\mathsf{median}}\left(
%%							\frac{x_{t - \tau} - x_{t - \tau-1} + \alpha \nabla_{S_{t-\tau-1}} f(x_{t - \tau - 1} )}
%%							{x_{t - \tau-1} - x_{t - \tau-2}}
%%					\right)$ \Comment{Measuring total momentum}
%%	\State $\mu \leftarrow \mu + \gamma \cdot (\mu^* - \hat{\mu_T})$ \Comment{Closing the loop}
%%	\EndFor
%%\end{algorithmic}
%%\label{alg:async-algo}
%%\end{algorithm}
%

\section{Practical implementation}
\label{sec:practical_impl}
In Section~\ref{sec:oracles}, we discuss estimators for learning rate and momentum tuning in \tuner. In our experiment practice, we have identified a few practical implementation details which are important for improving estimators. Zero-debias is proposed by~\citet{kingma2014adam}, which accelerates the process where exponential average adapts to the level of original quantity in the beginning. We applied zero-debias to all the exponential average quantities involved in our estimators. In some LSTM models, we observe that our estimated curvature may decrease quickly along the optimization process. In order to better estimate extremal curvature $h_{\max}$ and $h_{\min}$ with fast decreasing trend, we apply zero-debias exponential average on the logarithmic of $h_{\max, t}$ and $h_{\min, t}$, instead of directly on $h_{\max, t}$ and $h_{\min, t}$. Except from the above two techniques, we also implemented the slow start heuristic proposed by~\citep{schaul2013no}. More specifically, we use $\alpha = \min\{\alpha_t, t \cdot \alpha_t / (10 \cdot w) \}$ as our learning rate with $w$ as the size of our sliding window in $h_{\max}$ and $h_{\min}$ estimation. It discount the learning rate in the first $10 \cdot w$ steps and helps to keep the learning rate small in the beginning when the exponential averaged quantities are not accurate enough.
\section{Adaptive gradient clipping in \tuner}
\label{sec:adapt_clip}
\begin{figure}
%	%\begin{minipage}{0.61\textwidth}
\centering
%% \vspace{-2.25em}
  \includegraphics[width=0.7\linewidth]{experiment_results/clipping_example.pdf} 
% \vspace{-0.75em}
\caption{A variation of the LSTM architecture in \citep{zhu2016trained} exhibits exploding gradients.
The proposed adaptive gradient clipping threshold (blue) stabilizes the training loss.}
\label{fig:stability}
%\end{minipage}
\end{figure}

Gradient clipping has been established in literature as a standard---almost necessary---tool for training such objectives \citep{pascanu2013difficulty,Goodfellow-et-al-2016,gehring2017convolutional}. 
However, the classic tradeoff between adaptivity and stability applies: 
setting a clipping threshold that is too low can hurt performance;
setting it to be high, can compromise stability.
\tuner, keeps running estimates of extremal gradient magnitude squares, $h_{max}$ and $h_{min}$ in order to estimate a generalized condition number.
We posit that $\sqrt{h_{max}}$ is an ideal gradient norm threshold for adaptive clipping.
In order to ensure robustness to extreme gradient spikes, like the ones in Figure~\ref{fig:stability}, we also limit the growth rate of the envelope $h_{max}$ in Algorithm~\ref{alg:curv_func} as follows:
\begin{equation}
 h_{max} 
 \leftarrow
 \beta \cdot h_{max}
 	+ (1-\beta) \cdot \textrm{min}\left\{
 		h_{max,t}, 100 \cdot h_{max}
 	\right\}
\end{equation}
Our heuristics follows along the lines of classic recipes like~\cite{pascanu2013difficulty}. However, instead of using the average gradient norm to clip, it uses a running estimate of the maximum norm $h_{\max}$.

In Section~\ref{sec:stability}, we saw that adaptive clipping stabilizes the training on objectives that exhibit exploding gradients. In Figure~\ref{fig:infl_clip}, we demonstrate that the adaptive clipping does not hurt performance on models that do not exhibit instabilities without clipping. Specifically, for both PTB LSTM and CIFAR10 ResNet, the difference between \tuner with and without adaptive clipping diminishes quickly.  
\label{sec:infl_clip}
\begin{figure}
\centering
\begin{tabular}{c c}
	\includegraphics[width=0.35\linewidth]{experiment_results/ptb/clip_cmp.pdf} &
	\includegraphics[width=0.35\linewidth]{experiment_results/resnet/cifar10_clip_cmp.pdf}
\end{tabular}
\caption{Training losses on PTB LSTM (left) and CIFAR10 ResNet (right) for YellowFin with and without adaptive clipping.}
\label{fig:infl_clip}
\end{figure}

\section{Closed-loop \tuner for asynchronous training}
\label{sec:async_app}
In Section~\ref{sec:async_tuner}, we briefly discuss the closed-loop momentum control mechanism in \asynctuner. In this section, after presenting more preliminaries on asynchrony, we show with details on the mechanism: 
it measures the dynamics on a running system and controls momentum with a negative feedback loop.
\paragraph{Preliminaries}
Asynchrony is a popular parallelization technique \citep{recht2011hogwild} that avoids synchronization barriers.
When training on $M$ asynchronous workers, staleness (the number of model updates between a worker's read and write operations) is on average $\tau=M-1$,
i.e., the gradient in the SGD update is delayed by $\tau$ iterations as $\nabla f_{S_{t - \tau}}(x_{t - \tau} )$.
Asynchrony yields faster steps, but can
increase the number of iterations to achieve the same solution,
a tradeoff between hardware and statistical 
efficiency~\citep{DBLP:journals/pvldb/ZhangR14}.
\citet{mitliagkas2016asynchrony} interpret asynchrony as added momentum dynamics.
Experiments in \citet{hadjis2016omnivore} support this finding, and demonstrate that reducing algorithmic momentum can compensate for asynchrony-induced momentum
and significantly reduce the number of iterations for convergence.
Motivated by that result, we use the model
in~\eqref{equ:exp_async_update_app}, where the total momentum, $\mu_T$, includes both asynchrony-induced and algorithmic  momentum, $\mu$, in~\eqref{eqn:momentum_gd}.
\begin{equation}
	\mathbb{E}[ x_{t+1} - x_t ] 
	= \mu_T \mathbb{E}[x_t - x_{t-1}] - \alpha \mathbb{E}\nabla f(x_{t})
\label{equ:exp_async_update_app}
\end{equation}
We will use this expression to design an estimator for the value of total momentum, $\hat{\mu}_T$.
This estimator is a basic building block of \asynctuner, that {\em removes the need to manually compensate for the effects of asynchrony}.



\paragraph{Measuring the momentum dynamics}
\Asynctuner estimates total momentum $\mu_{T}$ on a running system and uses a negative feedback loop to adjust algorithmic momentum accordingly.
Equation~\eqref{equ:exp_async_update} gives an estimate of $\hat{\mu}_T$ on a system with staleness $\tau$, based on \eqref{equ:exp_async_update}.
\begin{align}
\hat{\mu}_T
					= \mathop{\mathsf{median}}\left(
							\frac{x_{t - \tau} - x_{t - \tau-1} + \alpha \nabla_{S_{t-\tau -1}} f(x_{t - \tau - 1} )}
							{x_{t - \tau-1} - x_{t - \tau-2}}
					\right)
\label{eqn:momentum_measurement}
\end{align}
We use $\tau$-stale model values to match the staleness of the gradient,  and perform all operations in an elementwise fashion. 
This way we get a total momentum measurement from each variable; 
the median combines them into a more robust estimate.

\paragraph{Closing the asynchrony loop}
Given a reliable measurement of $\mu_{T}$, 
we can use it to adjust the value of algorithmic momentum so that the total momentum matches the \emph{target momentum} as decided by \tuner in Algorithm~\ref{alg:basic-algo}.
\Asynctuner in Algorithm~\ref{alg:async-algo} %(in Appendix~\ref{sec:async_yf}) 
uses a simple negative feedback loop to achieve the adjustment.
%Figure~\ref{fig:we-can-measure} demonstrates that under asynchrony the measured total momentum is strictly higher than the algorithmic momentum (middle plot), as expected from theory;
%closing the feedback loop (right plot) leads to total momentum matching the target momentum.
%Closing the loop, as we will see, improves performance significantly.
%Note for asynchronous-parallel training, as the estimates and parameter tuning is unstable in the beginning when there are only a small number of iterations, we use initial learning $\frac{1}{\tau + 1}$ instead of $1.0$ to prevent overflow in the beginning. 

%\begin{algorithm}[H]
%	\caption{\Asynctuner}
%	\begin{algorithmic}[1]
%%	\State Input: $\mu\gets0$, $\alpha \gets \frac{1}{\tau + 1}$, $\gamma\gets0.01, \tau$ (staleness)
%	\State Input: $\mu\gets0$, $\alpha \gets 0.0001$, $\gamma\gets0.01, \tau$ (staleness)
%	\For { $t\gets1$ to $T$}
%	\State $x_t\!\gets\!x_{t - 1} + \mu (x_{t - 1} - x_{t - 2} ) - \alpha \nabla_{S_t} f(x_{t - \tau - 1} )$
%	\State $\mu^*,\alpha \gets \Call{\tuner}{\nabla_{S_t} f(x_{t - \tau - 1} ), \beta}$ %(get momentum from the dynamic range)
%	\State $\hat{\mu_T} 
%					\gets \mathop{\mathsf{median}}\left(
%							\frac{x_{t - \tau} - x_{t - \tau-1} + \alpha \nabla_{S_{t-\tau-1}} f(x_{t - \tau - 1} )}
%							{x_{t - \tau-1} - x_{t - \tau-2}}
%					\right)$ \Comment{Measuring total momentum}
%	\State $\mu \leftarrow \mu + \gamma \cdot (\mu^* - \hat{\mu_T})$ \Comment{Closing the loop}
%	\EndFor
%\end{algorithmic}
%\label{alg:async-algo}
%\end{algorithm}




%In Section~\ref{sec:async_tuner}, we briefly discuss the mechanism of our designed \Asynctuner in asynchronous-parallel setting. In this appendix, we expand the details in total momentum estimator, $\hat{\mu_T}$, and present the full \Asynctuner in Algorithm~\ref{alg:async-algo} with extensive discussion.
%\paragraph{Measuring the momentum dynamics}
%Remember, we use the formula in~\eqref{equ:exp_async_update_app} to model the momentum dynamics in asynchronous-parallel systems
%\Asynctuner estimates total momentum $\mu_{T}$ on a running system and uses a negative feedback loop to adjust algorithmic momentum accordingly.
%\begin{equation}
%	\mathbb{E}[ x_{t+1} - x_t ] 
%	= \mu_T \mathbb{E}[x_t - x_{t-1}] - \alpha \mathbb{E}\nabla f(x_{t})
%\label{equ:exp_async_update_app}
%\end{equation}
%Equation~\eqref{eqn:momentum_measurement_app} gives an estimate of $\hat{\mu_T}$ on a system with staleness $\tau$, based on \eqref{equ:exp_async_update_app}.
%\begin{align}
%\hat{\mu_T}
%					= \mathop{\mathsf{median}}\left(
%							\frac{x_{t - \tau} - x_{t - \tau-1} + \alpha \nabla_{S_{t-\tau -1}} f(x_{t - \tau - 1} )}
%							{x_{t - \tau-1} - x_{t - \tau-2}}
%					\right)
%\label{eqn:momentum_measurement_app}
%\end{align}
%We use $\tau$-stale model values to match the staleness of the gradient,  and perform all operations in an elementwise fashion. 
%This way we get a total momentum measurement from each variable; 
%the median combines them into a more robust estimate.
%
%%\label{subsec:closed_loop_YF}
%%\begin{figure}
%%\centering
%%\includegraphics[width=0.95\linewidth]{experiment_results/resnet/mom_dynamic_3_annotated.pdf}
%%	\caption{
%%	Momentum dynamics on CIFAR100 ResNet.
%%	Running \tuner, total momentum is equal to algorithmic momentum in a synchronous setting (left). Total momentum is greater than algorithmic momentum on 16 asynchronous workers, due to asynchrony-induced momentum (middle).
%%	Using the momentum feedback mechanism of \asynctuner, lowers algorithmic momentum and brings total momentum to match the target value on 16 asynchronous workers (right).
%%	Red dots are individual total momentum estimates, $\hat{\mu}_T$, at each iteration. 
%%The solid red line is a running average of those estimates.	
%%	}
%%	\label{fig:we-can-measure}
%%\end{figure}
%
%\paragraph{Closing the asynchrony loop}
%Given a reliable measurement of $\mu_{T}$, 
%we can use it to adjust the value of algorithmic momentum so that the total momentum matches the \emph{target momentum} as decided by \tuner in Algorithm~\ref{alg:basic-algo}.
%\Asynctuner in Algorithm~\ref{alg:async-algo} %(in Appendix~\ref{sec:async_yf}) 
%uses a simple negative feedback loop to achieve the adjustment.
%%Figure~\ref{fig:we-can-measure} demonstrates that under asynchrony the measured total momentum is strictly higher than the algorithmic momentum (middle plot), as expected from theory;
%%closing the feedback loop (right plot) leads to total momentum matching the target momentum.
%%Closing the loop, as we will see, improves performance significantly.
%%%Note for asynchronous-parallel training, as the estimates and parameter tuning is unstable in the beginning when there are only a small number of iterations, we use initial learning $\frac{1}{\tau + 1}$ instead of $1.0$ to prevent overflow in the beginning. 
%%
%%%\begin{algorithm}[H]
%%%	\caption{\Asynctuner}
%%%	\begin{algorithmic}[1]
%%%%	\State Input: $\mu\gets0$, $\alpha \gets \frac{1}{\tau + 1}$, $\gamma\gets0.01, \tau$ (staleness)
%%%	\State Input: $\mu\gets0$, $\alpha \gets 0.0001$, $\gamma\gets0.01, \tau$ (staleness)
%%%	\For { $t\gets1$ to $T$}
%%%	\State $x_t\!\gets\!x_{t - 1} + \mu (x_{t - 1} - x_{t - 2} ) - \alpha \nabla_{S_t} f(x_{t - \tau - 1} )$
%%%	\State $\mu^*,\alpha \gets \Call{\tuner}{\nabla_{S_t} f(x_{t - \tau - 1} ), \beta}$ %(get momentum from the dynamic range)
%%%	\State $\hat{\mu_T} 
%%%					\gets \mathop{\mathsf{median}}\left(
%%%							\frac{x_{t - \tau} - x_{t - \tau-1} + \alpha \nabla_{S_{t-\tau-1}} f(x_{t - \tau - 1} )}
%%%							{x_{t - \tau-1} - x_{t - \tau-2}}
%%%					\right)$ \Comment{Measuring total momentum}
%%%	\State $\mu \leftarrow \mu + \gamma \cdot (\mu^* - \hat{\mu_T})$ \Comment{Closing the loop}
%%%	\EndFor
%%%\end{algorithmic}
%%%\label{alg:async-algo}
%%%\end{algorithm}
%%
%
%
%




\begin{algorithm}[h]
	\caption{\Asynctuner}
	\begin{algorithmic}[1]
%	\State Input: $\mu\gets0$, $\alpha \gets \frac{1}{\tau + 1}$, $\gamma\gets0.01, \tau$ (staleness)
	\State Input: $\mu\gets0$, $\alpha \gets 0.0001$, $\gamma\gets0.01, \tau$ (staleness)
	\For { $t\gets1$ to $T$}
	\State $x_t\!\gets\!x_{t - 1} + \mu (x_{t - 1} - x_{t - 2} ) - \alpha \nabla_{S_t} f(x_{t - \tau - 1} )$
	\State $\mu^*,\alpha \gets \Call{\tuner}{\nabla_{S_t} f(x_{t - \tau - 1} ), \beta}$ %(get momentum from the dynamic range)
	\State $\hat{\mu_T} 
					\gets \mathop{\mathsf{median}}\left(
							\frac{x_{t - \tau} - x_{t - \tau-1} + \alpha \nabla_{S_{t-\tau-1}} f(x_{t - \tau - 1} )}
							{x_{t - \tau-1} - x_{t - \tau-2}}
					\right)$ \Comment{Measuring total momentum}
	\State $\mu \leftarrow \mu + \gamma \cdot (\mu^* - \hat{\mu_T})$ \Comment{Closing the loop}
	\EndFor
\end{algorithmic}
\label{alg:async-algo}
\end{algorithm}


\section{Model specification}
\label{sec:model_spec}
The model specification is shown in Table~\ref{tab:model_specification} for all the experiments in Section~\ref{sec:experiments}. 
CIRAR10 ResNet uses the regular ResNet units while CIFAR100 ResNet uses the bottleneck units. Only the convolutional layers are shown with filter size, filter number as well as the repeating count of the units. The layer counting for ResNets also includes batch normalization and Relu layers. The LSTM models are also diversified for different tasks with different vocabulary sizes, word embedding dimensions and number of layers.
\begin{table}
\vspace{1em}
\begin{small}
\centering
	\begin{tabular}{c@{\hskip 0.1in} c@{\hskip 0.1in} c@{\hskip 0.1in} c@{\hskip 0.1in} c@{\hskip 0.1in} c}
	\toprule
	network & \# layers & Conv 0 & Unit 1s & Unit 2s & Unit 3s \\
	%\midrule 
	\midrule
	CIFAR10 ResNet & 110 
	& $\left[\begin{array}{c c} 3 \times 3, & 4 \end{array} \right] $
	& $\left[\begin{array}{c c} 3 \times 3, & 4  \\ 3\times 3, & 4\end{array} \right]\times 6 $ 
	& $\left[\begin{array}{c c} 3 \times 3, & 8  \\ 3\times 3, & 8\end{array} \right]\times 6 $
	& $\left[\begin{array}{c c} 3 \times 3, & 16  \\ 3\times 3, & 16\end{array} \right]\times 6 $
	\\
	\midrule
	CIFAR100 ResNet & 164 
	& $\left[\begin{array}{c c} 3 \times 3, & 4 \end{array} \right] $
	& $\left[\begin{array}{c c} 1 \times 1, & 16  \\ 3\times 3, & 16 \\ 1 \times 1, & 64  \end{array} \right]\times 6  $
	& $\left[\begin{array}{c c} 1 \times 1, & 32  \\ 3\times 3, & 32 \\ 1 \times 1, & 128  \end{array} \right]\times 6  $
	& $\left[\begin{array}{c c} 1 \times 1, & 64  \\ 3\times 3, & 64 \\ 1 \times 1, & 256  \end{array} \right]\times 6  $ 
		\\
%	\midrule
%		CIFAR100 ResNext & 164 
%	& $\left[\begin{array}{c c} 3 \times 3, & 4 \end{array} \right] $
%	& $\left[\begin{array}{c c} 1 \times 1, & 16  \\ 3\times 3, & 16 \\ 1 \times 1, & 64  \end{array} \right]\times 6  $
%	& $\left[\begin{array}{c c} 1 \times 1, & 32  \\ 3\times 3, & 32 \\ 1 \times 1, & 128  \end{array} \right]\times 6  $
%	& $\left[\begin{array}{c c} 1 \times 1, & 64  \\ 3\times 3, & 64 \\ 1 \times 1, & 256  \end{array} \right]\times 6  $ 
%		\\
	\midrule
	\midrule
	network & \# layers & Word Embed. & Layer 1 & Layer 2 & Layer 3 \\
	\midrule
	TS LSTM & 2 & [65 vocab, 128 dim] & 128 hidden units & 128 hidden units & --  \\
	\midrule
	PTB LSTM & 2 & [10000 vocab, 200 dim] & 200 hidden units & 200 hidden units & -- \\
	\midrule
	WSJ LSTM & 3 & [6922 vocab, 500 dim] & 500 hidden units & 500 hidden units & 500 hidden units\\
%	\midrule
%	PTB Tied LSTM & 2 & [10000 vocab, 650 dim] & 650 hidden units & 650 hidden units & \\
	\bottomrule
	\end{tabular}
\end{small}
\caption{Specification of ResNet and LSTM model architectures.}
\label{tab:model_specification}
\end{table}


\section{Specification for synchronous experiments}
\label{sec:exp_spec}
In Section~\ref{subsec:sync_exp}, we demonstrate the synchronous experiments with extensive discussions. 
For the reproducibility, we provide here the specification of learning rate grids. The number of iterations as well as epochs, i.e. the number of passes over the full training sets, are also listed for completeness. For \tuner in all the experiments in Section~\ref{sec:experiments}, we uniformly use sliding window size $20$ for extremal curvature estimation and $\beta = 0.999$ for smoothing. For momentum SGD and Adam, we use the following configurations.
\begin{itemize}
	\item CIFAR10 ResNet
		\begin{itemize}
			\item $40$k iterations (${\sim} 114$ epochs)
			\item Momentum SGD learning rates $\{0.001, 0.01 \text{(best)}, 0.1, 1.0\}$, momentum 0.9
			\item Adam learning rates $\{0.0001, 0.001 \text{(best)}, 0.01, 0.1\}$
		\end{itemize}
	\item CIFAR100 ResNet
		\begin{itemize}
			\item $120$k iterations (${\sim} 341$ epochs)
			\item Momentum SGD learning rates $\{0.001, 0.01 \text{(best)}, 0.1, 1.0\}$, momentum 0.9
			\item Adam learning rates $\{0.00001, 0.0001\text{(best)}, 0.001, 0.01\}$
		\end{itemize}
%	\item CIFAR100 ResNext
%		\begin{itemize}
%			\item ${\sim}53$k iterations (${\sim} 40$ epochs)
%			\item Momentum SGD learning rates $\{0.001, 0.01 \text{(best)}, 0.1, 1.0\}$, momentum 0.9
%			\item Adam learning rates $\{0.00001, 0.0001\text{(best)}, 0.001, 0.01\}$
%		\end{itemize}
	\item PTB LSTM
		\begin{itemize}
			\item 30k iterations (${\sim} 13$ epochs)
			\item Momentum SGD learning rates $\{0.01, 0.1, 1.0 \text{(best)}, 10.0\}$, momentum 0.9
			\item Adam learning rates $\{0.0001, 0.001 \text{(best)}, 0.01, 0.1\}$
		\end{itemize}
	\item TS LSTM
		\begin{itemize}
			\item ${\sim}21$k iterations ($50$ epochs)
			\item Momentum SGD learning rates $\{0.05, 0.1, 0.5, 1.0 \text{(best)}, 5.0\}$, momentum 0.9
			\item Adam learning rates $\{0.0005, 0.001, 0.005 \text{(best)}, 0.01, 0.05\}$
			\item Decrease learning rate by factor 0.97 every epoch for all optimizers, following the design by~\citet{karpathy2015visualizing}.
		\end{itemize}
	\item WSJ LSTM
		\begin{itemize}
			\item ${\sim} 120$k iterations ($50$ epochs)
			\item Momentum SGD learning rates $\{0.05, 0.1, 0.5 \text{(best)}, 1.0, 5.0\}$, momentum 0.9
			\item Adam learning rates $\{0.0001, 0.0005, 0.001 \text{(best)}, 0.005, 0.01\}$
			\item Vanilla SGD learning rates $\{0.05, 0.1, 0.5, 1.0 \text{(best)}, 5.0\}$
			\item Adagrad learning rates $\{0.05, 0.1, 0.5 (\text{best}), 1.0, 5.0\}$
			\item Decrease learning rate by factor 0.9 every epochs after 14 epochs for all optimizers, following the design by~\citet{charniakparsing}.
		\end{itemize}
\end{itemize}

\section{Additional experiment results}
\label{sec:add_exp}
\subsection{Training losses on CIFAR10 and CIFAR100 ResNet}
In Figure~\ref{fig:loss_result_cifar}, we demonstrate the training loss on CIFAR10 ResNet and CIFAR100 ResNet. Specifically, \tuner can match the performance of hand-tuned momentum SGD, and achieves 1.93x and 1.38x speedup comparing to hand-tuned Adam respectively on CIFAR10 and CIFAR100 ResNet.
\begin{figure}
\centering
	\begin{tabular}{c c}
		\includegraphics[width=0.4\linewidth]{experiment_results/resnet/resnet_loss.pdf} &
		\includegraphics[width=0.4\linewidth]{experiment_results/resnet/resnet_bottleneck_loss.pdf}
	\end{tabular}
	\caption{
	Training loss for ResNet on 100-layer CIFAR10 ResNet (left) and 164-layer CIFAR100 bottleneck ResNet. }
	\label{fig:loss_result_cifar}
\end{figure}

%\subsection{Importance of momentum adaptivity}
%\label{sec:importance_momentum}
%To further emphasize the importance of momentum adaptivity in \tuner, we run YF on CIFAR100 ResNet and TS LSTM. In the experiments, \tuner tunes the learning rate. Instead of also using the momentum tuned by YF, we continuously feed prescribed momentum value $0.0$ and $0.9$ to the underlying momentum SGD optimizer which YF is tuning. In Figure~\ref{fig:cmp_fix_mom}, when comparing to \tuner with prescribed momentum 0.0 or 0.9, \tuner with adaptively tuned momentum achieves observably faster convergence on both TS LSTM and CIFAR100 ResNet. It empirically demonstrates the essential role of momentum adaptivity in \tuner.
%\begin{figure}
%\centering	
%\begin{tabular}{c c}
%	\includegraphics[width=0.4\linewidth]{experiment_results/tf_charrnn_train_loss_fix_mom.pdf} &
%	\includegraphics[width=0.4\linewidth]{experiment_results/resnet/resnet_bottleneck_loss_fix_mom_cmp.pdf}
%\end{tabular}
%\caption{Training loss comparison between \tuner with adaptive momentum and \tuner with fixed momentum value. This comparison is conducted on TS LSTM (left) and CIFAR100 ResNet (right).}
%\label{fig:cmp_fix_mom}
%\end{figure}


\subsection{Tuning momentum can improve Adam in asynchronous-parallel setting}
\begin{wrapfigure}[11]{r}{0.31\linewidth}
\vspace{-4.5em}
\begin{minipage}{1.0\linewidth}
	\begin{figure}[H]
		\includegraphics[width=\linewidth]{experiment_results/ptb/adam_stale_15_tuning.pdf}
			\vspace{-1.5em}
		\caption{Hand-tuning Adam's momentum under asynchrony.}
		\label{fig:adam_async_mom}
	\end{figure}
%	\vspace{-1.5em}
%	\begin{figure}[H]
%		\includegraphics[width=0.975\linewidth]{experiment_results/resnet/resnet_bottleneck_cmp_tuner_adam.pdf}
%%		\caption{Adam, \tuner and \asynctuner on CIFAR100 with 16 async. workers. Sync. baseline uses \tuner.}
%	\vspace{-1.5em}
%		\caption{Asynchronous performance on CIFAR100 ResNet.}
%		\label{fig:full_async_cmp}
%	\end{figure}
\end{minipage}	
\end{wrapfigure}
%\subsection{Asynchronous experiments}
%In this section, we evaulate \tuner in an asynchronous-parallel setting,
%where we focus on {\em statistical efficiency}: the number of iterations to reach a certain solution. 
%To that end, we run $M$ asynchronous workers on a single machine and force them to update the model in a round-robin fashion,
%i.e. the staled gradient is delayed for $(M-1)$ iterations.
%We demonstrate
%%(1) Adam suffers a convergence speed penalty due to not tuning momentum in asynchronous settings;
%(1) \asynctuner (cf.\ Section~\ref{sec:async_tuner}) improves the convergence of \tuner dramatically, which leads to
%(3) \asynctuner having much faster convergence than Adam. 
%\paragraph{State-of-the-art adaptive methods suffer from lack of momentum tuning} 
We conduct experiments on PTB LSTM with 16 asynchronous workers using Adam using the same protocol as in Section~\ref{sec:async_exp}.
Fixing the learning rate to the value achieving the lowest smoothed loss in Section~\ref{subsec:sync_exp}, we sweep the smoothing parameter $\beta_1$~\citep{kingma2014adam} of the first order moment estimate in grid $\{-0.2, 0.0, 0.3, 0.5, 0.7, 0.9\}$. $\beta_1$ serves the same role as momentum in SGD and we call it the momentum in Adam. Figure~\ref{fig:adam_async_mom} shows tuning momentum for Adam under asynchrony gives measurably better training loss. 
This result emphasizes the importance of momentum tuning in asynchronous settings and suggests that state-of-the-art adaptive methods can perform sub-optimally when using prescribed momentum.

\subsection{Accelerating \tuner with finer grain learning rate tuning}
\label{sec:boost_exp}
 As an adaptive tuner, \tuner does not involve manual tuning. It can present faster development iterations on model architectures than grid search on optimizer hyperparameters. In deep learning practice for computer vision and natural language processing, after fixing the model architecture, extensive optimizer tuning (e.g. grid search or random search) can further improve the performance of a model. A natural question to ask is can we also slightly tune \tuner to accelerate convergence and improve the model performance. Specifically, we can manually multiply a positive number, the learning rate factor, to the auto-tuned learning rate in \tuner to further accelerate. 
 
In this section, we empirically demonstrate the effectiveness of learning rate factor on a 29-layer ResNext (2x64d)~\citep{xie2016aggregated} on CIFAR10 and a Tied LSTM model~\citep{press2016using} with 650 dimensions for word embedding and two hidden units layers on the PTB dataset. 
%The architecture of the models are specified in Table~\ref{tab:model_specification} in Appendix~\ref{sec:model_spec}.
 	 When running \tuner, we search for the optimal learning rate factor in grid $\{\frac{1}{3}, 0.5, 1, 2(\text{best for ResNext} ), 3 (\text{best for Tied LSTM} ), 10\}$. 
 	 Similarly, we search the same learning rate factor grid for Adam, multiplying the factor to its default learning rate $0.001$. 
 	 To further strengthen the performance of Adam as a baseline, we also run it on conventional logarithmic learning rate grid $\{5e^{-5}, 1e^{-4}, 5e^{-4}, 1e^{-3}, 5e^{-3}\}$ for ResNext and $\{1e^{-4}, 5e^{-4}, 1e^{-3}, 5e^{-3}, 1e^{-2}\}$ for Tied LSTM. We report the best metric from searching the union of learning rate factor grid and logarithmic learning rate grid as searched Adam results.
 	 Empirically, learning factor $\frac{1}{3}$ and $1.0$ works best for Adam respectively on ResNext and Tied LSTM. 
% 	 To provide another conventional baseline, we also demonstrate the performance from Adam with learning rate search on logarithmic grids. In this grid, we pick the learning rate giving the best validation/test metric. More specifically, we use grid $\{5e^{-5}, 1e^{-4}(\text{best}), 5e^{-4}, 1e^{-3}, 5e^{-3}\}$ for ResNext and $\{1e^{-4}, 5e^{-4}, 1e^{-3} (\text{best}), 5e^{-3}, 1e^{-2}\}$ for Tied LSTM. 
 	 
As shown in Figure~\ref{fig:yf_boost}, with the searched best learning rate factor, \tuner can improve validation perplexity on Tied LSTM from $88.7$ to $80.5$, an improvement of more than $9\%$. Similarly, the searched learning rate factor can improve test accuracy from $92.63$ to $94.75$ on ResNext. More importantly, we can observe, with learning rate factor search on the two models, \tuner can achieve better validation metric than the searched Adam results. It demonstrates that finer-grain learning rate tuning, i.e. the learning rate factor search, can be effectively applied on \tuner to improve the performance of deep learning models.
 	
 	
\begin{figure}
\centering
\begin{tabular}{c c} 
 	\includegraphics[width=0.4\linewidth]{experiment_results/pytorch_tied_ptb_test_perp_boost_all_seed.pdf} &
 	\includegraphics[width=0.4\linewidth]{experiment_results/pytorch_cifar_test_acc_boost.pdf}
\end{tabular}
\caption{Validation perplexity on Tied LSTM and validation accuracy on ResNext. Learning rate fine-tuning using grid-searched factor can further improve the performance of \tuner in Algorithm~\ref{alg:basic-algo}. \tuner with learning factor search can outperform hand-tuned Adam  on validation metrics on both models.}
\label{fig:yf_boost}
\end{figure}






%\input{test_perp}
%%%%%%%%%%%%%%%%%%%%%%%%%%%%%%%%%%%%%%%%%%%%%%%%%%%%%%%%%%%%%%%%%%%%%%%%%%%%%%%%
%%%%%%%%%%%%%%%%%%%%%%%%%%%%%%%%%%%%%%%%%%%%%%%%%%%%%%%%%%%%%%%%%%%%%%%%%%%%%%%%
%% DELETE THIS PART. DO NOT PLACE CONTENT AFTER THE REFERENCES!
%%%%%%%%%%%%%%%%%%%%%%%%%%%%%%%%%%%%%%%%%%%%%%%%%%%%%%%%%%%%%%%%%%%%%%%%%%%%%%%%
%%%%%%%%%%%%%%%%%%%%%%%%%%%%%%%%%%%%%%%%%%%%%%%%%%%%%%%%%%%%%%%%%%%%%%%%%%%%%%%%
%\appendix
%\section{Do \emph{not} have an appendix here}
%
%\textbf{\emph{Do not put content after the references.}}
%%
%Put anything that you might normally include after the references in a separate
%supplementary file.
%
%We recommend that you build supplementary material in a separate document.
%If you must create one PDF and cut it up, please be careful to use a tool that
%doesn't alter the margins, and that doesn't aggressively rewrite the PDF file.
%pdftk usually works fine. 
%
%\textbf{Please do not use Apple's preview to cut off supplementary material.} In
%previous years it has altered margins, and created headaches at the camera-ready
%stage. 
%%%%%%%%%%%%%%%%%%%%%%%%%%%%%%%%%%%%%%%%%%%%%%%%%%%%%%%%%%%%%%%%%%%%%%%%%%%%%%%%
%%%%%%%%%%%%%%%%%%%%%%%%%%%%%%%%%%%%%%%%%%%%%%%%%%%%%%%%%%%%%%%%%%%%%%%%%%%%%%%%


\end{document}


% This document was modified from the file originally made available by
% Pat Langley and Andrea Danyluk for ICML-2K. This version was created
% by Iain Murray in 2018. It was modified from a version from Dan Roy in
% 2017, which was based on a version from Lise Getoor and Tobias
% Scheffer, which was slightly modified from the 2010 version by
% Thorsten Joachims & Johannes Fuernkranz, slightly modified from the
% 2009 version by Kiri Wagstaff and Sam Roweis's 2008 version, which is
% slightly modified from Prasad Tadepalli's 2007 version which is a
% lightly changed version of the previous year's version by Andrew
% Moore, which was in turn edited from those of Kristian Kersting and
% Codrina Lauth. Alex Smola contributed to the algorithmic style files.
